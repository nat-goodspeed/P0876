
Every C++ function that has been called, but has neither returned nor
suspended, requires some storage for its variables with automatic storage
duration [basic.stc.auto]. Typically this storage also contains bookkeeping
data such as the location of the function's most recent caller, to support
the \cpp{return} statement [stmt.return]. This storage is called the
function's \emph{activation record} or \emph{activation frame}.

Activation frames for active functions are logically kept in a \emph{function
stack}, or simply \emph{stack}, as discussed in Exceptions [except],
particularly sections Constructors and destructors [except.ctor], Handling an
exception [except.handle], The \cpp{std::terminate} function
[except.terminate] and The \cpp{std::uncaught\_exceptions()} function
[except.uncaught].

The term ``stack'' simply means that each function call causes the called
function to construct a new activation frame. Return from that function
destroys that activation frame, in last-in-first-out fashion. Each recursive
call to a function causes a new activation frame to be constructed even though
its previous activation frame(s) still exist.

The implementation of such a stack is implementation-dependent.
