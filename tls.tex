\abschnitt{multi-threading environment}

Member function \usessysstack returns \cpp{true} if the stack used by the
\fiber instance was created by the operating system (main application or thread
stack); otherwise \cpp{false} will be returned. Instance that use a stack
provided by the operating system must not be migrated to other threads!
\footnote{A possible implementation could mark the first stack frame by creating
a special marker (for instance \emph{0xBADCAFFEE} etc.) at a specific offset
in the first stack frame or use a special function name for the first function 
and walk the stack searching for these markers.}\\

To decide if a \fiber instance might safely resumed on  a thread \prevtid
returns the \cpp{std::thread::id} of the thread that has suspended the fiber. If
the fiber was not resumed for the first time, a default constructed
\cpp{std::thread::id} will be returned.\\

\fiber is TLS-agnostic - best practice related to TLS applies to fibers too
(see P0772R0.)\\

%A \fiber may not migrate between threads: it may only be resumed on the
%\thread on which it was launched.
%
%Certain \fiber instances may be migrated between threads, using special
%methods as described below. However, if \usessysstack returns \cpp{false}, that
%instance may not migrate between threads.
%
%\cpp{cross\_thread\_fiber\_context::resume()}
%and \cpp{cross\_thread\_fiber\_context::resume\_with()} may only be called on
%the same \thread on which the fiber represented by that \fiber was last
%running.
%
%If \usessysstack
%returns \cpp{true}, \cpp{cross\_thread\_fiber\_context::resume\_other\_thread()}
%and \cpp{cross\_thread\_fiber\_context::resume\_other\_thread\_with()} may be
%called on a different \thread than the one on which the fiber represented by
%that \fiber was last running. \fiber does not have either of these methods.
