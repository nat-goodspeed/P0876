\abschnitt{multi-threading environment}

Member function \canxtresume returns \cpp{false} if the stack represented by the
\fiber instance was created by the operating system (main application or thread
stack), and the caller is running on a different thread. When the stack
represented by the \fiber instance was created by \fiber's
constructor, \canxtresume returns \cpp{true}.\footnote{A possible
implementation could mark the first stack frame by creating a special marker
(for instance \emph{0xBADCAFFEE} etc.) at a specific offset in the first stack
frame or use a special function name for the first function and walk the stack
searching for these markers.} You must not attempt to resume an instance
representing a stack provided by the operating system on some other thread!

\canresume can be called to determine whether a given \fiber instance might
safely be resumed on a particular thread by calling \resume or \resumewith.

\fiber is TLS-agnostic - best practices related to TLS apply to fibers too
(see P0772R0.)

There could potentially be Undefined Behavior if:
\begin{itemize}
    \item code running on a fiber references \cpp{thread\_local} variables
    \item the compiler/runtime implementation caches a pointer
          to \cpp{thread\_local} storage on the stack
    \item that fiber is suspended, and
    \item the suspended fiber is resumed on a different thread.
\end{itemize}

The cached TLS pointer is now pointing to storage belonging to some other
thread. If the original thread terminates before the new thread, the cached
TLS pointer is now dangling.

For a runtime that caches TLS pointers in such fashion, an implementation
of \xtresume or\\
\xtresumewith could conceivably walk the suspended stack,
patching cached pointers.
