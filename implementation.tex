\abschnitt{possible implementation strategies}\label{implementations}

\zs{This proposal does \so{NOT} want to standardize a special implementation or
calling convention!}
\xspace\\

Modern \bfs{micro-processors} are \bfs{register machines}; the content of
processor registers represent the execution context of the program at a given
point in time.\\
\bfs{Operating systems} maintain for each process all relevant data (execution
context, other hardware registers etc.) in the process table.\\
Operating system's \bfs{CPU scheduler} periodically suspends and resumes
processes in order to share CPU time between multiple processes. When a process
is suspended, its execution context (processor registers, instruction pointer,
stack pointer, ...) is stored in the associated process table entry. On
resumption, the CPU scheduler loads the execution context into the CPU and the
process continues to be executed.\\
The CPU scheduler does a \bfs{full context switch}, beside of preserving
the execution context (complete CPU state), the cache has to be invalidated and
the memory map modified.\\
A kernel-level context switch is several orders of magnitude slower than a
context switch at user-level\cite{Tanenbaum2009}.

\uabschnitt{fiber preserves the complete CPU state} This strategy tries to
preserve the complete CPU state, e.g. all CPU registers. This requires, that the
code identifies the concrete micro-processor type and supported processor
features. For instance the x86-architecture has several flavours of extensions
like MMX, SSE1-4, AVX1-2, AVX-512.\\
Depending on the detected processor features, implementations of certain
functionality are switched on or off. The CPU scheduler in the operating system
uses those informations for the context switch  between processes.\\
A fiber implementation using this strategy needs such a detection mechanism
too (equivalent to swapper/\cpp{system_32()} in the Linux kernel).\\
Beside of the complexity of such detection mechanisms, preserving the complete
CPU state for each fiber switch is expensive.

\zs{A context switch facility that preserves the complete CPU state like a
operating system is possible but impractical for the user-land.}

\uabschnitt{fiber switch using the calling convention}\label{callingconvention}
For \fiber, not all registers are required to be preserved because the fiber
switch is effected by a visible function call. It need not be undetectable like
an operating-system context switch; it only needs to be as transparent as a call
to any other function. The calling convention -- the part of the ABI that
specifies how a function's arguments and return values are passed -- determines
which subset of micro-processor registers must be preserved by the called
subroutine.\\

The \bfs{calling convention}\cite{SYSVABI} of \bfs{SYSV ABI} for \bfs{x86\_64}
architecture determines that general purpose registers R12, R13, R14, R15, RBX
and RBP must be preserved by the sub-routine - the first arguments are passed
to functions via RDI, RSI, RDX, RCX, R8 and R9 and return values are stored in
RAX, RDX.\\
In fact, for \resume the \bfs{general purpose registers} (R12-R15, RBX and RBP)
specified by the calling convention are preserved. In addition, the \bfs{stack
pointer} and \bfs{instruction pointer} are preserved and exchanged too -- thus,
from the point of view of calling code, \resume behaves like an ordinary
function call.\\
In other words, \resume acts on the level of a simple function invocation --
with the same performance characteristics (in terms of CPU cycles).\\

This technique is used in \bcontext\cite{bcontext} which acts as building block
for \fbfibers. The \fbfibers\xspace framework itself is the basis of many
critical applications at Facebook. For instance in \fbmcrouter\cite{fbmcrouter}
and some other Facebook services/libraries like ServiceRouter (routing framework
for \fbthrift\cite{fbthrift}), Node API (graph ORM API for graph databases) ...

\uabschnitt{in-place substitution at compile time} During the code generation
the compiler could inject the assembler code responsible for the fiber switch
directly in the function that calls \resume. This saves an extra indirection
(JMP + PUSH/MOV of certain registers used to invoke \resume).

\uabschnitt{CPU state at the stack} Because each fiber has to preserve CPU
registers at suspension and to load those registers at resumption, some storage
is required.\\
Instead of allocating extra memory for each fiber, a implementation could use
the stack by simply advancing the stack pointer at suspension and pushing the
CPU registers (CPU state) onto the stack (owned by the suspending fiber). When
the fiber is resumed, the fiber pops the values from its stack and loads them
into the appropriate registers.\\

This strategy works because only a resumed fiber creates new stack frames
(advancing the stack pointer). If a fiber is suspended it is save to keep the
CPU state on its stack.\\

Using the stack as storage for the CPU state has the advantage that \fiber needs
only to aggregate a pointer to the stack location (memory footprint is equal to
a pointer).\\
Section \nameref{synthesizing} describes how global variables are prevented
by synthesizing a fiber from the active fiber (execution context) and passing
this synthesized fiber (representing the now suspended fiber) into the resumed
fiber. Using the stack as storage makes this mechanism very easy to implement
\footnote{The implementation of \bcontext\cite{bcontext} utilizes this
technique.}.
Inside \resume the code pushes the relevant CPU registers onto the stack and
creates from the resulting stack address a new \fiber. This instance is then
passed into the resume fiber as parameter if the resumed fiber is started for the
first time or in-place assigned to the \fiber instance  that has previously
suspended the resumed fiber (see example in \nameref{synthesizing}).\\

\zs{Using fiber's stack as storage for the CPU state is efficient because no
additional allocations and deallocations are required.}
