\abschnitt{stack destruction}\label{destruction}

On construction of a \fiber a stack is allocated. When the \entryfn returns,
the stack will be destroyed. If the function has not yet returned,
the \fiber object representing that fiber must not be destroyed.
%% and the
%% destructor of the \fiber object representing that context is called,
%% the stack will be unwound and destroyed.
%% the calling program is responsible for unwinding the stack.

%%The fiber's \cancelfn is used to trigger cleanup.
%%
%%Consider a running fiber \cpp{f2} that destroys the \fiber object
%%representing \cpp{f1}.
%%
%%\cpp{f1}'s destructor, running on \cpp{f2}, implicitly calls member-function
%%\resumewith, passing the fiber's \cancelfn as
%%argument. Fiber \cpp{f1} will be temporarily resumed and the \cancelfn is
%%invoked.
%%
%%The \cancelfn must communicate to fiber \cpp{f1} the need to terminate. It
%%might throw an exception. It might set a distinguished value in some object
%%tested by code on \cpp{f1}. In any case, fiber \cpp{f2} remains suspended
%%in \cpp{f1}'s destructor until \cpp{f1}'s \entryfn returns (the \fiber
%%object synthesized for) \cpp{f2} -- or until \cpp{f2} is explicitly resumed
%%in some other way.

%% Function \unwindfib caches an object of type \fiber that
%% represents \cpp{f2}, then unwinds \cpp{f1}'s stack
%% (walking the stack and destroying automatic variables in reverse order of
%% construction).
%% The first frame on \cpp{f1}'s stack, the one created by \fiber's constructor,
%% stops the unwinding. It terminates \cpp{f1} by returning
%% \cpp{f2}. Control is returned to \cpp{f2} and \cpp{f1}'s
%% stack gets deallocated.

%% The stack on which \main is executed, as well as the stack implicitly
%% created by \thread's constructor, is allocated by the operating
%% system. Such stacks are recognized by \fiber, and are not deallocated by its
%% destructor.
