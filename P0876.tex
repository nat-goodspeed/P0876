%//////////////////////////////////////////////////////////////////////////////

\documentclass[fontsize=10pt,paper=A4,pagesize,DIV=15]{scrartcl}

\usepackage[T1]{fontenc}
\usepackage[utf8]{inputenc}
\usepackage[american]{babel}        % required for ISO dates
\usepackage[iso,american]{isodate}  % ISO format of dates
\usepackage[final]{listings}        % code listings
\usepackage{booktabs}               % fancy tables
\usepackage[color]{changebar}       % changebars for large inserted passages
\usepackage{longtable}              % auto breaking tables
\usepackage{ltcaption}              % fix captions for long tables
\usepackage{relsize}                % provide relative font size changes
%\usepackage{underscore}             % remove special status of '_' in ordinary text
%\usepackage{verbatim}               % improved verbatim environment
\usepackage{parskip}                % handle non-indented paragraphs "properly"
\usepackage{array}                  % new column definitions for tables
\usepackage[normalem]{ulem}         % underline commands
\usepackage{xcolor}                 % driver-independent color extensions
\usepackage{amsmath}                % mathematical symbols
\usepackage{mathrsfs}               % mathscr font
\usepackage{xspace}                 % inserts a space to replace one "eaten" by TeX
\usepackage[final]{microtype}       % micro-typographic extensions introduced by pdfTeX
\usepackage{xstring}                % manipulating strings
\usepackage{fixme}                  % collaborative annotations
\usepackage{multicol}               % intermix single and multiple columns
\usepackage{perpage}                % counter reset at every page boundary
\usepackage{palatino}               % Adobe Palatino font
\usepackage{overcite}               % citations
\usepackage{boxedminipage}          % framed mini-pages
\usepackage{fancyhdr}               % control of page headers and footers
\usepackage{soul}                   % hyphenatable spacingout), underlining, striking out, et.
\usepackage{svg}                    % SVG pictures
\usepackage{tikz}                   % creating PS and PDF graphics
\usetikzlibrary{arrows,automata}

\cbcolor{green}

\usepackage[pdftex,
            pdftitle    = {fibers without scheduler},
            pdfsubject  = {},
            pdfauthor   = {Oliver Kowalke},
            pdfkeywords = {C++,callcc,call/cc,context,continuation,coroutine,execution,fiber,fiber_context,switch,P0099,P0534,P0876},
            bookmarks=true,
            bookmarksnumbered=true,
            pdfpagelabels=true,
            pdfpagemode=UseOutlines,
            pdfstartview=FitH,
            linktocpage=true,
            colorlinks=true,
            linkcolor=blue,
            plainpages=false
           ]{hyperref}

%//////////////////////////////////////////////////////////////////////////////

\setlength{\parindent}{0pt} 
\renewcommand\sfdefault{phv}

\makeatletter
    \renewcommand*\l@subsection{\@dottedtocline{2}{0em}{2.3em}}
    \renewcommand*\l@subsection{\@dottedtocline{3}{0em}{3.2em}}
    \renewcommand{\tableofcontents}{\@starttoc{toc}}
\makeatother

\MakePerPage{footnote}
\renewcommand*{\thefootnote}{\fnsymbol{footnote}}

\newcommand{\pdfimg}[1]{\pdfximage{pics/#1}\pdfrefximage\pdflastximage}
\newcommand{\img}[1]{\mbox{\pdfimg{#1}}}
\newcommand{\imgc}[1]{\begin{center}\img{#1}\end{center}}
\newcommand{\graph}[1]{\input{graphs/#1}}
\newcommand{\graphc}[1]{\begin{center}\graph{#1}\end{center}}
\newcommand{\bfs}[1]{{\bfseries #1}}
\newcommand{\zs}[1]{\begin{boxedminipage}[t]{16.8cm}\bfs{#1}\end{boxedminipage}}

\newcommand{\cpp}[1]{{\lstinline[
		basicstyle=\ttfamily\small\color{black},
        breakatwhitespace=true,
        breaklines=true,
        captionpos=b,
        columns=flexible,
        commentstyle=\ttfamily\color{red},
        keepspaces=true,
        keywordstyle=\ttfamily\color{blue},
        language={C++},
        morekeywords={},
        showspaces=false,
        showstringspaces=false,
        showtabs=false,
        stringstyle=\ttfamily\color{magenta}
] !#1!}\xspace}
\newcommand{\cppf}[1]{\lstinputlisting[
		basicstyle=\ttfamily\small\color{black},
        breakatwhitespace=true,
        breaklines=true,
        captionpos=b,
        columns=flexible,
        commentstyle=\ttfamily\color{red},
        keepspaces=true,
        keywordstyle=\ttfamily\color{blue},
        language={C++},
        morekeywords={},
        showspaces=false,
        showstringspaces=false,
        showtabs=false,
        stringstyle=\ttfamily\color{magenta}
] {code/#1.cpp}}
\newcommand{\cppfl}[1]{\lstinputlisting[
		basicstyle=\ttfamily\small\color{black},
        breakatwhitespace=true,
        breaklines=true,
        captionpos=b,
        columns=flexible,
        commentstyle=\ttfamily\color{red},
        keepspaces=true,
        keywordstyle=\ttfamily\color{blue},
        language={C++},
        morekeywords={},
        numbers=left,
        showspaces=false,
        showstringspaces=false,
        showtabs=false,
        stringstyle=\ttfamily\color{magenta}
] {code/#1.cpp}}

\newcommand{\dtor}{\cpp{\~fiber\_context()}}
\newcommand{\main}{\cpp{main()}}
\newcommand{\fiber}{\cpp{std::fiber\_context}}
\newcommand{\op}{\cpp{operator()()}}
\newcommand{\opbool}{\cpp{operator bool()}}
\newcommand{\resume}{\cpp{resume()}}
\newcommand{\resumewith}{\cpp{resume\_with()}}
\newcommand{\xtresume}{\cpp{resume\_from\_any\_thread()}}
\newcommand{\xtresumewith}{\cpp{resume\_from\_any\_thread\_with()}}
\newcommand{\canresume}{\cpp{can\_resume()}}
\newcommand{\canxtresume}{\cpp{can\_resume\_from\_any\_thread()}}
\newcommand{\thread}{\cpp{std::thread}}
\newcommand{\Currthread}{The calling thread\xspace}
\newcommand{\currthread}{the calling thread\xspace}
\newcommand{\lastthread}{the thread on which the fiber represented by \cpp{*this} was most recently run}
\newcommand{\unwindex}{\cpp{std::unwind\_exception}}
\newcommand{\unwindfib}{\cpp{std::unwind\_fiber()}}

\newcommand{\sym}{\emph{symmetric}\xspace}
\newcommand{\asym}{\emph{asymmetric}\xspace}
\newcommand{\entryfn}{\emph{entry-function}}

\newcommand{\abschnitt}[1]{\addcontentsline{toc}{subsection}{#1}\subsection*{#1}}
\newcommand{\uabschnitt}[1]{\paragraph*{#1}}

\newcommand{\tsabschnitt}[1]{\subsection[]{#1}}
\newcommand{\tsuabschnitt}[1]{\subsubsection[]{#1}}
\newcommand{\tsparagraph}[1]{\paragraph[]{#1}}

\newcommand{\tsnote}[1]{[ \emph{Note:} {#1} -- \emph{end note} ]}

\newcommand{\bcontext}{
        \href{http://www.boost.org/doc/libs/release/libs/context/doc/html/index.html}
        {\emph{Boost.Context}}}
\newcommand{\bcoroutine}{
        \href{http://www.boost.org/doc/libs/release/libs/coroutine2/doc/html/index.html}
        {\emph{Boost.Coroutine2}}}
\newcommand{\bfiber}{
        \href{http://www.boost.org/doc/libs/release/libs/fiber/doc/html/index.html}
        {\emph{Boost.Fiber}}}
\newcommand{\fbmcrouter}{
        \href{https://code.facebook.com/posts/296442737213493/introducing-mcrouter-a-memcached-protocol-router-for-scaling-memcached-deployments}
        {\emph{mcrouter}}}
\newcommand{\fbfibers}{
        \href{https://github.com/facebook/folly/tree/master/folly/fibers}
        {\emph{folly::fibers}}}
\newcommand{\fbthrift}{
        \href{https://github.com/facebook/fbthrift}
        {\emph{Thrift}}}
\newcommand{\synca}{
        \href{https://github.com/gridem/Synca}
        {\emph{Synca}}}

\def\Sec#1[#2]#3{%
\ifcase#1\let\s=\chapter
      \or\let\s=\section
      \or\let\s=\subsection
      \or\let\s=\subsubsection
      \or\let\s=\paragraph
      \or\let\s=\subparagraph
      \fi%
\s[#3]{#3\hfill[#2]}\label{#2}}

\newcounter{SectionDepthBase}
\newcounter{scratch}

\def\rSec#1[#2]#3{%
\setcounter{scratch}{#1}                      
\addtocounter{scratch}{\value{SectionDepthBase}}
\Sec{\arabic{scratch}}[#2]{#3}}


%//////////////////////////////////////////////////////////////////////////////

\begin{document}
\small
\begin{tabbing}
    Document number: \= P0876R21\\
    Date:            \> 2025-07-13\\
    Author:          \> Oliver Kowalke (oliver.kowalke@gmail.com)\\
                     \> Nat Goodspeed (nat.cognitoy@gmail.com)\\
    Audience:        \> LWG, CWG\\
\end{tabbing}

\section*{\emph{fiber\_context} - fibers without scheduler}

%//////////////////////////////////////////////////////////////////////////////

\tableofcontents

%//////////////////////////////////////////////////////////////////////////////

\abschnitt{abstract}

This paper addresses concerns, questions and suggestions from the past meetings.
The proposed API supersedes the former proposals N3985\cite{N3985},
P0099R1\cite{P0099R1}, P0534R3\cite{P0534R3}, P0876R0\cite{P0876R0}
and P0876R2\cite{P0876R2}.\\
Because of name clashes with \emph{coroutine} from coroutine TS,
\emph{execution context} from executor proposals and \emph{continuation} used
in the context of \cpp{future::then()}, the committee has indicated
that \emph{fiber} is preferable. However, given the foundational, low-level
nature of this proposal, we choose \emph{fiber\_context}, leaving the
term \emph{fiber} for a higher-level facility built on top of this one.\\

A previous revision of this proposal suggested \emph{fiber\_context}, but SG1
felt that use of the term ``context'' posed potential confusion. The name 
\emph{basic\_fiber} has also been suggested. Consider that the name is not yet
final.\\

The minimal API enables stackful context switching \bfs{without} the need for a
\bfs{scheduler}. The API is suitable to act as building-block for high-level
constructs such as stackful coroutines as well as cooperative multitasking
(aka user-land/green threads that incorporate a \bfs{scheduling facility}).\\

Informally within this proposal, the term \emph{fiber} is used to denote the
lightweight thread of execution launched and represented by the first-class
object \fiber.

\abschnitt{Recent WG21 History}\label{wg21_history}

In St. Louis in June 2024,
\href{https://docs.google.com/document/d/1ebdFai3h2Y4g5NayNf_pG6Gl2qidYdF6G-smMWNw-Go/edit?tab=t.0#heading=h.9aqgmrf0hvh1}{LWG tentatively approved}
P0876 Library wording.

In Tokyo in March 2024, CWG finished initial P0876 Core wording review, with
\href{https://wiki.edg.com/bin/view/Wg21tokyo2024/CoreWorkingGroup#D0876R16_fiber_context}{one requested change}:
that P0876 mandate per-fiber exception state. That required EWG approval.

In St. Louis in June 2024,
\href{https://wiki.edg.com/bin/view/Wg21stlouis2024/NotesEWGP0876}{EWG approved} the change:

\begin{table}[ht]
\begin{tabular}{|r|r|r|r|r|} % right-just columns (5 columns)
\hline %inserts horizontal line
SF & F & N & A & SA \\ [0.5ex] % inserts table heading
\hline % inserts single horizontal line
6 & 8 & 3 & 0 & 0 \\ % [1ex] % [1ex] adds vertical space
\hline %inserts single line
\end{tabular}
\end{table}

However, EWG did not forward P0876 back to CWG, requesting implementation
experience with the proposed change.

In Wrocław in November 2024, Nat Goodspeed presented implementation experience
with libstdc++.
\href{https://wiki.edg.com/bin/view/Wg21wroclaw2024/NotesEWGP0876}{Microsoft requested}
time to consult the backend team. EWG agreed to defer to Hagenberg.

In Hagenberg in February 2025, late in the week,
\href{https://lists.isocpp.org/ext/2025/02/25138.php}{Microsoft conceded} that
per-fiber exception state is implementable with the MSVC runtime (while voicing
performance concerns). Unfortunately this response arrived so late that EWG
ran out of time without considering P0876.

In Sofia in June 2025,
\href{https://docs.google.com/document/d/1wItX212LurEjkJK53XxnxumciqTRzxTSTJ48GwFihpE/edit?tab=t.0#heading=h.zc6lxwkrnobj}{EWG forwarded P0876}
back to CWG and LWG for inclusion in C++26:

\begin{table}[ht]
\begin{tabular}{|r|r|r|r|r|} % right-just columns (5 columns)
\hline %inserts horizontal line
SF & F & N & A & SA \\ [0.5ex] % inserts table heading
\hline % inserts single horizontal line
10 & 14 & 4 & 5 & 1 \\ % [1ex] % [1ex] adds vertical space
\hline %inserts single line
\end{tabular}
\end{table}

But both CWG and LWG ran out of time in Sofia without considering P0876,
thereby postponing it to C++29.

Concerning a <feature> that fails to make the deadline for C++<NN>,
\href{https://www.open-std.org/jtc1/sc22/wg21/docs/papers/2024/p1000r6.pdf}{P1000R6}
says:
``Just wait a couple more meetings and C++<NN+3> will be open for business and <feature> can be
the first thing voted into the C++<NN+3> working draft.''

This is the promise of the train model. It matters to all of us that the train
model works as promised.

\abschnitt{Revision History}\label{history}
This document supersedes P0876R18.

\uabschnitt{Changes since P0876R18}

\begin{itemize}
    \item Move exception state test programs to Appendices.
    \item Link Boost.Context patch that produces correct fiber-specific
          exception behavior on Windows and Linux using libstdc++.
\end{itemize}

\uabschnitt{Changes since P0876R17}

\begin{itemize}
    \item Distinguish between a \emph{prepared} and a \emph{suspended} fiber.
    \item Distinguish the two context switches implied by entry to, and return
          from, \resumewith.
    \item Remove \cpp{current\_exception\_within\_fiber()}, which became moot
          in P0876R17.
\end{itemize}

\uabschnitt{Changes since P0876R16}

\begin{itemize}
    \item Update to reference N4981.
    \item Add \cpp{<fiber\_context>} header file to headers table.
    \item Remove \resumewith ``Case A'' and ``Case B'' in favor of
          nested bullet lists. Fix a bug in definition of internal-resume.
    \item Revert \resumewith\xspace\returns and \except clauses to R15
          structure, eliminating ``Case C'' and ``Case D''.
    \item Use scoped exposition-only terms \emph{calling fiber}, \emph{target
          fiber} and \emph{previous fiber} instead of quoting the phrases.
          Give previous fiber definition its own bullet.
    \item Eliminate \emph{internal-resume} parameter \cpp{after}, also
          definitions of \cpp{after\_entry\_copy}, \cpp{after\_stack\_copy}
          and \cpp{after\_deleter\_copy}. Describe \emph{internal-resume} in
          terms of the currently running fiber.
    \item Explicitly state that \emph{internal-resume} is exposition only, and
          italicize references.
    \item Move predicate for first \emph{internal-resume} definition to start
          of bullet text. Don't state the inverse predicate for the second.
    \item Remove one level of bullet list nesting from the
          second \emph{internal-resume} definition. Sequence the bullet list
          by appending ``, then'' after each item.
    \item Per EWG in St. Louis, remove implementation defined meaning
          of \emph{currently handled exception} and \cpp{uncaught\_exceptions}.
          Now both are fiber-specific.
    \item Clarify explicit constructor \except clause.
\end{itemize}

\uabschnitt{Changes since P0876R15}

\begin{itemize}
    \item \null [fiber.context.overview] is now ``Overview'' instead of ``Preamble''.
    \item Make default \fiber constructor \cpp{= default}.
          Remove its section from member descriptions.
    \item In unary constructor, move ``Mandates'' before ``Constraints''. The
          ``Preconditions'' entry is actually a Constraint: move it and remove
          ``Preconditions''.
    \item In span constructor, stated ``Preconditions'' are actually ``Constraints''.
    \item In constructor descriptions, use less precise language about copying
          \cpp{entry}, \cpp{stack} and \cpp{deleter}. Add Note about them not
          being \fiber members. Move mention of stack to a Note.
    \item Rephrase \resumewith Note about emptying its \fiber object to avoid
          the appearance of a normative statement.
    \item Remove mention of ``legacy behaviour'' from
          \cpp{current\_exception\_within\_fiber()}.
    \item Remove mention of ``thread of execution'' from ``abstract,''
          ``control transfer mechanism'' and the section on \exfns.
    \item Simplify definitions of implicit fiber vs. explicit fiber.
    \item Add [intro.fibers] statement that a thread is always running one
          fiber, but can switch between fibers. This replaces the more
          detailed description of what happens when a fiber calls \anyresume.
    \item In [intro.fibers], hoist ``owning thread'' definition to its own
          paragraph 3 and clarify.
    \item Remove assertion that a fiber is an execution agent.
    \item Modify \stdclause{except.throw} paragraphs 2 and 4, and \stdclause{except.handle}
          paragraph 6, to constrain exception propagation to a fiber.
    \item Describe explicit fiber as being ``prepared,'' with a statement that
          it comes into existence on first resumption.
    \item Remove a few stray instances of ``may''.
    \item Move assertion that a received \fiber object could represent either
          an explicit fiber or an implicit fiber to a Note.
    \item Move assertion that no \fiber object represents a running fiber up
          to Overview.
    \item Use \cpp{successor} rather than the more generic \cpp{continuation}
          to reference the \fiber object returned by a terminating fiber.
    \item Remove nesting from \resumewith \except.
    \item Remove Note that the caller of \resumewith can detect whether the
          previous fiber has terminated: not necessarily.
    \item Hoist section on \exfns to have its own table of contents entry.
          Extend with examples of bad behavior when switching out of a catch
          block to a fiber which itself catches some exception before
          switching back to the original fiber.
    \item Remove explicit \cpp{delete} declarations of copy constructor and
          copy assignment: these are implicitly deleted.
    \item Since we want the constructor's \cpp{entry} and \cpp{deleter}
          parameters to support move-only objects,
          remove \emph{Cpp17CopyConstructible} requirements.
    \item For the same reason, state that \cpp{entry\_copy}
          and \cpp{deleter\_copy} are initialized rather than copied.
    \item Therefore ``any exception from initialization of \cpp{entry\_copy}''
          and the same for \cpp{deleter\_copy}.
    \item Remove mention of ``function call stack'' from constructor \except.
    \item \cpp{stack.data()} and \cpp{stack.size()} must meet implementation
          requirements, not the \cpp{span<byte>} itself.
    \item Remove \postcond \cpp{other.empty()} from move constructor and
          move assignment: these are implied by definition.
    \item Move statement about UB from stack overflow to \fiber Overview.
    \item Modify example about early destruction of exceptions to add sequence
          comments, highlight access to a destroyed exception object.
    \item Fix erroneous [fibercontext.mumble] references in class comments.
    \item Add green changebars for entirely new sections.
    \item Remove \cpp{std::} qualification from \cpp{decay\_t} in \effects.
    \item Remove the destructor Note about encouraging a fiber to terminate
          voluntarily.
    \item Clarify that \cpp{current\_exception\_within\_fiber()} is \true if
          \curex reports exceptions ``only'' within the current fiber.
          Remove \cpp{constexpr}: compiler can produce object code that might
          be linked with alternative runtimes.
    \item Remove ``.'' after ``;''.
    \item \resumewith\xspace\mandates\xspace\cpp{is\_invocable\_r<...>} is \true.
          Add periods to \mandates and \precond.
    \item Add \precond to span constructor that \cpp{deleter} must not throw.
          Remove cleanup exceptions from \resumewith \xspace \except. Remove
          ``before this point, no exceptions'' bullet in \effects.\
    \item \resumewith evaluates \cpp{invoke\_r(fn)}. Merge Notes about what
          its \cpp{returned} can be.
    \item Substantially rework \resumewith description. Break out and label
          the four cases: (target not yet entered, target previously
          suspended); (previous exited, previous called \resumewith). Use case
          labels in \effects, \returns and \except. Break out internal-resume
          operation because it's self-referential.
    \item Add span constructor \precond for \cpp{decay\_t<D>}
          meeting \emph{Cpp17MoveConstructible} requirements.
    \item Remove \canresume Note about ``can resume.''
    \item For \resume, use \cpp{std::identity} instead of identity lambda.
\end{itemize}

\uabschnitt{Changes since D0876R15}

\begin{itemize}
    \item Updated to reference N4971.
    \item Inserted a section to clarify relationship between threads and fibers.
    \item Borrowed ``single flow of control'' definition for ``fiber.''
    \item Added Note clarifying ``flow of control'' as state, with reference
    to \stdclause{stacktrace.general}.
    \item Changed stacktrace ``invocation sequence'' to reference ``fiber'' rather
    than ``thread of execution.''
    \item Changed ``thread'' definition to be the execution agent that runs fibers.
    \item Clarified that if a fiber terminates by returning an empty \fiber
          instance, \cpp{std::terminate} is called.
    \item Added \cpp{constexpr fiber\_context::current\_exception\_within\_fiber}.
    \item Removed definition of ``function call stack.''
    \item Removed change to definition of expression evaluation conflict.
    \item Removed Note about the second fiber in the program.
    \item Changed ``\fiber instance'' to ``\fiber object.''
    \item Changed ``method'' to ``member function.''
    \item Removed paragraph numbers from internal cross-references.
    \item Clarified editorial directives amongst not-green new text.
    \item Used ``fiber.context'' in stable labels.
    \item Changed the lone remaining preamble section in [fiber.context] from
    ``Empty vs. Non-Empty'' to ``Preamble.''
    \item Moved to ``Preamble'' the 1:1 relationship between non-empty \fiber
    objects and suspended fibers.
    \item Used ``Effects: Equivalent to \cpp{return <expression>}'' for
    \cpp{empty()} and \cpp{operator bool()}.
    \item Referenced \justmain instead of \main.
\end{itemize}

\uabschnitt{Changes since P0876R14}

\begin{itemize}
    \item Invoked ``blocks with forward progress guarantee delegation'' words
          of power for \resumewith, guaranteeing mutual exclusion.
    \item Fixed Mandates and Throws concerning the entry function and deleter
          passed to the implicit-stack or explicit-stack constructor.
    \item Cleaned up wording around initializing, assigning and testing the
          exposition-only \cpp{state} member.
    \item Dampened the optimism of the proposed feature-test macro.
\end{itemize}

\uabschnitt{Changes since P0876R13}

\begin{itemize}
    \item At LEWG's request, retracted changes to \cpp{uncaught\_exceptions()}
          and \cpp{current\_exception()}, instead clarifying that results may
          reflect exceptions on other fibers running on the current thread.
    \item Updated against draft standard N4958.
    \item Deleted ``User-Mode'' from new section title ``Cooperative Threads''
          and removed the explanatory paragraph.
    \item Removed \cpp{explicit} from the explicit-stack constructor.
    \item Added \cpp{system\_error}: \cpp{resource\_unavailable\_try\_again}
          to the \except clause of the implicit-stack constructor.
    \item Changed \cpp{bad\_alloc}
          to \cpp{system\_error}: \cpp{resource\_unavailable\_try\_again}
          in the \except clause of the explicit-stack constructor.
    \item Stated that the move constructor and move assignment operator empty
          the moved-from \fiber.
    \item Removed the \emptyfn precondition from assignment operator; instead
          added the same \cpp{(\! empty())} effect as for the destructor.
    \item Removed \resumewith references to ``execution context.'' Existing
          section 7.6.1.3 Function call \stdclause{expr.call} makes no mention
          of saving or restoring state.
    \item Removed bullets in \resumewith \returns and \except clauses
          regarding \resume, since they can be inferred from \resumewith and
          the trivial-lambda equivalence described for \resume.
    \item Removed the \remarks about concurrent calls from multiple threads
          from \canresume, leaving in place the editorial note about the
          intentional absence of \cpp{const}.
    \item Changed exposition-only \cpp{state} member from unspecified-type
          to \cpp{void*}.
    \item Sanitized stable names.
    \item Moved feature-test macro to appropriate section.
    \item Cleaned up the header-file synopsis.
    \item Grouped class members with forward references.
    \item Added \cpp{std::swap()} specialization.
    \item Added obtrusive paragraph numbers.
    \item Streamlined single-item dash lists.
    \item Changed \textit{Ensures} to \textit{Postconditions}.
    \item Changed template parameters from \cpp{typename} to \cpp{class}.
    \item Tweaked constructor \precond / \mandates.
    \item Clarified that \cpp{entry\_copy}, \cpp{stack\_copy}
          and \cpp{deleter\_copy} are not intended to be data members
          of \fiber.
    \item Streamlined initialization of these exposition objects.
    \item ``Instantiates a \fiber'' => ``Initializes \cpp{state}''
    \item \emptyfn returns \true => \emptyfn is \true, et al.
    \item Removed explicit-stack constructor \precond for stack size and
          alignment, since \except explicitly specifies exceptions for
          violations.
    \item Rephrased \effects of move constructor.
    \item Extracted ``Let'' statements from \effects to preceding paragraphs.
\end{itemize}

\uabschnitt{Changes since P0876R12}

\begin{itemize}
    \item Proposed that \cpp{uncaught\_exceptions()}
          and \cpp{current\_exception()} be specific to the current thread of
          execution.
    \item Specified that constructors \emph{decay-copy} the \entryfn.
    \item Changed \cpp{span<byte, N>} constructor param to
          simply \cpp{span<byte>}; also accepted deleter function, which it
          must \emph{decay-copy}.
    \item Specified constructor exceptions.
    \item Specified that destroying a non-empty \fiber
          calls \cpp{terminate()}.
    \item Clarified that when \resumewith is called, \emptyfn becomes
          true immediately.
    \item Introduced exposition-only \cpp{fiber\_context::state} member to
          streamline wording.
    \item Removed \cpp{concurrency\_v2} namespace.
    \item Changed ``Equivalent to'' to ``As-if''.
    \item Clarified Preconditions vs. Mandates.
\end{itemize}

\uabschnitt{Changes since P0876R11}

\begin{itemize}
    \item Removed \getsource, \gettoken, \reqstop and
          exposition-only \cpp{ssource} members.
    \item Added a \fiber constructor accepting a caller-provided
          uninitialized memory area for the new fiber's function call stack.
\end{itemize}

Bundling a \cpp{stop\_source} into \fiber presented implementability concerns.
Although each fiber (specifically, its function call stack) is itself a
persistent entity, the \fiber representing that fiber is not: a
new \fiber object is synthesized on every suspension. This
presents a problem: how does the code that suspends a fiber find its
associated \cpp{stop\_source} shared state?

A consumer wishing to pass a \cpp{std::stop\_token} to a new fiber can itself
instantiate \cpp{std::stop\_source}, obtain from it a \cpp{stop\_token} and
bind that \cpp{stop\_token} in a lambda passed to the \fiber
constructor. Accordingly, the \fiber API need not explicitly support that.

\uabschnitt{Changes since P0876R10}

\begin{itemize}
    \item Removed \cpp{cancel()} method and the \cancelfn constructor
          argument. Replaced with the \cpp{std::jthread} stop token handling
          API: \getsource, \gettoken and \reqstop. This simplifies examples by
          eliminating \cpp{launch()} and \cpp{assert\_on\_cancel}.
    \item Added a section exploring the relationship of \fiber to the larger
          C++ ecosystem.
    \item Reordered some sections to make the paper more accessible for new readers.
\end{itemize}

\uabschnitt{Changes since P0876R9}

\begin{itemize}
    \item Removed \xtresume, \xtresumewith, \xtcancel and \canxtresume, along
          with stated support for resuming a suspended fiber on some thread
          other than the one on which it was launched.
\end{itemize}

In Belfast, EWG came down strongly against cross-thread fiber resumption. The
most emphatic objection was that for a function referencing TLS, multiple
compilers cache TLS pointers on the function's stack frame. Resuming a fiber
containing that stack frame on some other thread would cause problems. In the
best case, the resumed function would merely reference TLS belonging to the
wrong thread -- but at some point the original thread will terminate, its TLS
will be destroyed, and the cached pointers will be left dangling.

With \fiber, any opaque function call might possibly suspend -- but
invalidating cached TLS pointers across every opaque function call is deemed
unacceptable overhead.

\uabschnitt{Changes since P0876R8}

\begin{itemize}
    \item Reinstated cancellation function constructor argument.
    \item Added \cpp{cancel()} and \cpp{cancel\_from\_any\_thread()} member
          functions.
    \item Re-removed \unwindfib.
\end{itemize}

SG1 directed P0876R9 to conform to the Cologne 2019 recommendations, with any
other changes proposed in a separate paper.

\uabschnitt{Changes since D0876R7}

\begin{itemize}
    \item Cancellation function removed from \fiber constructor.
    \item \unwindfib re-added, with implementation-defined behaviour.
    \item Added elaboration of \cpp{filament} example to bind cancellation
          function.
\end{itemize}

P0876R8 diverged from the recommendations of the second SG1 round in Cologne
2019. It did not introduce \cpp{cancel()} or \cpp{cancel\_from\_any\_thread()}
member functions. In fact it removed the \cancelfn constructor argument.

\fiber is intended as the lowest-level stackful context-switching API. Binding
a \cancelfn on the fiber stack is a flourish rather than a necessity. It adds
overhead in both space (on the fiber stack) and time (to traverse the stack to
retrieve the \cancelfn). For this API, it should suffice to pass the desired
\cancelfn to \anyresumewith. If it is important to associate a \cancelfn with
a particular fiber earlier in the lifespan of the fiber, a struct serves.

A more compelling reason to avoid constructing an explicit fiber with
a \cancelfn is that no implicit fiber has any such \cancelfn\xspace -- and the
consuming application cannot tell, a priori, whether a given \fiber instance
represents an explicit or an implicit fiber. If \this represents an
implicit fiber, what should the proposed \cpp{cancel()} member function do?

Passing a specific \cancelfn to \anyresumewith avoids that problem.

P0876R8 follows SG1 recommendation in making it Undefined Behaviour to destroy
(or assign to) a non-empty \fiber instance.

\unwindfib was reintroduced with implementation-defined behaviour to allow fiber
cleanup leveraging implementation internals. Its use was entirely optional (and
auditable). 

\uabschnitt{Changes since P0876R6}

\begin{itemize}
    \item Implicit stack unwinding (by non-C++ exception) removed.
    \item \unwindfib removed.
    \item Cancellation function added to \fiber constructor.
\end{itemize}

In Cologne 2019, SG1 took the position that:

\begin{itemize}
    \item The \fiber facility is not the only C++ feature that
          requires ``special'' unwinding (special function exit path).
    \item Such functionality should be decoupled from \fiber. It requires its
          own proposal that follows its own course through WG21 process.
    \item Depending on this (yet to be written) proposal would unduly delay
          the \fiber facility.
    \item For now, the \fiber facility should adopt a ``less is
          more'' approach, removing promises about implicit unwinding, placing
          the burden on the consumer of the facility instead.
    \item This leaves the way open for \fiber to integrate with
          a new, improved unwind facility when such becomes available.
\end{itemize}

The idea of making \fiber's constructor accept a cancellation function was
suggested to permit consumer opt-in to P0876R5 functionality where
permissible, or convey to the fiber in question by any suitable means the need
to clean up and terminate.

Requiring the cancellation function is partly because it remains unclear what
the default should be. This could be one of the questions to be answered by a
TS. Moreover, the absence of a default permits specifying later that the
default engages the new, improved unwind facility.

\uabschnitt{Changes since P0876R5}

\begin{itemize}
    \item \cpp{std::unwind\_exception} removed.
    \item \cpp{fiber\_context::can\_resume\_from\_any\_thread()} renamed to
      \canxtresume.
    \item \cpp{fiber\_context::valid()} renamed to \emptyfn with inverted
      sense.
    \item Material has been added concerning the top-level wrapper
      logic governing each fiber.
\end{itemize}

\unwindex was removed in response to deep
discussions in Kona 2019 of the surprisingly numerous problems surfaced by
using an ordinary C++ exception for that purpose.

Problems resolved by discarding \unwindex:
\begin{itemize}
    \item When unwinding a fiber stack, it is essential to know the subsequent
          fiber to resume. \unwindex therefore bound a \fiber. \fiber is
          move-only. But C++ exceptions must be copyable.
    \item It was possible to catch and discard \unwindex, with problematic
          consequences for its bound \fiber.
    \item Similarly, it was possible to catch \unwindex but not rethrow it.
    \item If we attempted to address the problem above by introducing a
          \unwindex operation to extract the bound \fiber, it became possible
          to rethrow the exception with an empty (moved-from) \fiber instance.
    \item Throwing a C++ exception during C++ exception unwinding terminates
          the program. It was possible for an exception implementation based
          on \cpp{thread\_local} to become confused by exceptions on different
          fibers on the same thread.
    \item It was possible to capture \unwindex with \cpp{std::exception\_ptr}
          and migrate it to a different fiber -- or a different thread.
\end{itemize}

\newpage
\abschnitt{P3620R0: Concerns with the proposed addition of fibers to C++26}

At a high level, P3620R0\cite{P3620R0} appears to argue that unless fibers are
appropriate for all use cases, they must not be available for any use case.
This ignores the industry experience cited in \nameref{low_level}.

Not every C++ feature is applicable to every environment. \cpp{breakpoint()}
is not generally found in production code. A library that writes to
\cpp{std::cerr} will cause problems for an application running in a windowed
environment that has no stderr file handle. A library that throws exceptions
is a poor choice for an application that forbids exceptions. A library that
creates \cpp{std::thread}s will cause trouble for an application that's not
expecting them.

\uabschnitt{Fibers are not lightweight threads}

P3620R0 states that operating system vendors have largely abandoned attempts
to support fibers as N:M threading, because operating system threads have more
state than it's feasible to manage with fibers.

\fiber does not claim to support lightweight threads. \fiber is a tool for
organizing the flow of control within an operating system thread. It does not
need to manage signals, signal masks or other facilities beyond the C++
abstract machine.

\uabschnitt{TLS}

P3620R0 notes that \tlocal storage is shared between all the fibers on a
thread. P3346R0\cite{P3346R0} proposed to modify \tlocal to mean
fiber-specific. This was rejected by SG1 in Wrocław.\cite{wroclawp3346}

This semantic can nonetheless be addressed by a higher-level library. For
instance, Boost.Fiber\cite{bfiber} provides
\href{https://www.boost.org/doc/libs/release/libs/fiber/doc/html/fiber/fls.html}
{\cpp{fiber\_specific\_ptr}}.

P3620R0 further claims that C++20 coroutines do not have this problem.
Actually, they do. If, on entry, a coroutine links an object into a linked
list anchored with a \cpp{static} or \tlocal pointer, then unlinks it on final
return, reaching that coroutine from different interleaved invocation
sequences will corrupt that linked list. This issue did not block adoption
of C++20 coroutines.

It may be worth noting that coroutines provide no entity analogous to a fiber.
It would not be straightforward to support chain-of-coroutines-local storage.

\uabschnitt{Deadlocks}

P3620R0 points out that switching fibers within a thread while holding a lock
may lead to accidental deadlock.

This semantic can be addressed by a higher-level library. For instance,
Boost.Fiber\cite{bfiber} provides fiber-aware synchronization primitives such as
\href{https://www.boost.org/doc/libs/release/libs/fiber/doc/html/fiber/synchronization/mutex_types.html}
{\cpp{boost::fibers::mutex}}.

C++20 coroutines have the same problem. This issue did not block adoption of
C++20 coroutines.

It would not be straightforward to support chain-of-coroutines-aware
synchronization primitives.

\abschnitt{\fiber and the larger C++ ecosystem}

\uabschnitt{higher-level libraries}

\nameref{low_level} enumerates a number of higher-level abstraction libraries
built upon the \bcontext\xspace implementation of the API proposed in this paper.
This is not an exhaustive list, but it suffices to illustrate that there is
widespread interest in this functionality.

The most significant point about this proposal is that, given \fiber, all
those libraries can be written in standard C++. They need not themselves be
integrated into the Standard.

Because it creates and switches between different function call stacks,
though, the \fiber facility cannot be written in portable C++. There is real
value to integrating this library into the Standard.

\emph{Boost.Context} is maintained by one individual to support the specific
set of processors and operating systems to which he has access. The \fiber
facility will ensure support in every implementation of the C++ runtime,
extending into the future.

Given the lively ecosystem of open-source libraries, it's possible that
standardizing \cpp{fiber\_context} could suffice. It is not essential that
WG21 must standardize additional higher-level libraries before the facility
would become useful. The uptake of \emph{Boost.Context} illustrates that the
community can make good use of \cpp{fiber\_context}.

However, the evolution of this proposal and the WG21 discussions thereof have
surfaced a number of interesting adjacencies.

\uabschnitt{cancellation}

Given C++ support for concurrency, in various forms, within a process,
cancellation of an asynchronous task remains a topic of widespread interest.
It has been much discussed, e.g. in P1677R2\cite{P1677R2},
P1820R0\cite{P1820R0} and P2175R0\cite{P2175R0}.

Previous revisions of this paper have proposed canceling a suspended fiber by
injecting an exception, e.g. using \fiber\cpp{::}\resumewith. A comparable
approach was rejected for \cpp{std::jthread}, although it's worth noting that
cooperative fibers differ in a very significant respect: every fiber suspends
at a well-defined point, namely a call to \resumewith.\footnote{Although
exception-based cancellation is not implicitly supported, a consumer of \fiber
may still explicitly pass to \resumewith an invocable that raises an exception
in the suspended fiber.}

Evolution of the exception mechanism itself\cite{P0709R4} may affect the
viability of using exceptions for cancellation.

That said, this paper proposes that \fiber
adopt \cpp{std::stop\_source}, \cpp{std::stop\_token} from the
Standard\cite{Standard}, section 33.3.

\uabschnitt{modules and optimizations}

Before modules, the only information the compiler could know about a function
in an external translation unit was what a human coder stated in the relevant
header file. But since the information in a module is prepared by the compiler
itself, a subsequent compile of a translation unit that imports that module
can know as much about each module function as it would if the function's
source code was found within the current translation unit.

This permits the compiler to infer and propagate attributes. If a function
neither contains a throw statement nor calls other functions, the compiler can
conclude that it doesn't throw. It can encode this information in the module
produced for that translation unit, so that subsequent compiles can make use
of the knowledge. If another function contains no throw statement and calls
only functions known not to throw, it too can be implicitly marked nothrow.

Similarly, when compiling a function that can never return, the compiler can
so indicate in the output module. Any caller whose code path leads
unconditionally to any such function can also be known never to return.

In much the same way, the module describing the
library's \fiber\cpp{::}\resumewith method can mark it as \emph{can-suspend}.
Then any caller of \resumewith will also be marked \emph{can-suspend}, and so
forth. The compiler can use this to improve its optimization tactics around
any call to a \emph{can-suspend} function.

(The \emph{can-suspend} characteristic of a \cpp{co\_await} coroutine function
is just as pervasive, but in that case the coder must manually propagate it.)

\uabschnitt{synchronization primitives}

The Standard\cite{Standard} provides an assortment of primitives for
synchronizing work between threads, e.g. sections 33.6, 33.7, 33.8, 33.9,
33.10. An essential behaviour of many such synchronization primitives is to
pause, or suspend, execution of the current thread until some external
condition is satisfied.

Such suspension is very different from fiber suspension as proposed in this
paper. This proposal neither requires nor implies a scheduler. A fiber
suspends by explicitly designating the next fiber to resume, either by passing
its \fiber to \resumewith or by returning that \fiber from its \entryfn.

C++ threads, in contrast, assume a thread scheduler, usually provided by the
operating system. Suspending a thread means passing control to the scheduler,
which reallocates CPU resources to other pending threads. At some future time,
the scheduler is responsible for directing some CPU core to resume the suspended
thread.

Fiber suspension as implemented by \fiber is independent of thread suspension.
Suspending the running fiber simply means directing the thread to run a
different fiber; the thread continues running. Conversely, suspending the host
thread (e.g. by invoking a synchronization primitive) means that \emph{no}
fiber is running on that thread.

A higher-level fiber-based library that emulates the \cpp{std::thread} API,
such as \bfiber\cite{bfiber}, necessarily implements a fiber scheduler,
permitting implicit fiber suspension. Standardizing such a library would raise
the interesting question of how to present fiber-aware synchronization
primitives.

A straightforward approach is to present a suite of fiber-aware
synchronization primitives distinct from, but analogous to, the thread-based
synchronization primitives.\footnote{This is the approach taken
by \emph{Boost.Fiber}.} An application using multiple fibers within a thread
would use fiber-aware synchronization primitives rather than thread-based
synchronization primitives. Evaluating a thread-based synchronization
primitive would suspend the entire thread, as usual, halting all fibers within
that thread.

It is tempting to contemplate modifying the semantics of the present suite of
synchronization primitives to make them fiber-aware. Naturally this is a
matter of some concern.

For purposes of this \fiber proposal, though, it is entirely moot.

\uabschnitt{Execution Agent Local Storage}

A similar question arises concerning variable storage duration. Should the
Standard introduce a fiber-specific storage duration, e.g. \cpp{fiber\_local},
analogous to \cpp{thread\_local}\cite{Standard}? (section 6.7.5.3 \bfs{Thread
storage duration})

The Standard defines the general term \emph{execution agent} (section
33.2.5.1) to allow for multiple kinds of parallelism. It seems reasonable to
assume that over time, new types of execution agents will be defined. Will we
want the Standard to present a new \cpp{xyz\_local} storage duration for each
new ``xyz'' execution agent type?

P0772R1\cite{P0772R1} notes that library code should not have to care what
kind of execution agent is running it. Already it's important to ensure that
library code avoids \cpp{static} variables because any such variable prohibits
calling that library from more than one thread. P0772R1 suggests a generalized
variable storage duration dynamically local to the innermost current execution
agent.

(The same consideration about library code impacts the above question about
presenting fiber-aware synchronization primitives.)

It's true that if:

\begin{itemize}
    \item a particular function relies on a \cpp{thread\_local} variable
    \item the function calls a function that resumes a different fiber
    \item the other fiber makes a different call to that same function, or to
          another function that modifies the same \cpp{thread\_local} variable
\end{itemize}

then on resumption of the original fiber, the function will observe the new
value for the \cpp{thread\_local} variable.

This is analogous to use of a \cpp{static} variable by multiple threads in the
same program -- though not as bad, since it doesn't produce race-related
Undefined Behaviour on top of correctness problems.

\cpp{std::thread} was introduced despite this problem because it's \emph{useful.}

Multiple C++ implementations cache a pointer to thread-local storage in the
stack frame of a function referencing TLS. If a suspended fiber were resumed
by a thread other than the one on which it previously ran, such cached TLS
pointers would point to TLS for the wrong thread. This is why such
cross-thread resumption is forbidden.

(This is the only optimization that has yet been surfaced by implementers as a
potentially problematic interaction with fibers.)

\uabschnitt{tooling} One particularly valuable consequence of adding \fiber to
the Standard will be to add fiber awareness to debuggers, performance
analyzers and other tools that inspect a running C++ program.

Such tools need only be aware of \fiber. They would \emph{not} need to be
further adapted to support higher-level libraries built on
the \cpp{fiber\_context} facility.

\newpage

\abschnitt{control transfer mechanism}

According to the literature\cite{Moura2009}, coroutine-like control-transfer
operations can be distinguished into the concepts of \sym and \asym
operations.

\uabschnitt{symmetric fiber} A symmetric fiber provides a single
control-transfer operation. This single operation requires that the control is
passed explicitly between the fibers.\\
\graphc{symm}

\cppfl{symmetric_op}

In the pseudo-code example above, a chain of fibers is created.\\
Control is transferred to fiber \cpp{f1} at line 15 and the lambda
passed to constructor of \cpp{f1} is entered. Control is transferred from
fiber \cpp{f1} to \cpp{f2} at line 12 and from \cpp{f2} to \cpp{f3} (line 9) and
so on. Fiber \cpp{f4} itself transfers control directly back to
fiber \cpp{f1} at line 3.

\uabschnitt{asymmetric fiber} Two control-transfer operations are part of
asymmetric fiber's interface: one operation for resuming (\resume) and one for
suspending (\cpp{suspend()}) the fiber. The suspending operation returns
control back to the calling fiber.\\
\graphc{asymm}

\cppfl{asymmetric_op}
In the pseudo code above execution control is transferred to fiber \cpp{f1} at
line 16. Fiber \cpp{f1} resumes fiber \cpp{f2} at line 13 and so on. At line 2
fiber \cpp{f4} calls its suspend operation \cpp{self::suspend()}. Fiber \cpp{f4}
is suspended and \cpp{f3} resumed. Inside the lambda, \cpp{f3} returns from
\cpp{f4.resume()} and calls \cpp{self::suspend()} (line 6). Fiber \cpp{f3} gets
suspended while \cpp{f2} will be resumed and so on ...\\
The asymmetric version needs \bfs{N-1 more} fiber switches than the variant
using symmetric fibers.\\

\zs{While asymmetric fibers establish a caller-callee relationship (strongly
coupled), symmetric fibers operate as siblings (loosely coupled).}\\
\newline

Symmetric fibers represent independent units of execution, making symmetric
fibers a suitable mechanism for concurrent programming. Additionally,
constructs that produce sequences of values (\emph{generators}) are easily
constructed out of two symmetric fibers (one represents the caller, the other
the callee).\\

Asymmetric fibers incorporate additional fiber switches as shown in the pseudo
code above. It is obvious that for a broad range of use cases, asymmetric
fibers are less efficient than their symmetric counterparts.\\
Additionally, the calling fiber must be kept alive until the called fiber
terminates. Otherwise the call of \cpp{suspend()} will be undefined behaviour
(where to transfer execution control to?).\\

\zs{Symmetric fibers are more efficient, have fewer restrictions (no
caller-callee relationship) and can be used to create a wider set of
applications (generators, cooperative multitasking, backtracking ...).}

\abschnitt{\fiber as a first-class object}

Because the symmetric control-transfer operation requires explicitly passing
control between fibers, fibers must be expressed as
\emph{first-class objects}.\\

Fibers exposed as first-class objects can be passed to and returned from
functions, assigned to variables or stored into containers. With fibers as
first-class objects, a program can \bfs{explicitly control the flow of
execution} by suspending and resuming fibers, enabling control to pass into a
function at exactly the point where it previously suspended.\\

\zs{Symmetric control-transfer operations require fibers to be first-class
objects. First-class objects can be returned from functions, assigned to
variables or stored into containers.}

\abschnitt{encapsulating the stack}\label{stackmgmt}

Each fiber is associated with a stack and is responsible for managing the lifespan
of its stack (allocation at construction, deallocation when fiber terminates). The
RAII-pattern\footnote{resource acquisition is initialisation} should apply.\\

Copying a \fiber must not be permitted!\\
If a \fiber were copyable, then its stack with all the objects allocated on it
must be copied too. That presents two implementation choices.
\begin{itemize}
    \item One approach would be to capture sufficient metadata to permit
          object-by-object copying of stack contents. That would require
          dramatically more runtime information than is presently available --
          and would take considerably more overhead than a coder might expect.
          Naturally, any one move-only object on the stack would prohibit
          copying the entire stack.
    \item The other approach would be a bytewise copy of the memory occupied
          by the stack. That would force undefined behaviour if any stack
          objects were RAII-classes (managing a resource via RAII pattern). When the first
          of the fiber copies terminates (unwinds its stack), the RAII class destructors
          will release their managed resources. When the second copy terminates, the same
          destructors will try to doubly-release the same resources, leading to undefined
          behavior.
\end{itemize}

\zs{
A fiber API must:
\begin{itemize}
    \item encapsulate the stack
    \item manage lifespan of an explicitly-allocated stack: the stack gets
          deallocated when \fiber goes out of scope
    \item prevent accidentally copying the stack
\end{itemize}
Class \fiber must be \emph{moveable-only}.\\
}

\abschnitt{invalidation at resumption}\label{invalidation}

The framework must prevent the resumption of an already running or terminated
(computation has finished) fiber.\\
Resuming an already running fiber will cause overwriting and corrupting the stack
frames (note, the stack is not copyable).  Resuming a terminated fiber will
cause undefined behaviour because the stack might already be unwound (objects
allocated on the stack were destroyed or the memory used as stack was already
deallocated).\\
As a consequence each call of \resume will invalidate the \fiber instance, i.e.
no valid instance of \fiber represents the currently-running fiber.\\
Whether a \fiber is valid or not can be tested with member function \opbool.\\
To make this more explicit, functions \resume, \resumewith, \xtresume
and\\
\xtresumewith are rvalue-reference qualified.
%If a fiber calls \cpp{f.resume()} then the  fiber is suspended and \cpp{f} is
%invalidated. When the fiber is resumed later, it returns from \cpp{f.resume()}
%and instance \cpp{f} references the calling fiber (the fiber that has resumed
%the current fiber).\\

The essential points:
\begin{itemize}
    \item regardless of the number of \fiber declarations, exactly one \fiber
          instance represents each suspended fiber
    \item no \fiber instance represents the currently-running fiber
\end{itemize}

Section \nameref{solution_gpub} describes how an instance of\\
\fiber is synthesized from the active fiber that suspends.\\

\zs{
A fiber API must:
\begin{itemize}
    \item prevent accidentally resuming a running fiber
    \item prevent accidentally resuming a terminated (computation has finished)
          fiber
    \item \resume, \resumewith, \xtresume and \xtresumewith are
          rvalue-reference qualified to bind on rvalues only
\end{itemize}
}

\abschnitt{problem: avoiding non-const global variables and
undefined behaviour}\label{problem_gpub}

According to \emph{C++ core guidelines}\cite{coreguidlines}, non-const global
variables should be avoided: they hide dependencies and make the
dependencies subject to unpredictable changes.

Global variables can be changed by assigning them indirectly using a pointer or
by a function call. As a consequence, the compiler can't cache the value of a
global variable in a register, degrading performance (unnecessary
loads and stores to global memory especially in performance critical loops).

Accessing a register is one to three orders of magnitude faster than accessing
memory (depending on whether the cache line is in cache and not invalidated by
another core; and depending on whether the page is in the TLB).

The order of initialisation (and thus destruction) of static global
variables is not defined, introducing additional problems with static
global variables.

\zs{A library designed to be used as building block by other higher-level
frameworks should avoid introducing global variables. If this API were
specified in terms of internal global variables, no higher level layer could undo
that: it would be stuck with the global variables.}

\uabschnitt{switch back to \emph{main()} by returning}
Switching back to \main by returning from the fiber function has two drawbacks:
it requires an internal global variable pointing to the suspended \main and
restricts the valid use cases.
\cppf{return_to_main}
For instance the generator pattern is impossible because the only way
for a fiber to transfer execution control back to \main is to terminate. But
this means that no way exists to transfer data (sequence of values) back and
forth between a fiber and \main.

\zs{Switching to \main only by returning is impractical because it limits the
applicability of fibers and requires an internal global variable pointing to
\main.}

\uabschnitt{static member function returns active \fiber}
P0099R0\cite{P0099R0} introduced a static member function\\
(\cpp{execution_context::current()}) that returned an instance of the active
fiber. This allows passing the active fiber \cpp{m} (for instance representing
\main) into the fiber \cpp{f} via lambda capture. This mechanism enables
switching back and forth between the fiber and \main, enabling a rich set of
applications (for instance generators).
\cppf{static_current}

But this solution requires an internal global variable pointing to the active
fiber and some kind of reference counting. Reference counting is needed because
\cpp{fiber\_context::current()} necessarily requires multiple instances of \fiber for the
active fiber. Only when the last reference goes out of scope can the fiber be
destroyed and its stack deallocated.
\cppf{multi_current}

Additionally a static member function returning an instance of the active fiber
would violate the protection requirements of sections \nameref{stackmgmt} and
\nameref{invalidation}. For instance you could accidentally attempt to resume
the active fiber by invoking \resume.
\cppf{resume_current}

\zs{A static member function returning the active fiber requires a reference
counted global variable and does not prevent accidentally attempting to resume
the active fiber.}

\abschnitt{solution: avoiding non-const global variables and
undefined behaviour}\label{solution_gpub}

\zs{The \emph{avoid non-const global variables} guideline has an important
impact on the design of the fiber API!}

\uabschnitt{synthesizing the suspended fiber}\label{synthesizing}
The problem of global variables or the need for a static member function
returning the active fiber can be avoided by \bfs{synthesizing} the
\bfs{suspended fiber} and passing it into the resumed fiber (as parameter when the
fiber is started the first time or returned from \resume).
\cppfl{synthesized_foo}

In the pseudo-code above the fiber \cpp{f} is started by invoking its member
function \resume at line 7. This operation suspends \cpp{foo}, invalidates
instance \cpp{f} and synthesizes a new fiber \cpp{m} that is passed as parameter
to the lambda of \cpp{f} (line 2).\\
Invoking \cpp{m.resume()} (line 3) suspends the lambda, invalidates \cpp{m} and
synthesizes a fiber that is returned by \cpp{f.resume()} at line 7. The
synthesized fiber is assigned to \cpp{f}. Instance \cpp{f} now represents the
suspended fiber running the lambda (suspended at line 3). Control is
transferred from line 3 (lambda) to line 7 (\cpp{foo()}).\\
Call \cpp{f.resume()} at line 8 invalidates \cpp{f} and suspends \cpp{foo()}
again. A fiber representing the suspended \cpp{foo()} is synthesized, returned
from \cpp{m.resume()} and assigned to \cpp{m} at line 3. Control
is transferred back to the lambda and instance \cpp{m} represents the suspended
\cpp{foo()}.\\
Function \cpp{foo()} is resumed at line 4 by executing \cpp{m.resume()} so that
control returns at line 8 and so on ...\\

Class \cpp{symmetric_coroutine<>::yield_type} from N3985\cite{N3985} is
\bfs{not} equivalent to the synthesized fiber.\\
\cpp{symmetric_coroutine<>::yield_type} does not represent the suspended context,
instead it is a special representation of the same coroutine. Thus \main or
the current thread's \entryfn\xspace can \bfs{not} be represented by \cpp{yield_type}
(see next section \nameref{representation}).\\
Because \cpp{symmetric_coroutine<>::yield_type()} yields back to the starting
point, e.g. invocation of\\
\cpp{symmetric_coroutine<>::call_type::operator()()},
both instances (\cpp{call_type} as well as \cpp{yield_type}) must be preserved.
Additionally the caller must be kept alive until the called coroutine terminates
or UB happens at resumption.\\

\zs{This API is specified in terms of passing the suspended fiber. A higher
level layer can hide that by using private variables.}

\uabschnitt{representing \emph{main()} and thread's \entryfn\xspace as fiber}\label{representation}
As shown in the previous section a synthesized fiber is created and passed
into the resumed fiber as an instance of \fiber.\\
\cppf{synthesized_main}

The mechanism presented in this proposal describes jumping between stacks (each
fiber has its own stack). The stacks of \main and the threads are not excluded
and can be used as jump targets too.\\
This is a nice feature because it allows to represent (the stacks) of \main and
the thread's as fibers. A fiber representing \main or thread's \entryfn\xspace
can be handled like an explicitly created fiber: it can passed to and returned
from functions or stored in a container.\\

In the code snippet above the suspended \main is represented by instance
\cpp{m} and could be stored in containers or managed just like \cpp{f}
by a scheduling algorithm.\\

\zs{The proposed fiber API allows representing and handling \main and the
current thread's \entryfn\xspace by an instance of \fiber in the same way as
explicitly created fibers.}

\uabschnitt{fiber returns (terminates)} When a fiber returns (terminates), what
should happen next? Which fiber should be resumed next? The only way to avoid
internal global variables that point to \main is to explicitly return a valid
fiber instance that will be resumed after the active fiber terminates.
\cppfl{terminating_fiber}

In line 5 the fiber is started by invoking \resume on instance \cpp{f}. \main
is suspended and an instance of type \cpp{fiber} is synthesized and passed as
parameter \cpp{m} to the lambda at line 2. The fiber terminates by returning
\cpp{m}. Control is transferred to \main (returning from \cpp{f.resume()} at
line 5) while fiber \cpp{f} is destroyed.\\

In a more advanced example another fiber is used as return value instead of the
passed in synthesized fiber.
\cppfl{terminating_fiber_complex}

At line 13 fiber \cpp{f2} is resumed and the lambda is entered at line 8. The
synthesized fiber \cpp{f} (representing suspended \main) is passed as a
parameter \cpp{f} and stored in \cpp{m} (captured by the lambda) at line 10.
This is necessary in order to prevent destructing \cpp{f} when the lambda
returns. Fiber \cpp{f2} uses \cpp{f1}, that was also captured by the lambda, as
return value. Fiber \cpp{f2} terminates while fiber \cpp{f1} is resumed (entered
the first time). The synthesized fiber \cpp{f} passed into the lambda at line 3
represents the terminated fiber \cpp{f2} (e.g. the calling fiber). Thus instance
\cpp{f} is invalid as the assert statement verifies at line 5. Fiber \cpp{f1} uses
the captured fiber \cpp{m} as return value (line 6). Control is returned to
\main, returning from \cpp{f2.resume()} at line 13.\\

\zs{The function passed to \fiber's constructor must have
signature `\cpp{fiber(fiber&&)}`. Using \fiber as the return value from such a
function avoids global variables.}

\uabschnitt{returning synthesized fiber instance from \cpp{resume()}}\label{fiberreturn}
An instance of \fiber remains invalid after return from \resume or \resumewith --
the synthesized fiber is returned, instead of implicitly updating the \fiber
instance on which \resume was called.\\
If the \fiber object were implicitly udpated, the fiber would 
change its identity because each fiber is associated with a stack. Each stack
contains a chain of function calls (call stack). If this association were
implicitly modified, unexpected behaviour happens.\\
The example below demonstrates the problem:
\cppfl{return_from_resume_inplace}

In this pseudo-code the \fiber object is implicitly updated.\\
The example creates a circle of fibers: each fiber prints its name and resumes
the next fiber (f1 -> f2 -> f3 -> f1 -> ...).\\
Fiber \cpp{f1} is started at line 26. The synthesized fiber \cpp{main} passed 
to the resumed fiber is not used (control flow cycles through the three
fibers).\footnote{The operating-system stack provided for \main or the current
thread's \entryfn\xspace is not destroyed when the corresponding \fiber instance is
destroyed.}
The for-loop prints the name \emph{f1} and resumes fiber \cpp{f2}. Inside 
\cpp{f2}'s for-loop the name is printed and \cpp{f3} is resumed. Fiber \cpp{f3}
resumes fiber \cpp{f1} at line 7. Inside \cpp{f1} control returns from
\cpp{f2.resume()}. \cpp{f1} loops, prints out the name and invokes \cpp{f2.resume()}. But
this time fiber \cpp{f3} instead of \cpp{f2} is resumed. This is caused by the
fact the instance \cpp{f2} gets the synthesized fiber of \cpp{f3} implicitly
assigned. Remember that at line 7 fiber \cpp{f3} gets suspended while \cpp{f1}
is resumed through \cpp{f1.resume()}.\\

This problem can be solved by returning the synthesized fiber from  \resume or
\resumewith. 
\cppf{return_from_resume_invalid}

In the example above the synthesized fiber returned by \resume is
move-assigned to the invoking fiber instance (that has resumed the current
fiber).\\

\zs{The synthesized fiber must be returned from \resume and \resumewith in order
to prevent changing the identity of the fiber.}
\xspace\newline

If the overall control flow isn't known, member function \resumewith (see section
\nameref{resumewith}) can be used to assign the synthesized \fiber to the correct \fiber
instance (the caller).
\cppf{assign_resumewith}

Member function \cpp{resume_next()} resumes the next \cpp{filament} passed as
parameter. The active fiber invokes \resumewith on the fiber aggregated by
\cpp{fila}. The lambda captures \cpp{this}, the current fiber is suspended, a
new \fiber is synthesized and passed as parameter \cpp{f} to the function
(that's the lambda) injected to the resumed fiber of \cpp{fila}.\\
Inside the lambda \cpp{f} is moved into the instance \cpp{f_} aggregated by the
\cpp{filament} that was suspended.\footnote{\bfiber\cite{bfiber} uses this 
pattern for resuming user-land threads.
The injected lambda assigns the synthesized fiber to the caller and unlocks a
spinlock if provided or re-schedules the suspended user-land thread.}

\zs{It is not necessary to know the overall control flow. It is sufficient to
pass a reference/pointer of the \emph{caller} (fiber that gets suspended) to the
resumed fiber that move-assigns the synthesized fiber to \emph{caller} (updating
the instance).}

\abschnitt{inject function into suspended fiber}\label{resumewith}
Sometimes it is useful to inject a new function (for instance, to throw an
exception or assign the synthesized fiber to the caller as described in
\nameref{fiberreturn}) into a suspended fiber. For this purpose
\cpp{resume\_with(Fn&& fn)} (or \xtresumewith) may be called, passing the
function \cpp{fn()} to execute.

\cppfl{suspender}
The \resumewith call at line 11 injects function \cpp{fn()} into
fiber \cpp{f} as if the \resume call at line 3 had directly
called \cpp{fn()}.

Like an \entryfn passed to \fiber, \cpp{fn()} must accept
\cpp{std::fiber\_context&&} and return\\
\fiber. The \fiber instance returned by \cpp{fn()} will, in turn, be returned
to \cpp{f}'s lambda by the \resume at line 3.

Suppose that code running on the program's main fiber calls \resume (line 12 below), thereby
entering the first lambda shown below. This is the point at which \cpp{m} is
synthesized and passed into the lambda at line 2.

Suppose further that after doing some work (line 4), the lambda calls
\cpp{m.resume()}, thereby switching back to the main fiber. The lambda remains
suspended in the call to \cpp{m.resume()} at line 5.

At line 18 the main fiber calls \cpp{f.resume\_with()} where the passed lambda
accepts \cpp{fiber\_context &&}. That new lambda is called on the fiber of the suspended
lambda. It is as if the \cpp{m.resume()} call at line 8 directly called the second
lambda.

The function passed to \resumewith has almost the same range of possibilities as
any function called on the fiber represented by \cpp{f}. Its special invocation
matters when control leaves it in either of two ways:

\begin{enumerate}
  \item If it throws an exception, that exception unwinds all previous stack
        entries in that fiber (such as the first lambda's) as well, back to
        a matching \cpp{catch} clause.\footnote{As stated
        in \nameref{exceptions}, if there is no matching \cpp{catch}
        clause in that fiber, \cpp{std::terminate()} is called.}
  \item If the function returns, the returned \fiber instance is returned by
        the suspended \cpp{m.resume()} (or \resumewith, or \xtresume, or
        \xtresumewith) call.
\end{enumerate}

\cppfl{ontop}

The \cpp{f.resume\_with(<lambda>)} call at line 18 passes control to the second
lambda on the fiber of the first lambda.

As usual, \resumewith synthesizes a \fiber instance representing the calling
fiber, passed into the lambda as \cpp{m}. This particular lambda returns \cpp{m}
unchanged at line 21; thus that \cpp{m} instance is returned by the \resume call
at line 8.

Finally, the first lambda returns at line 10 the \cpp{m} variable updated at
line 8, switching back to the main fiber.

One case worth pointing out is when you call \resumewith (or \xtresumewith) on
a\\
\fiber that has not yet been resumed for the first time:
\cppfl{initial_resume_with}

In this situation, \cpp{injected()} is called with a \fiber instance
representing the caller of \resumewith. When \cpp{injected()} eventually
returns that (or some other) \fiber instance, the returned\\
\fiber instance is passed into \cpp{topfunc()} as its \cpp{prev} parameter.

\zs{Member functions \resumewith and \xtresumewith allow you to inject a
function into a suspended fiber.}

\abschnitt{passing data between fibers}

Data can be transferred between two fibers via global pointer, calling
wrapper (like \cpp{std::bind}) or lambda capture.
\cppfl{passing_lambda}

The \resume call at line 8 enters the lambda and passes 1 into the
new fiber. The value is incremented by one, as shown at line 4. The expression
\cpp{caller.resume()} at line 5 resumes the original context (represented
within the lambda by \cpp{caller}).\\
The call to \cpp{lambda.resume()} at line 10 resumes the lambda, returning from
the \cpp{caller.resume()} call at line 5. The \fiber instance \cpp{caller}
invalidated by the \resume call at line 5 is replaced with the new instance
returned by that same \resume call.\\
Finally the lambda returns (the updated) \cpp{caller} at line 6, terminating its
context.\\

Since the updated \cpp{caller} represents the fiber suspended by the call at
line 10, control returns to \main.\\

However, since context \cpp{lambda} has now terminated, the updated \cpp{lambda}
is invalid. Its \opbool returns \cpp{false}; its \cpp{operator\!()} returns
\cpp{true}.\\

\zs{Using lambda capture is the preferred way to transfer data between two
fibers; global pointer and calling wrapper (like \cpp{std::bind}) are
alternatives.}

\abschnitt{termination}

There are a few different ways to terminate a given fiber without
terminating the whole process, or engaging undefined behavior.\\

When a \fiber instance is constructed with an \entryfn, its new stack is
initialized with the frame of an implicit top-level function that can
catch \unwindex. \unwindex binds a \fiber instance; the implicit \cpp{catch}
clause returns the bound \fiber from that top-level function.\\

Therefore, any of the following will gracefully terminate a fiber:

\begin{itemize}
    \item Cause its \entryfn to return a valid \fiber.
    \item From within the fiber you wish to terminate, call \unwindfib with a
          valid \fiber. This throws a \unwindex instance that binds the passed
          \fiber; that fiber will be resumed when the active fiber terminates.
    \item From within the fiber you wish to terminate, construct and
          throw \unwindex, binding the \fiber you intend to resume next. This
          is what \unwindfib does internally.
    \item Call \cpp{fiber\_context::resume\_with(unwind\_fiber)}. This is what \dtor
          does. Since\\\unwindfib accepts a \fiber, and since \resumewith
          synthesizes a\\\fiber representing its caller and passes it to the
          subject function, this terminates the fiber referenced by the
          original \fiber instance and switches back to the caller.
    \item Engage \dtor: switch to some other fiber, which will
          receive a \fiber instance representing the current fiber. Make that
          other fiber destroy the received \fiber instance.
\end{itemize}

(However, since the operating system allocates the stack for \main and for a
thread's \entryfn, of course there is no implicit top-level stack frame, no
implicit \cpp{catch (std::unwind\_exception)}. In a conforming implementation,
returning from a thread's \entryfn\xspace may terminate all fibers on that
thread. Returning from \main may terminate the whole process.)\\

In an environment that forbids exceptions, every \fiber you launch must
terminate gracefully, by returning from its top-level function. You may not
call \unwindfib. You may not call \dtor, explicitly or implicitly, on a
valid \fiber instance.\\

When an explicitly-launched fiber's \entryfn\xspace returns a valid \fiber
instance, that fiber is terminated. Control switches to the fiber indicated by
the returned \fiber instance.\\

Returning an invalid \fiber instance (\opbool returns \cpp{false}) invokes
undefined behavior.\\

If the \entryfn\xspace returns the same \fiber instance it was originally
passed (or rather, the\\\fiber instance most recently returned by \resume,
\resumewith, \xtresume or \xtresumewith), control returns to the fiber that
most recently resumed the running fiber. However, the \entryfn\xspace may
return (switch to) any reachable valid \fiber instance.\\

\emph{Calling} \resume means: ``Please switch to the indicated fiber; I
am suspending; please resume me later.''\\

\emph{Returning} a particular \fiber means: ``Please switch to the indicated
fiber; and by the way, I am done.''

\abschnitt{exceptions}\label{exceptions}

In general, if an uncaught exception escapes from the \entryfn,
\cpp{std::terminate} is called.
%There is one exception: \unwindex. The fiber
%facility internally uses \unwindex to clean up the stack of a suspended context
%being destroyed. This exception must be allowed to propagate out of an \entryfn.\\

%A correct \entryfn\ \cpp{try/catch} block looks like this:
%\cppf{rethrow_unwind}

Of course, no \cpp{try/catch} block is needed if neither \entryfn\xspace nor
anything it calls throws exceptions.

%%\abschnitt{stack destruction}\label{destruction}

On construction of a fiber a stack is allocated. If the \entryfn\xspace returns,
the stack will be destroyed. If the function has not yet returned and the
\nameref{destructor} of the \fiber instance representing that context is called,
the stack will be unwound and destroyed.\\

Consider a running fiber \cpp{f2} that destroys the \fiber instance
representing \cpp{f1}.\\

\cpp{f1}'s destructor, running on \cpp{f2}, implicitly calls member-function
\resumewith, passing \unwindfib as
argument. Fiber \cpp{f1} will be temporarily resumed and \unwindfib is
invoked. Function \unwindfib binds an instance of \fiber that
represents \cpp{f2}, then throws exception \unwindex, which
unwinds \cpp{f1}'s stack
(walking the stack and destroying automatic variables in reverse order of
construction).
The first frame on \cpp{f1}'s stack, the one created by \fiber's constructor,
catches the exception,
extracts the bound \fiber representing \cpp{f2} and terminates \cpp{f1} by returning
\cpp{f2}. Control is returned to \cpp{f2} and \cpp{f1}'s
stack gets deallocated.\\

The StackAllocator's deallocate operation runs on the fiber that invoked
\dtor, in this case \cpp{f2}.\\

The stack on which \cpp{main()} is executed, as well as the stack implicitly
created by \cpp{std::thread}'s constructor, is allocated by the operating
system. Such stacks are recognized by \fiber, and are not deallocated by its
destructor.


\abschnitt{stack allocators}\label{stackalloc}

Stack allocators are used to create stacks and might implement arbitrary stack
strategies. For instance, a stack allocator might append a guard page at the end
of the stack, or cache stacks for reuse, or create stacks that grow on demand.\\

Because stack allocators are provided by the implementation, and are only used
as parameters of the constructor, the StackAllocator concept is an
implementation detail, used only by the internal mechanisms of the
implementation. Different implementations might use different StackAllocator
concepts.\\

An implementation must provide at least \cpp{fixedsize}. It may provide
additional stack allocators. When an implementation provides a stack allocator
matching one of the descriptions below, it should use the specified name.\\

Possible types of stack allocators:
\begin{itemize}
    \item \cpp{protected\_fixedsize}: The constructor accepts a \cpp{size\_t}
          parameter. This stack allocator constructs a contiguous stack of
          specified size, appending a guard page at the end to protect against
          overflow. If the guard page is accessed (read or write operation), a
          segmentation fault/access violation is generated by the operating
          system.
    \item \cpp{fixedsize}: The constructor accepts a \cpp{size\_t} parameter.
          This stack allocator constructs a contiguous stack of specified size.
          In contrast to \cpp{protected\_fixedsize}, it does not append a guard
          page. The memory is simply managed by \cpp{std::malloc()}
          and \cpp{std::free()}, avoiding kernel involvement.
    \item \cpp{segmented}: The constructor accepts a \cpp{size\_t} parameter.
          This stack allocator creates a segmented stack\cite{gccsplit} with the
          specified initial size, which \bfs{grows on demand}.\footnote{An
          implementation of the \cpp{segmented} StackAllocator necessarily
          interacts with the C++ runtime. For instance, with gcc, the
          Boost.Context\cite{bcontext} library invokes
          the \cpp{\_\_splitstack\_makecontext()}
          and \cpp{\_\_splitstack\_releasecontext()} intrinsic
          functions.\cite{splitalloc}}
\end{itemize}

It is expected that the StackAllocator's allocation operation will run in the
context of the \fiber constructor, and that the StackAllocator's deallocation
operation will run on the fiber calling \dtor (after control returns from the
destroyed fiber). No special constraints need apply to either operation.

\abschnitt{\fiber as building block for higher-level frameworks}\label{low_level}

A low-level API enables a rich set of higher-level frameworks that provide
specific syntaxes/semantics suitable for specific domains. As an example, the
following frameworks are based on the low-level fiber switching API of
\bcontext\cite{bcontext} (which implements the API proposed here).

\uabschnitt{\bcoroutine}\cite{bcoroutine2} implements \bfs{asymmetric coroutines}
\cpp{coroutine<>::push_type} and\\
\cpp{coroutine<>::pull_type}, providing a
unidirectional transfer of data. These stackful coroutines are only used in
pairs. When \cpp{coroutine<>::push_type} is explicitly
instantiated, \cpp{coroutine<>::pull_type} is synthesized and passed as
parameter into the coroutine function. In the
example below, \cpp{coroutine<>::push_type} (variable \cpp{writer}) provides the
resume operation, while \cpp{coroutine<>::pull_type} (variable \cpp{in})
represents the suspend operation. Inside the lambda,\cpp{in.get()}
pulls strings provided by \cpp{coroutine<>::push_type}'s output iterator support.
\cppf{bcoroutine_ex}

\uabschnitt{\synca}\cite{synca} (by Grigory Demchenko) is a small, efficient
library to perform asynchronous operations using source code that resembles synchronous
operations. The main
features are a \bfs{GO-like} syntax, support for transferring execution context
explicitly between different thread pools or schedulers (portals/teleports) and
asynchronous network support.
\cppf{synca_ex}

The code itself looks like synchronous invocations while internally it uses
asynchronous scheduling.

\uabschnitt{\bfiber}\cite{bfiber} implements \bfs{user-land threads} and combines
fibers with schedulers (the scheduler algorithm is a customization point). The API
is modelled after the \thread API and contains objects such as
\cpp{future}, \cpp{mutex},\\
\cpp{condition_variable} ...
\cppf{bfiber_ex}

\uabschnitt{Facebook's \fbfibers}\cite{fbfiber} is an asynchronous C++ framework
using \bfs{user-land threads} for parallelism. In contrast to \bfiber,
\fbfibers\xspace exposes the scheduler and permits integration with various
event dispatching libraries.
\cppf{fbfiber_ex}

\fbfibers\xspace is used in many critical applications at Facebook for instance
in \fbmcrouter\cite{fbmcrouter} and some other Facebook services/libraries like
ServiceRouter (routing framework for \fbthrift\cite{fbthrift}), Node API (graph
ORM API for graph databases) ...

\uabschnitt{Bloomberg's \bbquantum}\cite{bbquantum} is a full-featured and
powerful C++ framework that allows users to dispatch units of work (a.k.a.
tasks) as coroutines and execute them concurrently using the 'reactor' pattern.
Its main features are support for streaming futures which allows faster processing
of large data sets, task prioritization, fast pre-allocated memory pools and
parallel \cpp{forEach} and \cpp{mapReduce} functions.
\cppf{bbquantum}

\bbquantum\xspace is used in large projects at Bloomberg.

\uabschnitt{Habanero Extreme Scale Software Research Project\cite{habanero}}
provides a task-based parallel programming model via its \hclib\cite{hclib}.
The runtime provides work-stealing, async-finish,\footnote{async-finish is a
variant of the fork-join model. While a task might fork a group of
child tasks, the child tasks might fork even more tasks. All tasks can
potentially run in parallel with each other. The model allows a parent task to
selectively join a subset of child tasks.}
parallel-for and future-promise parallel programming patterns. The library is not an exascale
programming system itself, but it manages intra-node resources and schedules
components within an exascale programming system.

\uabschnitt{Intel's \tbb}\cite{tbb} internally uses fibers for long running
jobs\footnote{because of the requirement to support a broad range of
architectures \href{https://github.com/intel/tbb/blob/tbb_2020/src/tbb/co_context.h\#L190}
{\swapcontext} was used} as reported by Intel.

\uabschnitt{\userver}\cite{userver} is a modern open source asynchronous
framework with a rich set of abstractions, database connectors/drivers,
protocols and synchronization primitives for fast and comfortable creation
of IO-bound C++ microservices, services and utilities.

\uabschnitt{Alibaba's \photon}\cite{photon} supports a large number of services
and clients, especially the image service of Alibaba’s container platform,
which supports various Internet services for billions of users.
Also used in some ByteDance services.

\uabschnitt{Alibaba's \libeasy}\cite{libeasy} supports a large number of
servers, including storage, database, etc. Not officially open-sourced, but
carried out by some open source projects, such as Oceanbase, tair, etc.

\uabschnitt{Baidu's \bthread}\cite{bthread} has 1 million+ deployed instances
(not counting clients) and thousands of kinds of services.

\uabschnitt{Tencent's \libco}\cite{libco} is a c/c++ coroutine library that
is widely used in backend service of WeChat, which is the largest IM service
in China, with billions of users. 

\uabschnitt{\libgo}\cite{libgo} is said to be developed by Meizu.

\zs{As shown in this section a low-level API can act as building block for a
rich set of high-level frameworks designed for specific application domains
that require different aspects of design, semantics and syntax.}

\abschnitt{interaction with STL algorithms}

In the following example STL algorithm \cpp{std::generate} and fiber \cpp{g}
generate a sequence of Fibonacci numbers and store them into \cpp{std::vector}
\cpp{v}.
\cppf{generator}

\seeappalaunch

\zs{The proposed fiber API does not require modifications of the STL and can be
used together with existing STL algorithms.}

\abschnitt{possible implementation strategies}\label{implementations}

\zs{This proposal does \so{NOT} seek to standardize any particular implementation or
impose any specific calling convention!}

Modern \bfs{micro-processors} are \bfs{register machines}; the content of
processor registers represents the execution context of the program at a given
point in time.

\bfs{Operating systems} maintain for each process all relevant data (execution
context, other hardware registers etc.) in the process table. The operating system's
\bfs{CPU scheduler} periodically suspends and resumes processes in order to
share CPU time between multiple processes. When a process is suspended, its
execution context (processor registers, instruction pointer, stack pointer, ...)
is stored in the associated process table entry. On resumption, the CPU
scheduler loads the execution context into the CPU and the process continues
execution.

The CPU scheduler does a \bfs{full context switch}. Besides preserving
the execution context (complete CPU state), the cache must be invalidated and
the memory map modified.

A kernel-level context switch is several orders of magnitude slower than a
context switch at user-level\cite{Tanenbaum2009}.

\uabschnitt{hypothetical fiber preserving complete CPU state} This strategy tries to
preserve the complete CPU state, e.g. all CPU registers. This requires that the
implementation identifies the concrete micro-processor type and supported processor
features. For instance the x86-architecture has several flavours of extensions
such as MMX, SSE1-4, AVX1-2, AVX-512.

Depending on the detected processor features, implementations of certain
functionality must be switched on or off. The CPU scheduler in the operating system
uses such information for context switching between processes.

A fiber implementation using this strategy requires such a detection mechanism
too (equivalent to swapper/\cpp{system_32()} in the Linux kernel).

Aside from the complexity of such detection mechanisms, preserving the complete
CPU state for each fiber switch is expensive.

\zs{A context switch facility that preserves the complete CPU state like an
operating system is possible but impractical for user-land.}

\uabschnitt{fiber switch using the calling convention}\label{callingconvention}
For \fiber, not all registers need be preserved because the context
switch is effected by a visible function call. It need not be completely transparent like
an operating-system context switch; it only needs to be as transparent as a call
to any other function. The calling convention -- the part of the ABI that
specifies how a function's arguments and return values are passed -- determines
which subset of micro-processor registers must be preserved by the called
subroutine.

The \bfs{calling convention}\cite{SYSVABI} of \bfs{SYSV ABI} for \bfs{x86\_64}
architecture determines that general purpose registers R12, R13, R14, R15, RBX
and RBP must be preserved by the sub-routine - the first arguments are passed
to functions via RDI, RSI, RDX, RCX, R8 and R9 and return values are stored in
RAX, RDX.

So on that platform, the \resume implementation preserves the \bfs{general
purpose registers} (R12-R15, RBX and RBP) specified by the calling convention.
In addition, the \bfs{stack pointer} and \bfs{instruction pointer} are
preserved and exchanged too -- thus, from the point of view of calling
code, \resume behaves like an ordinary function call.

In other words, \resume acts on the level of a simple function invocation --
with the same performance characteristics (in terms of CPU cycles).

This technique is used in \bcontext\cite{bcontext} which acts as building block
for (e.g.) \fbfibers\xspace and \bbquantum; see section \nameref{low_level}.

\uabschnitt{in-place substitution at compile time} During code generation,
a compiler-based implementation could inject the assembler code responsible
for the fiber switch directly into each function that calls \resume. That would save
an extra indirection (JMP + PUSH/MOV of certain registers used to
invoke \resume).

\uabschnitt{CPU state on the stack} Because each fiber must preserve CPU
registers at suspension and load those registers at resumption, some storage
is required.

Instead of allocating extra memory for each fiber, an implementation can use
the stack by simply advancing the stack pointer at suspension and pushing the
CPU registers (CPU state) onto the stack owned by the suspending fiber. When
the fiber is resumed, the values are popped from the stack and loaded into the
appropriate registers.

This strategy works because only a running fiber creates new stack frames
(moving the stack pointer). While a fiber is suspended, it is safe to keep the
CPU state on its stack.

Using the stack as storage for the CPU state has the additional advantage
that \fiber need not itself contain the stored CPU state: it need only contain
a pointer to the stack location.

Section \nameref{synthesizing} describes how global variables are avoided
by synthesizing a \fiber from the active fiber (execution context) and passing
this synthesized \fiber (representing the now-suspended fiber) into the resumed
fiber. Using the stack as storage makes this mechanism very easy to
implement.\footnote{The implementation of \bcontext\cite{bcontext} utilizes this
technique.}
Inside \resume the code pushes the relevant CPU registers onto the stack, and
from the resulting stack address creates a new \fiber. This instance is then
passed (or returned) into the resumed fiber (see \nameref{synthesizing}).

\zs{Using the active fiber's stack as storage for the CPU state is efficient because no
additional allocations or deallocations are required.}

\uabschnitt{\exfns}

Both \exfns should report exceptions solely on the current thread of execution.
Reporting exceptions on any other thread of execution would make them
unreliable in practice.

A straightforward implementation could make \allresume save and restore the
data underlying \exfns as part of saving and restoring the rest of the fiber
state. Since \exfns data is necessarily thread-local, the likely cost would be
a TLS access on every \anyresume call.

Alternatively, \fiber's constructor could update an internal associative
container whose key is the high end of the new fiber stack. \exfns could
call \cpp{upper\_bound()}, passing the current stack pointer, to discover
which stack is current. This would shift the cost from every context switch
to \exfns calls.

In 2023, though, the zeitgeist appears to be that \exfns are not sufficiently
important to warrant altering exception-handling implementations to fix their
behavior. This is why, in the \nameref{api} section, it is
implementation-defined whether \exfns are specific to the current thread of
execution or the current \thread.

\abschnitt{fiber switch at architectures with register window}

The implementation of fiber switch is possible -- many libc implementations
still provide the ucontext-API (\cpp{swapcontext()} and related
functions)\footnote{ucontext was removed from POSIX standard by POSIX.1-2008}
for architectures using register window (like SPARC). The implementation of
\cpp{swapcontext()} could be used as blue-print for the fiber-API.

\abschnitt{how fast is a fiber switch}

A fiber switch takes 19 CPU cycles on a \emph{x86\_64-Linux} system
\footnote{Intel XEON E5 2620v4 2.2GHz} using the implementation based on
the strategy described in \nameref{callingconvention} (implemented in
\bcontext\cite{bcontext}, branch \emph{fiber}).

\abschnitt{interaction with accelerators}

\abschnitt{multi-threading environment}\label{xthread}

Any thread in a program may be shared by multiple fibers.

A newly-instantiated fiber is not yet associated with any thread. However,
once a fiber has been resumed the first time by some thread, it must
thereafter be resumed only by that same thread.

There could potentially be Undefined Behavior if:
\begin{itemize}
    \item a function running on a fiber references \cpp{thread\_local} variables
    \item the compiler/runtime implementation caches a pointer
          to \cpp{thread\_local} storage in that function's stack frame
    \item that fiber is suspended, and
    \item the suspended fiber is resumed on a different thread.
\end{itemize}

The cached TLS pointer is now pointing to storage belonging to some other
thread. If the original thread terminates before the new thread, the cached
TLS pointer is now dangling.

For this reason, it is forbidden to resume a fiber on any thread other than
the one on which it was first resumed.

\abschnitt{acknowledgment}

The authors would like to thank Andrii Grynenko, Detlef Vollmann, Geoffrey Romer,
Grigory Demchenko, Lee Howes, David Hollman, Eric Fiselier and Yedidya Feldblum.

\newpage

\setcounter{section}{33}
\setcounter{subsection}{6}

\abschnitt{API}\label{api}

\rSec2[fiber-context]{Cooperative User-Mode Threads}

\rSec3[fiber-context.general]{General}

The extensions proposed here support creation and activation of cooperative
user-mode threads, here called \emph{fibers}.

The term ``user-mode'' means that control can be passed from one fiber to
another without entering the operating-system kernel.

The term ``cooperative'' means that typically multiple fibers share an
underlying execution agent, for example a \cpp{std::thread}. On the
underlying execution agent, only one fiber is running at any given time. Sharing
that agent is explicit rather than pre-emptive. The running
fiber \emph{suspends} (or \emph{yields}) to another fiber. This
action \emph{launches} a new fiber, or \emph{resumes} a previously-suspended
fiber.

Suspending the running fiber in order to resume (or launch) another is
called \emph{context switching}.

The term ``thread'' in ``cooperative user-mode thread'' means that even
though a given fiber may suspend and later be resumed, it is logically a
thread of execution as defined in [intro.multithread].

Launching a fiber logically creates a new function stack, which remains
associated with that fiber throughout its lifetime. Calling functions on a
particular fiber, and returning from them, is independent of function calls
and returns on any other fiber.

Context switching can be effected by designating some other fiber's stack as
current, in a manner appropriate to the existing implementation of function stacks.

\rSec3[fiber-context.empty]{Empty vs. Non-Empty}

A \fiber instance may be \emph{empty} or \emph{non-empty}. A
default-constructed \fiber is empty. A moved-from \fiber is empty. A \fiber
representing a suspended fiber is non-empty.

\rSec3[fiber-context.implicit]{Explicit Fiber vs. Implicit Fiber}

The default thread on which the program runs \main has an
initial \emph{default fiber} whose stack is the stack on which \main is
entered. \tsnote{Thus, when \main instantiates a new \fiber, it becomes the
second fiber in the program.} Similarly, every explicitly-launched
\cpp{std::thread} or \cpp{std::jthread} has an initial default fiber whose
stack is the stack on which the function passed to \cpp{std::thread} or
\cpp{std::jthread}'s constructor is entered.

We use the phrase \emph{explicit fiber} or \emph{explicitly-launched fiber} to
designate a fiber instantiated by user code; conversely, \emph{implicit fiber}
designates the default fiber on any thread. An implicit fiber's \emph{owning
thread} is the thread of which that fiber is the default fiber. An explicit
fiber has no owning thread. Instead, when necessary, we speak of the thread on
which a fiber was launched.

A fiber is explicitly instantiated by passing an \emph{\entryfn} to \fiber's
constructor. This function is not entered until the first call to one of
the \cpp{fiber\_context::resume()} family of methods.

When a fiber is first entered, a synthesized non-empty \fiber instance
representing the newly-suspended previous fiber is passed as a parameter to
its \entryfn. Once entered, a fiber may suspend by calling one of the \resume
family of methods on any available non-empty \fiber instance. When the
suspended fiber is resumed, that method returns a synthesized \fiber instance
representing the newly-suspended previous fiber.

The synthesized \fiber instance received in either of those ways might
represent either an explicit fiber or an implicit fiber.

An explicit fiber terminates by returning from its \entryfn. If the \entryfn
returns a non-empty \fiber instance, the fiber represented by that \fiber
instance is resumed.

%% \rSec3[fiber-context.toplevel]{Implicit Top-Level Function}

%% On every explicit fiber, the behaviour is equivalent to calling the \entryfn
%% passed to \fiber's constructor from an implicit top-level function.
%% If the fiber is later
%% unwound, this conceptual top-level stack frame serves as delimiter: this point
%% is where unwinding stops.

If the fiber's \entryfn exits via an exception, \cpp{std::terminate} is called.

%% Returning a \fiber instance from the explicit fiber's \entryfn is equivalent
%% to returning control to the implicit top-level function.
%% Similarly,
%% when \unwindfib unwinds a fiber stack, it conceptually returns the \fiber
%% instance it was passed to the implicit top-level function. Either way, the
%% The
%% conceptual implicit top-level function is responsible for deallocating the
%% explicit fiber's stack memory on return from the \entryfn.
%% 
%% Similarly, on every implicit fiber, the behaviour is equivalent to passing control through an
%% implicit top-level function above \main and above the \entryfn for
%% each \thread.
%% The conceptual stack frame for this implicit top-level function delimits
%% stack unwinding for each of these stacks. If the fiber stack is unwound,
%% control is conceptually returned to this implicit top-level function.
%% The conceptual top-level
%% function for an implicit fiber does not deallocate the fiber's stack memory,
%% since the host environment will do that.

%% \begin{itemize}
%%     \item
%%     \item If an empty \fiber instance is returned to the conceptual top-level
%%     function for an explicit fiber, the calling thread is terminated.
%%     \item If an empty \fiber instance is returned to the conceptual top-level
%%     function for the default fiber of an explicit thread, that thread is
%%     terminated.
%%     \item If an empty \fiber instance is returned to the conceptual top-level
%%     function above \main, the process is terminated.
%% \end{itemize}

\rSec3[fiber-context.synopsis]{Header <experimental/fiber\_context> synopsis}

\cppf{synopsis}

\rSec3[fiber-context.class]{Class fiber\_context}

\cppf{fiber}

\mbrhdr{fiber\_context() noexcept}\label{constructor}

\effects
\begin{description}
    \item[---] instantiates an empty \fiber.
\end{description}

\postcond
\begin{description}
    \item[---] \cpp{empty()} returns \cpp{true}.
\end{description}

\mbrhdr{template<typename Fn> explicit fiber\_context(Fn\&\& entry)}

\constraints
\begin{description}
    \item[---] This constructor template shall not participate in overload
              resolution unless \cpp{Fn}
              is \emph{Lvalue-Invocable} [func.wrap.func] for the argument
              type \cpp{std::fiber\_context&&} and the return type \fiber.
\end{description}

\effects
\begin{description}
    \item[---] instantiates a \fiber representing a fiber suspended before
              entry to \cpp{entry}.
              \tsnote{\cpp{entry} is entered only when \allresume is called.}
    \item[---] The stack and any other necessary resources are created.
\end{description}

\except
\begin{description}
    \item[---] Any exception resulting from failure to acquire necessary
               system resources.
\end{description}

\mbrhdr{fiber\_context(fiber\_context\&\& other) noexcept}

\effects
\begin{description}
    \item[---] moves underlying state from \cpp{other} to new \fiber
\end{description}

\postcond
\begin{description}
    \item[---] \cpp{empty()} returns the value previously returned by \cpp{other.empty()}
    \item[---] \cpp{other.empty()} returns \cpp{true}
\end{description}

\mbrhdr{\cpp{\~fiber\_context()}}

\effects
\begin{description}
    \item[---] destroys a \fiber instance.
\end{description}

\requires
\begin{description}
    \item[---] \cpp{empty()} returns \cpp{true}.
\end{description}

\tsnote{If a \fiber instance to be destroyed is not yet empty, an application
must call \cpp{get\_stop\_source().request\_stop()}, or otherwise convey to
the suspended fiber the need to terminate voluntarily.}

\mbrhdr{fiber\_context\& operator=(fiber\_context\&\& other) noexcept}

\requires
\begin{description}
    \item[---] \cpp{empty()} returns \cpp{true}.
\end{description}

\effects
\begin{description}
    \item[---] assigns the state of \cpp{other} to \cpp{*this}
\end{description}

\returns
\begin{description}
    \item[---] \cpp{*this}
\end{description}

\postcond
\begin{description}
    \item[---] \cpp{empty()} returns the value previously returned by \cpp{other.empty()}
    \item[---] \cpp{other.empty()} returns \cpp{true}
\end{description}

\mbrhdr{template<typename Fn> fiber\_context resume\_with(Fn\&\& fn) \&\&}

\constraints
\begin{description}
    \item[---] This member function template shall not participate in overload
               resolution unless \cpp{Fn} is \emph{Lvalue-Invocable} [func.wrap.func]
               for the argument type \cpp{std::fiber\_context&&} and the return
               type \fiber.\\
               \bfs{Needs update to \cpp{Invocable} concept.}
\end{description}

\requires
\begin{description}
    \item[---] \cpp{empty()} returns \cpp{false}
    \item[---] \currthread is the same as \lastthread
\end{description}

\effects
\begin{description}
    \item[---] Saves the execution context of the calling fiber.
    \item[---] Suspends the calling fiber.
    \item[---] Let \cpp{caller} be a synthesized \fiber instance representing
               the suspended caller.
    \item[---] Resumes the fiber represented by \cpp{*this}.
    \item[---] Restores the execution context of the resumed fiber.
    \item[---] Evaluates \cpp{fn(caller)} on the newly-resumed fiber.
               Let \cpp{returned} be the \fiber instance returned by \cpp{fn}.
               \tsnote{\cpp{returned} may or may not be the same as \cpp{caller}.}
               \tsnote{\cpp{returned} may be empty.}
    \item[---] If the fiber represented by \cpp{*this} has not previously been
               entered, passes \cpp{returned} to its \entryfn.
    \item[---] Otherwise, the fiber represented by \cpp{*this} previously
               suspended itself by calling one of \allresume.
               Returns \cpp{returned} from whichever of the resume functions
               was called.
\end{description}

\remarks
\begin{description}
    \item[---] A newly constructed but not yet resumed fiber may be resumed by
              any thread.
\end{description}

\returns
\begin{description}
    \item[fiber\_context] on resumption, \resumewith returns a \fiber
               representing the immediately preceding fiber: the fiber that
               resumed this one, thereby suspending itself
\end{description}

\except
\begin{description}
%   \item[---] \allresume throws
%             \unwindex when, while suspended, the \fiber instance representing
%             the suspended fiber is destroyed
    \item[---] Any exception thrown by evaluating the \cpp{fn} parameter
               passed to some other fiber's future call to
               \resumewith on a \fiber instance representing this suspended
               call to \resumewith.
\end{description}

\postcond
\begin{description}
    \item[---] \cpp{empty()} returns \cpp{true}
\end{description}

\tsnote{The returned \cpp{fiber\_context} indicates via \cpp{empty()} whether the previous active
fiber has terminated (returned from \entryfn).}

\tsnote{\allresume empties the instance on which it is called. In order to
express the state change explicitly, these methods are rvalue-reference
qualified. For this reason, no non-empty \fiber instance ever represents the
currently-running fiber.}

\mbrhdr{fiber\_context resume() \&\&}

\effects
Equivalent to:\\
\cpp{resume\_with([](fiber\_context&& caller)\{ return std::move(caller); \})}

\mbrhdr{[[nodiscard]] stop\_source get\_stop\_source() noexcept;}

\effects Equivalent to: \cpp{return ssource;}

\mbrhdr{[[nodiscard]] stop\_token get\_stop\_token() const noexcept;}

\effects Equivalent to: \cpp{return ssource.get\_token();}

\mbrhdr{bool request\_stop() noexcept;}

\effects Equivalent to: \cpp{return ssource.request\_stop();}

\mbrhdr{bool can\_resume() noexcept}

\returns
\begin{description}
    \item[---] \cpp{false} if \cpp{*this} is empty, or if \currthread is not the same
        as \lastthread.
\end{description}

\tsnote{When \canresume returns \cpp{true}, the \fiber instance may be resumed
by \allresume.}

\remarks
\begin{description}
    \item[---] \canresume must not be called concurrently from multiple threads.
\end{description}

\tsnote{\canresume is not marked \cpp{const} because in at least one
implementation, it requires an internal context switch. However, the
stack operations are effectively read-only.}

\mbrhdr{bool empty() const noexcept}

\returns
\begin{description}
    \item[---] \cpp{false} if \cpp{*this} represents a fiber of
               execution, \cpp{true} otherwise.
\end{description}

\tsnote{Regardless of the number of \fiber declarations, exactly one
\fiber instance represents each suspended fiber.}

\mbrhdr{explicit operator bool() const noexcept}

\begin{description}
    \item[---] Equivalent to \cpp{(\! empty())}
\end{description}

\mbrhdr{void swap(fiber\_context\& other) noexcept}

\effects
\begin{description}
    \item[---] Exchanges the state of \cpp{*this} with \cpp{other}.
\end{description}

%% \rSec3[fiber-context.unwinding]{Function unwind\_fiber()}
%% 
%% \mbrhdr{[[ noreturn ]] void unwind\_fiber(fiber\_context\&\& other)}
%% 
%% \effects
%% terminate the current running fiber.
%% 
%% \remarks
%% \begin{description}
%%     \item[---] The underlying Unwinding facility (for instance the unwind facility
%%                described in \emph{System V ABI for AMD64}) unwinds the stack
%%                to the implicit top-level stack frame and terminates the
%%                current fiber as described above.
%%     \item[---] Unwinding the fiber's stack causes its stack variables to be
%%                destroyed.
%%     \item[---] During this specific stack unwinding, 
%% %% only \catchall clauses are executed. No other
%%                no \cpp{catch} clauses are executed, not even \catchall.
%%     \item[---] Once the running fiber has been fully unwound, \cpp{other} is
%%                returned to the fiber's conceptual top-level function as
%%                described in \nameref{fiber-context.toplevel}.
%% %%  \item[---] Unwinding the fiber's stack causes relevant \catchall
%% %%             clauses to be executed.
%% %%  \item[---] During this specific stack unwinding, a \catchall
%% %%             clause that does not execute a \cpp{throw;} statement behaves
%% %%             as if it ended with a \cpp{throw;} statement.
%% %%  \item[---] During this specific stack unwinding, a \catchall
%% %%             clause that attempts to throw any C++ exception engages
%% %%             Undefined Behaviour.
%% \end{description}
%% 
%% \returns
%% \begin{description}
%%     \item[---] None: \unwindfib does not return
%% \end{description}
%% 
%% \except
%% \begin{description}
%%     \item[---] None catchable by C++
%% \end{description}

\newpage
\abschnitt{Appendix A: potential premature destruction of exception object}\label{exlife}

In \stdclause{except.throw} paragraph 4, the destruction of an exception
object is specified to potentially occur when an active handler for the
exception exits, not when a handler exits while the exception is still the
currently handled exception. With a Boost
implementation which predates the proposed changes to \stdclause{except}
(in an Itanium C++ ABI environment), it is possible to observe cases where an exception is
destroyed at a different point than specified (and, in particular, when a
handler for the exception is still active in a fiber). Consider
\href{https://github.com/secondlife/3p-boost/blob/nat/exstate/tests/early_exc_destroy.cpp}{the following program}.

\cppf{early_exc_destroy}

\newpage
\abschnitt{Appendix B: throw-expression with no operand}\label{throw}

Both \stdclause{expr.throw} paragraph 3 and \cpp{current\_exception()}
(\stdclause{propagation} paragraph 8) reference the ``currently handled
exception'' (\stdclause{except.handle} paragraph 8). Thus, the
construct \cpp{throw;} is by definition equivalent to\\
\cpp{std::rethrow\_exception(std::current\_exception());}
(\stdclause{propagation} paragraph 9).

The existing definition of currently handled exception:

``The exception with the most recently activated handler that is still active
is called the \emph{currently handled exception.}''

does not clearly constrain the scope to the current thread. This constraint
must be inferred from \stdclause{except.throw} paragraph 2:

``When an exception is thrown, control is transferred to the nearest handler
with a matching type \xref{except.handle}; ``nearest'' means the handler for
which the \nt{stmt.block}{compound-statement} or
\nt{class.base.init}{ctor-initializer} following the \cpp{try} keyword was
most recently entered by the thread of control and not yet exited.''

This is the reason for the proposed changes to \stdclause{except}.
If ``currently handled exception'' means the exception with the
most recently activated handler within any fiber on the current thread, we can get
\href{https://github.com/secondlife/3p-boost/blob/nat/exstate/tests/nullary_throw.cpp}{the following result}.

\cppfl{nullary_throw}

Worse still, the exceptions in question aren't necessarily related to each
other, and line 36 is more likely to read \cpp{catch (const Bad& caught)} --
in which case the \cpp{throw;} on line 34 would \emph{not} be caught.

\newpage
\abschnitt{Appendix C: \exfns}\label{exfns}

\href{https://github.com/secondlife/3p-boost/blob/nat/exstate/tests/fccurrexc.cpp}{The following program}
illustrates the output of \exfns in cases involving fiber context
switches within a destructor invoked by exception handling, and within a catch
block.

\cppf{fccurrexc}

With fiber-specific exception state, \uncexs never exceeds 1, and \curex
displays:

\cpp{fiber() catch block after: std::current\_exception() = fiber() exception}

and:

\cpp{main() catch block after: std::current\_exception() = main() exception}

Without fiber-specific exception state, \uncexs displays up to 2 (one
exception in \main, one in \cpp{fiber()}), and \curex displays:

\cpp{fiber() catch block after: std::current\_exception() = main() exception}

and:

\cpp{main() catch block after: std::current\_exception() = fiber() exception}

\newpage
\abschnitt{Appendix A: support code for examples}\label{launch}\label{appendixa}

Destroying a non-empty \fiber instance invokes Undefined Behaviour
(see \nameref{termination}). To simplify code examples in
this paper, we introduce an \cpp{autocancel} wrapper class that tracks the
sequence of \fiber instances representing a particular fiber. When
an \cpp{autocancel} instance is destroyed, it calls \reqstop on the
captured \source and loops until the fiber voluntarily terminates.

\cppf{autocancel}

\newpage
\addcontentsline{toc}{subsection}{references}
\begin{thebibliography}{99}

    \bibitem{SYSVAMD64}
        \href{https://software.intel.com/sites/default/files/article/402129/mpx-linux64-abi.pdf}{SYS V AMD64 unwinding}

    \bibitem{WinX64}
        \href{https://docs.microsoft.com/en-us/cpp/build/exception-handling-x64?view=vs-2019}{x64 Windows unwinding}

    \bibitem{WinARM64}
        \href{https://docs.microsoft.com/en-us/cpp/build/arm64-exception-handling?view=vs-2019}{ARM64 Windows unwinding}

    \bibitem{OpenAcc}
        {Chandrasekaran, Sunita and Juckeland, Guido (2018). "OpenACC for Programmers: Concepts and Strategies", (1st ed.).
         Pearson Education, Inc}

    \bibitem{CUDA}
        {Wilt, Nicolas (2013). "The CUDA Handbook: A Comprehensive Guide to GPU Programming", (1st ed.).
         Addison Wesley}

    \bibitem{Tanenbaum2009}
        {Tannenbaum, Andrew S. (2009). "Operating Systems. Design and Implementation", (3rd ed.).
         Pearson Education, Inc}

    \bibitem{Moura2009}
        \href{http://www.inf.puc-rio.br/~roberto/docs/MCC15-04.pdf}
        {Moura, Ana L\'{u}cia De and Ierusalimschy, Roberto. "Revisiting coroutines".
         ACM Trans. Program. Lang. Syst., Volume 31 Issue 2, February 2009, Article No. 6}

    \bibitem{N3985}
        \href{http://isocpp.org/files/papers/n3985.pdf}
        {N3985: A proposal to add coroutines to the C++ standard library}

    \bibitem{Standard}
        \href{https://www.open-std.org/jtc1/sc22/wg21/docs/papers/2024/n4981.pdf}
        {N4981: Working Draft, Programming Languages -- C++}

    \bibitem{P0099R0}
        \href{http://www.open-std.org/jtc1/sc22/wg21/docs/papers/2015/p0099r0.pdf}
        {P0099R0: A low-level API for stackful context switching}

    \bibitem{P0099R1}
        \href{http://www.open-std.org/jtc1/sc22/wg21/docs/papers/2016/p0099r1.pdf}
        {P0099R1: A low-level API for stackful context switching}

    \bibitem{P0534R3}
        \href{http://www.open-std.org/jtc1/sc22/wg21/docs/papers/2017/p0534r3.pdf}
        {P0534R3: call/cc (call-with-current-continuation): A low-level API for stackful
        context switching}

    \bibitem{P0660R10}
        \href{https://www.open-std.org/jtc1/sc22/wg21/docs/papers/2019/p0660r10.pdf}
        {P0660R10: Stop Tokens and a Joining Thread}

    \bibitem{P0709R4}
        \href{https://www.open-std.org/jtc1/sc22/wg21/docs/papers/2019/p0709r4.pdf}
        {P0709R4: Zero-overhead deterministic exceptions: Throwing values}

    \bibitem{P0772R1}
        \href{https://www.open-std.org/jtc1/sc22/wg21/docs/papers/2018/p0772r1.pdf}
        {P0772R1: Execution Agent Local Storage}

    \bibitem{P0876R0}
        \href{http://www.open-std.org/jtc1/sc22/wg21/docs/papers/2018/p0876r0.pdf}
        {P0876R0: fibers without scheduler}

    \bibitem{P0876R2}
        \href{http://www.open-std.org/jtc1/sc22/wg21/docs/papers/2018/p0876r2.pdf}
        {P0876R2: fibers without scheduler}

    \bibitem{P0876R3}
        \href{http://www.open-std.org/jtc1/sc22/wg21/docs/papers/2018/p0876r3.pdf}
        {P0876R3: fibers without scheduler}

    \bibitem{P0876R5}
        \href{http://www.open-std.org/jtc1/sc22/wg21/docs/papers/2019/p0876r5.pdf}
        {P0876R5: fibers without scheduler}

    \bibitem{P0876R6}
        \href{http://www.open-std.org/jtc1/sc22/wg21/docs/papers/2019/p0876r6.pdf}
        {P0876R6: fibers without scheduler}

    \bibitem{D0876R7}
        \href{http://wiki.edg.com/pub/Wg21cologne2019/SG1/D0876R7.pdf}
        {D0876R7: fibers without scheduler}

    \bibitem{P0876R8}
        \href{http://www.open-std.org/jtc1/sc22/wg21/docs/papers/2019/p0876r8.pdf}
        {P0876R8: fibers without scheduler}

    \bibitem{P0876R9}
        \href{http://www.open-std.org/jtc1/sc22/wg21/docs/papers/2019/p0876r9.pdf}
        {P0876R9: fibers without scheduler}

    \bibitem{P0876R10}
        \href{https://www.open-std.org/jtc1/sc22/wg21/docs/papers/2020/p0876r10.pdf}
        {P0876R10: fibers without scheduler}

    \bibitem{P0876R11}
        \href{https://www.open-std.org/jtc1/sc22/wg21/docs/papers/2022/p0876r11.pdf}
        {P0876R11: fibers without scheduler}

    \bibitem{P0876R12}
        \href{https://isocpp.org/files/papers/P0876R12.pdf}
        {P0876R12: fibers without scheduler}

    \bibitem{P0876R13}
        \href{https://www.open-std.org/jtc1/sc22/wg21/docs/papers/2023/p0876r13.pdf}
        {P0876R13: fibers without scheduler}

    \bibitem{P0876R14}
        \href{https://www.open-std.org/jtc1/sc22/wg21/docs/papers/2023/p0876r14.pdf}
        {P0876R14: fibers without scheduler}

    \bibitem{P0876R15}
        \href{https://www.open-std.org/jtc1/sc22/wg21/docs/papers/2024/p0876r15.pdf}
        {P0876R15: fibers without scheduler}

    \bibitem{P0876R16}
        \href{https://www.open-std.org/jtc1/sc22/wg21/docs/papers/2024/p0876r16.pdf}
        {P0876R16: fibers without scheduler}

    \bibitem{P0876R17}
        \href{https://www.open-std.org/jtc1/sc22/wg21/docs/papers/2024/p0876r17.pdf}
        {P0876R17: fibers without scheduler}

    \bibitem{P1677R2}
        \href{https://www.open-std.org/jtc1/sc22/wg21/docs/papers/2019/p1677r2.pdf}
        {P1677R2: Cancellation is serendipitous-success}

    \bibitem{P1820R0}
        \href{https://www.open-std.org/jtc1/sc22/wg21/docs/papers/2019/p1820r0.html}
        {P1820R0: Recommendations for a compromise on handling errors and cancellations in executors}

    \bibitem{P2175R0}
        \href{https://www.open-std.org/jtc1/sc22/wg21/docs/papers/2020/p2175r0.html}
        {P2175R0: Composable cancellation for sender-based async operations}

    \bibitem{coreguidlines}
        \href{http://isocpp.github.io/CppCoreGuidelines/CppCoreGuidelines#Ri-global}
        {C++ Core Guidelines}

    \bibitem{SYSVABI}
        \href{http://software.intel.com/sites/default/files/article/402129/mpx-linux64-abi.pdf}
        {System V Application Binary Interface AMD64 Architecture Processor
        Supplement}

    \bibitem{bcontext}
        \href{http://www.boost.org/doc/libs/release/libs/context/doc/html/index.html}
        {Library \emph{Boost.Context}}

    \bibitem{bcoroutine2}
        \href{http://www.boost.org/doc/libs/release/libs/coroutine2/doc/html/index.html}
        {Library \emph{Boost.Coroutine2}}

    \bibitem{bfiber}
        \href{http://www.boost.org/doc/libs/release/libs/fiber/doc/html/index.html}
        {Library \emph{Boost.Fiber}}

    \bibitem{fbmcrouter}
        \href{https://code.facebook.com/posts/296442737213493/introducing-mcrouter-a-memcached-protocol-router-for-scaling-memcached-deployments}
        {Facebook's \emph{mcrouter}}

    \bibitem{fbthrift}
        \href{https://github.com/facebook/fbthrift}
        {Facebook's \emph{Thrift}}

    \bibitem{fbfiber}
        \href{https://github.com/facebook/folly/tree/master/folly/fibers}
        {Facebook's \emph{folly::fibers}}

    \bibitem{bbquantum}
        \href{https://github.com/bloomberg/quantum}
        {Bloomberg's \emph{quantum}}

    \bibitem{habanero}
		\href{https://wiki.rice.edu/confluence/display/HABANERO/Habanero+Extreme+Scale+Software+Research+Project}
		{Habanero Extreme Scale Software Research Project}

    \bibitem{hclib}
		\href{https://github.com/habanero-rice/hclib}
		{Habanero HClib}

    \bibitem{synca}
        \href{https://github.com/gridem/Synca}
        {Library \emph{Synca}}

    \bibitem{tbb}
        \href{https://github.com/intel/tbb}
        {Intels's \emph{TBB}}

    \bibitem{userver}
		\href{https://github.com/userver-framework}
		{userver - The C++ Framework}

\end{thebibliography}


%//////////////////////////////////////////////////////////////////////////////

\end{document}
