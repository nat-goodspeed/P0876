\setlength{\parindent}{0pt} 
\renewcommand\sfdefault{phv}

\makeatletter
    \renewcommand*\l@subsection{\@dottedtocline{2}{0em}{2.3em}}
    \renewcommand*\l@subsection{\@dottedtocline{3}{0em}{3.2em}}
    \renewcommand{\tableofcontents}{\@starttoc{toc}}
\makeatother

\MakePerPage{footnote}
\renewcommand*{\thefootnote}{\fnsymbol{footnote}}

\newcommand{\pdfimg}[1]{\pdfximage{pics/#1}\pdfrefximage\pdflastximage}
\newcommand{\img}[1]{\mbox{\pdfimg{#1}}}
\newcommand{\imgc}[1]{\begin{center}\img{#1}\end{center}}
\newcommand{\graph}[1]{\input{graphs/#1}}
\newcommand{\graphc}[1]{\begin{center}\graph{#1}\end{center}}
\newcommand{\bfs}[1]{{\bfseries #1}}
\newcommand{\zs}[1]{\begin{boxedminipage}[t]{16.8cm}\bfs{#1}\end{boxedminipage}}

\newcommand{\cpp}[1]{{\lstinline[
		basicstyle=\ttfamily\small\color{black},
        breakatwhitespace=true,
        breaklines=true,
        captionpos=b,
        columns=flexible,
        commentstyle=\ttfamily\color{red},
        keepspaces=true,
        keywordstyle=\ttfamily\color{blue},
        language={C++},
        morekeywords={},
        showspaces=false,
        showstringspaces=false,
        showtabs=false,
        stringstyle=\ttfamily\color{magenta}
] !#1!}\xspace}
\newcommand{\cppf}[1]{\lstinputlisting[
		basicstyle=\ttfamily\small\color{black},
        breakatwhitespace=true,
        breaklines=true,
        captionpos=b,
        columns=flexible,
        commentstyle=\ttfamily\color{red},
        keepspaces=true,
        keywordstyle=\ttfamily\color{blue},
        language={C++},
        morekeywords={},
        showspaces=false,
        showstringspaces=false,
        showtabs=false,
        stringstyle=\ttfamily\color{magenta}
] {code/#1.cpp}}
\newcommand{\cppfl}[1]{\lstinputlisting[
		basicstyle=\ttfamily\small\color{black},
        breakatwhitespace=true,
        breaklines=true,
        captionpos=b,
        columns=flexible,
        commentstyle=\ttfamily\color{red},
        keepspaces=true,
        keywordstyle=\ttfamily\color{blue},
        language={C++},
        morekeywords={},
        numbers=left,
        showspaces=false,
        showstringspaces=false,
        showtabs=false,
        stringstyle=\ttfamily\color{magenta}
] {code/#1.cpp}}

%% based on https://tex.stackexchange.com/a/147883
\makeatletter
\newcommand{\dotref}[1][]{%
  \def\@tempa{#1}%
  \ifx\@tempa\@empty
  \else
\cpp{#1.}
  \fi
}

\newcommand{\red}[1]{\colorbox{red}{#1}}
\newcommand{\green}[1]{\colorbox{green}{#1}}
\newcommand{\delete}[1]{\red{#1}}
%% \insert is already defined, with a different meaning
\newcommand{\add}[1]{\green{#1}}
\newcommand{\replace}[2]{\delete{#1}\add{#2}}
\newcommand{\eel}[2]{\href{https://eel.is/c++draft/#1}{#2}}
\newcommand{\nt}[2]{\eel{#1\#nt:#2}{\emph{#2}}}
\newcommand{\stdclause}[1]{\eel{#1}{[#1]}}                          
\newcommand{\xref}[1]{(\stdclause{#1})}
\newcommand{\stdterm}[2]{#1\xspace\xref{#2}}
\newcommand{\stdsection}[2]{§#1\xspace\stdclause{#2}}
\newcommand{\true}{\cpp{true}}
\newcommand{\false}{\cpp{false}}
\newcommand{\dtor}{\cpp{\~fiber\_context()}}
\newcommand{\main}{\cpp{main()}}
\newcommand{\justmain}{\cpp{main}}
\newcommand{\stdfiber}{std::\fiber}
\newcommand{\fiber}{\cpp{fiber\_context}}
\newcommand{\op}[1][]{\dotref[#1]\cpp{operator()()}}
\newcommand{\opbool}[1][]{\dotref[#1]\cpp{operator bool()}}
\newcommand{\resume}[1][]{\dotref[#1]\cpp{resume()}}
\newcommand{\resumewith}[1][]{\dotref[#1]\cpp{resume\_with()}}
\newcommand{\xtresume}[1][]{\dotref[#1]\cpp{resume\_from\_any\_thread()}}
\newcommand{\xtresumewith}[1][]{\dotref[#1]\cpp{resume\_from\_any\_thread\_with()}}
\newcommand{\someresume}[1][]{\resume[#1]} %% {\resume[#1] or \xtresume[#1]}
\newcommand{\anyresumewith}[1][]{\resumewith[#1]} %% {\resumewith[#1] or \xtresumewith[#1]}
\newcommand{\allresumewith}[1][]{\resumewith[#1]} %% {\resumewith[#1] and \xtresumewith[#1]}
\newcommand{\resumesome}[1][]{\resume[#1] or \resumewith[#1]}
\newcommand{\xtresumesome}[1][]{\xtresume[#1] or \xtresumewith[#1]}
\newcommand{\anyresume}[1][]{\resume[#1] or \resumewith[#1]} %% {\resume[#1], \resumewith[#1], \xtresume[#1] or \xtresumewith[#1]}
\newcommand{\allresume}[1][]{\resume[#1] and \resumewith[#1]} %% {\resume[#1], \resumewith[#1], \xtresume[#1] and \xtresumewith[#1]}
\newcommand{\cancel}[1][]{\dotref[#1]\cpp{cancel()}}
\newcommand{\xtcancel}[1][]{\dotref[#1]\cpp{cancel\_from\_any\_thread()}}
\newcommand{\anycancel}[1][]{\cancel[#1]}
\newcommand{\emptyfn}[1][]{\dotref[#1]\cpp{empty()}}
%% referenced in history.tex
\newcommand{\source}{\cpp{stop\_source}}
\newcommand{\getsource}{\cpp{get\_stop\_source()}}
\newcommand{\gettoken}{\cpp{get\_stop\_token()}}
\newcommand{\reqstop}{\cpp{request\_stop()}}
\newcommand{\canresume}[1][]{\dotref[#1]\cpp{can\_resume()}}
\newcommand{\canxtresume}[1][]{\dotref[#1]\cpp{can\_resume\_from\_this\_thread()}}
\newcommand{\canresumesome}[1][]{\canresume[#1]}
\newcommand{\thread}{\cpp{std::thread}}
\newcommand{\jthread}{\cpp{std::jthread}}
\newcommand{\Currthread}{The calling thread\xspace}
\newcommand{\currthread}{the calling thread\xspace}
\newcommand{\ownthread}{the owning thread of \thisfiber}
\newcommand{\thisfiber}{\thefiber{\this}}
\newcommand{\Thefiber}[1]{The fiber represented by #1}
\newcommand{\thefiber}[1]{the fiber represented by #1}
\newcommand{\this}{\cpp{*this}}
\newcommand{\unwindex}{\cpp{std::unwind\_exception}}
\newcommand{\unwindfib}{\cpp{std::unwind\_fiber()}}
\newcommand{\uwforced}{\cpp{\_Unwind\_ForcedUnwind()}}
\newcommand{\curex}{\cpp{std::current\_exception()}}
\newcommand{\uncexs}{\cpp{std::uncaught\_exceptions()}}
\newcommand{\exfns}{\uncexs and \curex}
\newcommand{\catchall}{\cpp{catch (...)}}
\newcommand{\swapcontext}{\cpp{swapcontext()}}
\newcommand{\seeappalaunch}{(See \nameref{appendixa} for the definition of \cpp{autocancel}.)}

\newcommand{\sym}{\emph{symmetric}\xspace}
\newcommand{\asym}{\emph{asymmetric}\xspace}
\newcommand{\entryfn}{entry-function\xspace}
\newcommand{\cancelfn}{cancellation-function\xspace}
\newcommand{\dbframe}{\emph{.debug\_frame\xspace}}
\newcommand{\ehframe}{\emph{.eh\_frame}\xspace}
\newcommand{\foreignex}{\emph{foreign exception}\xspace}

\newcommand{\abschnitt}[1]{\addcontentsline{toc}{subsection}{#1}\subsection*{#1}}
\newcommand{\uabschnitt}[1]{\paragraph*{#1}}

\newcommand{\tsabschnitt}[1]{\subsection[]{#1}}
\newcommand{\tsuabschnitt}[1]{\subsubsection[]{#1}}
\newcommand{\tsparagraph}[1]{\paragraph[]{#1}}

\newcommand{\bcontext}{
        \href{http://www.boost.org/doc/libs/release/libs/context/doc/html/index.html}
        {\emph{Boost.Context}}}
\newcommand{\bcoroutine}{
        \href{http://www.boost.org/doc/libs/release/libs/coroutine2/doc/html/index.html}
        {\emph{Boost.Coroutine2}}}
\newcommand{\bfiber}{
        \href{http://www.boost.org/doc/libs/release/libs/fiber/doc/html/index.html}
        {\emph{Boost.Fiber}}}
\newcommand{\fbmcrouter}{
        \href{https://code.facebook.com/posts/296442737213493/introducing-mcrouter-a-memcached-protocol-router-for-scaling-memcached-deployments}
        {\emph{mcrouter}}}
\newcommand{\fbfibers}{
        \href{https://github.com/facebook/folly/tree/master/folly/fibers}
        {\emph{folly::fibers}}}
\newcommand{\fbthrift}{
        \href{https://github.com/facebook/fbthrift}
        {\emph{Thrift}}}
\newcommand{\bbquantum}{
        \href{https://github.com/bloomberg/quantum}
        {\emph{quantum}}}
\newcommand{\hclib}{
        \href{https://github.com/habanero-rice/hclib}
        {\emph{HClib}}}
\newcommand{\synca}{
        \href{https://github.com/gridem/Synca}
        {\emph{Synca}}}
\newcommand{\tbb}{
        \href{https://github.com/intel/tbb}
        {\emph{TBB}}}
\newcommand{\userver}{
        \href{https://github.com/userver-framework}
        {\emph{userver}}}

\def\Sec#1[#2]#3{%
\ifcase#1\let\s=\chapter
      \or\let\s=\section
      \or\let\s=\subsection
      \or\let\s=\subsubsection
      \or\let\s=\paragraph
      \or\let\s=\subparagraph
      \fi%
\s[#3]{#3\hfill[#2]}\label{#2}}

\newcounter{SectionDepthBase}
\newcounter{scratch}

\def\rSec#1[#2]#3{%
\setcounter{scratch}{#1}                      
\addtocounter{scratch}{\value{SectionDepthBase}}
\Sec{\arabic{scratch}}[#2]{#3}}

\newcommand{\para}{\paragraph{}}
%%\renewcommand{\theparagraph}{\arabic{paragraph}}

\newcommand{\EnterBlock}[1]{[\,\textit{#1:}\space}
\newcommand{\ExitBlock}[1]{\textit{\,---\,end #1}\,]\xspace}
\newcommand{\enternote}{\EnterBlock{Note}}
\newcommand{\enternoten}[1]{\EnterBlock{Note #1}}
\newcommand{\exitnote}{\ExitBlock{note}}

\newcommand{\tsnote}[1]{\enternote {#1} \exitnote}
\newcommand{\tsnoten}[2]{\enternoten{#1}{#2} \exitnote}
\newcommand{\mbrhdr}[1]{\subparagraph*{#1};\\}

%% The point of these macros is that we might need to adjust the formatting
%% for all such subheadings.
\newcommand{\constraints}{\textit{Constraints:}\xspace}
\newcommand{\mandates}{\textit{Mandates:}\xspace}
\newcommand{\requires}{\textit{Requires:}\xspace}
\newcommand{\remarks}{\textit{Remarks:}\xspace}
\newcommand{\effects}{\textit{Effects:}\xspace}
\newcommand{\returns}{\textit{Returns:}\xspace}
\newcommand{\precond}{\textit{Preconditions:}\xspace}
\newcommand{\postcond}{\textit{Postconditions:}\xspace}
\newcommand{\except}{\textit{Throws:}\xspace}
\newcommand{\errors}{\textit{Error conditions:}\xspace}
