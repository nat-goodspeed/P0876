\setlength{\parindent}{0pt} 
\renewcommand\sfdefault{phv}

\makeatletter
    \renewcommand*\l@subsection{\@dottedtocline{2}{0em}{2.3em}}
    \renewcommand*\l@subsection{\@dottedtocline{3}{0em}{3.2em}}
    \renewcommand{\tableofcontents}{\@starttoc{toc}}
\makeatother

\MakePerPage{footnote}
\renewcommand*{\thefootnote}{\fnsymbol{footnote}}

\newcommand{\pdfimg}[1]{\pdfximage{pics/#1}\pdfrefximage\pdflastximage}
\newcommand{\img}[1]{\mbox{\pdfimg{#1}}}
\newcommand{\imgc}[1]{\begin{center}\img{#1}\end{center}}
\newcommand{\graph}[1]{\input{graphs/#1}}
\newcommand{\graphc}[1]{\begin{center}\graph{#1}\end{center}}
\newcommand{\bfs}[1]{{\bfseries #1}}
\newcommand{\zs}[1]{\begin{boxedminipage}[t]{16.8cm}\bfs{#1}\end{boxedminipage}}

\newcommand{\cpp}[1]{{\lstinline[
		basicstyle=\ttfamily\small\color{black},
        breakatwhitespace=true,
        breaklines=true,
        captionpos=b,
        columns=flexible,
        commentstyle=\ttfamily\color{red},
        keepspaces=true,
        keywordstyle=\ttfamily\color{blue},
        language={C++},
        morekeywords={},
        showspaces=false,
        showstringspaces=false,
        showtabs=false,
        stringstyle=\ttfamily\color{magenta}
] !#1!}\xspace}
\newcommand{\cppf}[1]{\lstinputlisting[
		basicstyle=\ttfamily\small\color{black},
        breakatwhitespace=true,
        breaklines=true,
        captionpos=b,
        columns=flexible,
        commentstyle=\ttfamily\color{red},
        keepspaces=true,
        keywordstyle=\ttfamily\color{blue},
        language={C++},
        morekeywords={},
        showspaces=false,
        showstringspaces=false,
        showtabs=false,
        stringstyle=\ttfamily\color{magenta}
] {code/#1.cpp}}
\newcommand{\cppfl}[1]{\lstinputlisting[
		basicstyle=\ttfamily\small\color{black},
        breakatwhitespace=true,
        breaklines=true,
        captionpos=b,
        columns=flexible,
        commentstyle=\ttfamily\color{red},
        keepspaces=true,
        keywordstyle=\ttfamily\color{blue},
        language={C++},
        morekeywords={},
        numbers=left,
        showspaces=false,
        showstringspaces=false,
        showtabs=false,
        stringstyle=\ttfamily\color{magenta}
] {code/#1.cpp}}

\newcommand{\dtor}{\cpp{\~fiber\_context()}}
\newcommand{\main}{\cpp{main()}}
\newcommand{\fiber}{\cpp{std::fiber\_context}}
\newcommand{\op}{\cpp{operator()()}}
\newcommand{\opbool}{\cpp{operator bool()}}
\newcommand{\resume}{\cpp{resume()}}
\newcommand{\resumewith}{\cpp{resume\_with()}}
\newcommand{\xtresume}{\cpp{resume\_from\_any\_thread()}}
\newcommand{\xtresumewith}{\cpp{resume\_from\_any\_thread\_with()}}
\newcommand{\someresume}{\resume or \xtresume}
\newcommand{\someresumewith}{\resumewith or \xtresumewith}
\newcommand{\resumesome}{\resume or \resumewith}
\newcommand{\xtresumesome}{\xtresume or \xtresumewith}
\newcommand{\allresume}{\resume, \resumewith, \xtresume or \xtresumewith}
\newcommand{\cancel}{\cpp{cancel()}}
\newcommand{\xtcancel}{\cpp{cancel\_from\_any\_thread()}}
\newcommand{\anycancel}{\cancel or \xtcancel}
\newcommand{\canresume}{\cpp{can\_resume()}}
\newcommand{\canxtresume}{\cpp{can\_resume\_from\_this\_thread()}}
\newcommand{\canresumesome}{\canresume or \canxtresume}
\newcommand{\thread}{\cpp{std::thread}}
\newcommand{\Currthread}{The calling thread\xspace}
\newcommand{\currthread}{the calling thread\xspace}
\newcommand{\lastthread}{the thread on which the fiber represented by \cpp{*this} was most recently resumed}
\newcommand{\unwindex}{\cpp{std::unwind\_exception}}
\newcommand{\unwindfib}{\cpp{std::unwind\_fiber()}}
\newcommand{\uwforced}{\cpp{_Unwind_ForcedUnwind()}}
\newcommand{\curex}{\cpp{std::current_exception()}}
\newcommand{\uncex}{\cpp{std::uncaught_exception()}}
\newcommand{\uncexs}{\cpp{std::uncaught_exceptions()}}
\newcommand{\catchall}{\cpp{catch (...)}}

\newcommand{\sym}{\emph{symmetric}\xspace}
\newcommand{\asym}{\emph{asymmetric}\xspace}
\newcommand{\entryfn}{entry-function}
\newcommand{\cancelfn}{cancellation-function}
\newcommand{\dbframe}{\emph{.debug\_frame}}
\newcommand{\ehframe}{\emph{.eh\_frame}}
\newcommand{\foreignex}{\emph{foreign exception}\xspace}

\newcommand{\abschnitt}[1]{\addcontentsline{toc}{subsection}{#1}\subsection*{#1}}
\newcommand{\uabschnitt}[1]{\paragraph*{#1}}

\newcommand{\tsabschnitt}[1]{\subsection[]{#1}}
\newcommand{\tsuabschnitt}[1]{\subsubsection[]{#1}}
\newcommand{\tsparagraph}[1]{\paragraph[]{#1}}

\newcommand{\bcontext}{
        \href{http://www.boost.org/doc/libs/release/libs/context/doc/html/index.html}
        {\emph{Boost.Context}}}
\newcommand{\bcoroutine}{
        \href{http://www.boost.org/doc/libs/release/libs/coroutine2/doc/html/index.html}
        {\emph{Boost.Coroutine2}}}
\newcommand{\bfiber}{
        \href{http://www.boost.org/doc/libs/release/libs/fiber/doc/html/index.html}
        {\emph{Boost.Fiber}}}
\newcommand{\fbmcrouter}{
        \href{https://code.facebook.com/posts/296442737213493/introducing-mcrouter-a-memcached-protocol-router-for-scaling-memcached-deployments}
        {\emph{mcrouter}}}
\newcommand{\fbfibers}{
        \href{https://github.com/facebook/folly/tree/master/folly/fibers}
        {\emph{folly::fibers}}}
\newcommand{\fbthrift}{
        \href{https://github.com/facebook/fbthrift}
        {\emph{Thrift}}}
\newcommand{\bbquantum}{
        \href{https://github.com/bloomberg/quantum}
        {\emph{quantum}}}
\newcommand{\hclib}{
        \href{https://github.com/habanero-rice/hclib}
        {\emph{HClib}}}
\newcommand{\synca}{
        \href{https://github.com/gridem/Synca}
        {\emph{Synca}}}

\def\Sec#1[#2]#3{%
\ifcase#1\let\s=\chapter
      \or\let\s=\section
      \or\let\s=\subsection
      \or\let\s=\subsubsection
      \or\let\s=\paragraph
      \or\let\s=\subparagraph
      \fi%
\s[#3]{#3\hfill[#2]}\label{#2}}

\newcounter{SectionDepthBase}
\newcounter{scratch}

\def\rSec#1[#2]#3{%
\setcounter{scratch}{#1}                      
\addtocounter{scratch}{\value{SectionDepthBase}}
\Sec{\arabic{scratch}}[#2]{#3}}

\newcommand{\EnterBlock}[1]{[\,\textit{#1:}\space}
\newcommand{\ExitBlock}[1]{\textit{\,---\,end #1}\,]\xspace}
\newcommand{\enternote}{\EnterBlock{Note}}
\newcommand{\exitnote}{\ExitBlock{note}}

\newcommand{\tsnote}[1]{\enternote {#1} \exitnote}
\newcommand{\mbrhdr}[1]{\subparagraph*{#1};\\}

%% The point of these macros is that we might need to adjust the formatting
%% for all such subheadings.
\newcommand{\constraints}{\textit{Constraints:}\xspace}
\newcommand{\requires}{\textit{Requires:}\xspace}
\newcommand{\remarks}{\textit{Remarks:}\xspace}
\newcommand{\effects}{\textit{Effects:}\xspace}
\newcommand{\returns}{\textit{Returns:}\xspace}
\newcommand{\postcond}{\textit{Ensures:}\xspace}
\newcommand{\except}{\textit{Throws:}\xspace}
