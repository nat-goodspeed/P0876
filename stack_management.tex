\abschnitt{encapsulating the stack}\label{stackmgmt}

Each fiber is associated with a stack and is responsible to manage the lifespan
of its stack (allocation at construction, deallocated if fiber terminates). The
RAII-pattern \footnote{resource acquisition is initialisation} should apply.\\

Copying a fiber must not be permitted!\\
If a fiber were copyable, then its stack with all the objects allocated on it
would be copied too. That would force undefined behaviour if some of these
objects were RAII-classes (manage a resource via RAII pattern). When the first
of the fiber copies terminates (unwinds its stack), the RAII class destructors
will release their managed resources. When the second copy terminates, the same
destructors will try to doubly-release the same resources, leading to undefined
behavior.\\

\zs{
A fiber API has to:
\begin{itemize}
    \item encapsulate the stack (hiding for user)
    \item manage lifespan of an explicitly-allocated stack: the stack gets
          deallocated when fiber goes out of scope
    \item prevent accidentally copying the stack
\end{itemize}
Class \fiber must be \emph{moveable-only}.\\
}
