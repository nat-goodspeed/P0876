\abschnitt{encapsulating the stack}\label{stackmgmt}

Each fiber is associated with a stack and is responsible for managing the lifespan
of its stack (allocation at construction, deallocation when fiber terminates). The
RAII-pattern\footnote{resource acquisition is initialisation} should apply.\\

Copying a fiber must not be permitted!\\
If a fiber were copyable, then its stack with all the objects allocated on it
must be copied too. That presents two implementation choices.
\begin{itemize}
    \item One approach would be to capture sufficient metadata to permit
          object-by-object copying of stack contents. That would require
          dramatically more runtime information than is presently available --
          and would take considerably more overhead than a coder might expect.
          Naturally, any one move-only object on the stack would prohibit
          copying the entire stack.
    \item The other approach would be a bytewise copy of the memory occupied
          by the stack. That would force undefined behaviour if any stack
          objects were RAII-classes (managing a resource via RAII pattern). When the first
          of the fiber copies terminates (unwinds its stack), the RAII class destructors
          will release their managed resources. When the second copy terminates, the same
          destructors will try to doubly-release the same resources, leading to undefined
          behavior.
\end{itemize}

\zs{
A fiber API must:
\begin{itemize}
    \item encapsulate the stack (hiding for user)
    \item manage lifespan of an explicitly-allocated stack: the stack gets
          deallocated when fiber goes out of scope
    \item prevent accidentally copying the stack
\end{itemize}
Class \fiber must be \emph{moveable-only}.\\
}
