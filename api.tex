\newpage
\abschnitt{Wording}\label{api}

This wording is relative to N4971.\cite{Standard}

\zs{Append to §3.6 \stdclause{defns.block} as indicated:}

\add{\tsnoten{1 to entry}{Unless stated otherwise, blocking blocks the current
thread.}}

\zs{Modify §4.1.2 \stdclause{intro.abstract} paragraph 6.3 as indicated:}

\begin{description}
    \item[---] The input and output dynamics of interactive devices shall take
               place in such a fashion that prompting output is actually
               delivered before \replace{a program}{an input operation} waits
               for input. What constitutes an interactive device is
               implementation-defined.
\end{description}

%*****************************************************************************
%   TODO:
% * [defns.deadlock] threads are unable to continue execution because each is
%   blocked waiting for one or more of the others to satisfy some condition
% * [defns.unblock] blocked threads of execution

% * [intro.execution] the block is suspended (by call of a function,
%   suspension of a coroutine or receipt of a signal); each evaluation that
%   does not occur within F but is evaluated on the same thread
% * [intro.multithread.general] single flow of control
% * [intro.progress] standard library function that blocks intransitively;
%   thread that is not blocked in a std library function; blocking with
%   forward progress guarantee delegation; blocking synchronization
% * [basic.start.main] control flows off the end of main
% * [basic.start.term] flow of control passes through the definition
% * [expr.await] returning control flow; control flow returns
% * [expr.const] a control flow that passes through a decl of a block var
% * [stmt.dcl] control passes through its declaration; control enters the
%   declaration concurrently; the concurrent execution shall wait; upon each
%   transfer of control (including sequential execution of statements)
% * [dcl.fct.def.coroutine] control flows off the end of the coroutine;
%   flowing off the end of a coroutine
% * [except.pre] transferring control and information; transfers control to a
%   handler
% * [except.throw] nearest means the handler for which the try-block was most
%   recently entered by the thread of control and not yet exited
% * [except.handle] control reaches the end of a handler; flowing off the end
%   of the compount-statement
% * [diff.stat] If some flow paths execute a return...
% * [support.start.term] control is returned to the host environment
% * [support.signal] control entering a try-block
% * [stmt.switch] control is passed; control passes
% * [stmt.return] flowing off the end of a function
% * [stmt.return.coroutine] same ^
% * [stmt.jump.general] transfer control
% * [algorithms.parallel.exec] blocking synchronization; thread of execution;
%   block with forward progress guarantee delegation
% * [coroutine.handle.resumption] refs ex agent, thread, jthread, main;
%   coroutine ... "holding a mutex object"
% * [coro.generator.promise] control flow returns
% * [thread.req.timing] discusses "waiting function" without reference to agent
% * [thread.req.lockable.general] defines ex agent and calling agent (but
%   never referenced); ex agent owns a lock (code font; probably a misuse?)
% * [thread.req.lockable.basic] refs current ex agent; intransitive "blocks";
%   "without blocking"; ex agent holds a lock
% * [thread.req.lockable.shared] same ^
% * [thread.mutex.requirements.general] an ex agent owns a mutex
% * [thread.mutex.requirements.mutex.general] m.lock() blocks the calling
%   thread until ownership of the mutex can be obtained for the calling
%   thread; try_lock() without blocking
% * [thread.mutex.class] If one thread owns a mutex object, attempts by
%   another thread to acquire ownership of that object ... will block (for
%   lock()) until the owning thread has released ownership.
% * [thread.mutex.recursive] same ^
% * [thread.sharedmutex.requirements.general] refs ex agents and calling
%   thread; intransitive "block"; ex agents hold a shared lock; blocks the
%   calling thread
% * [thread.timedmutex.requirements.general] try_lock_for(), try_lock_until():
%   the function attempts to obtain ownership of the mutex ... without
%   blocking
% * [thread.sharedtimedmutex.requirements.general] same ^
% * [thread.timedmutex.class] thread owns object, attempts will block
% * [thread.timedmutex.recursive] same ^
% * [thread.lock] refs ex agent and calling thread; lock owns a lockable
%   object
% * [thread.lock.guard] calling thread holds a non-shared lock
% * [thread.lock.scoped] same ^
% * [thread.lock.unique.cons] same ^
% * [thread.lock.shared.cons] same ^
% * [stopcallback.cons] destructor does not block waiting...
% * [thread.thread.member] join() intransitive blocks; detach() ... calling
%   thread blocking
% * [thread.jthread.mem] same ^
% * [thread.thread.class.general] wait for a thread to complete
% * [thread.thread.this] sleep_until(), sleep_for() blocks the calling thread
% * [atomics.lockfree] atomic operations ... potentially block intransitive
% * [atomics.wait] (call to) operation may block intransitive
% * [atomics.ref.ops] wait() blocks intransitive, xref [atomics.wait];
%   notify_one(); notify_all()
% * [atomics.types.operations] same ^
% * [util.smartptr.atomic.shared] same ^
% * [util.smartptr.atomic.weak] same ^
% * [atomics.flag] same ^
% * [futures.state] the provider "unblocks any execution agents waiting";
%   waiting function potentially blocks; actions will not block; the waiting
%   thread
% * [futures.unique.future] wait(), wait_for(), wait_until() blocks
% * [futures.shared.future] same ^
% * [futures.async] a call shall block intransitively; destructor can block;
%   waiting function
% * [futures.task.members] threads blocked
% * [saferecl.rcu.domain.func] rcu_synchronize(), rcu_barrier() blocks intransitively
% * [syncstream.syncbuf.members] emit() may ... hold a lock
% * [thread.condition.general] sync primitives used to block a thread; wait on
%   unique_lock<mutex>
% * [thread.condition.condvar] thread blocked on *this; threads block on the
%   lock specified in the wait; threads blocked waiting for *this; wait(),
%   wait_until() blocks intransitively on *this; concurrently waiting threads
% * [thread.condition.nonmember] waiting threads ... holding the lock; thread
%   waiting on cond
% * [thread.sema.cnt] thread will block; acquire() blocks intransitively on
%   *this; threads waiting
% * [thread.latch.general] threads to block; threads can block
% * [thread.barrier.general] same ^
% * [except.terminate] must mention exception leaving a fiber function, or
%   destroying or assigning to a non-empty fiber_context
%*****************************************************************************

\zs{Modify §6.9.2.1 \stdclause{intro.multithread.general} paragraph 1 as indicated:}

A \emph{thread of execution} (also known as a \emph{thread}) is
\replace{a single flow of control}{the primary execution agent}\\
\add{\xref{thread.req.lockable.general}}
within a program\delete{, including the initial invocation of a specific top-level
function, and}\\
\replace{recursively including every function invocation subsequently
executed by the thread}{. When the host environment first}\\
\add{enters a program, it provides a default thread to perform the program's
execution steps}.

\add{When a thread is created, it runs a default fiber ([intro.fibers]).}

\zs{Insert before §6.9.3 \stdclause{basic.start} and renumber existing 6.9.3 to 6.9.4:}

\setcounter{section}{6}
\setcounter{subsection}{9}
\setcounter{subsubsection}{2}
\setcounter{secnumdepth}{4}

\rSec3[intro.fibers]{Fibers and Threads}

1 A \emph{fiber} is a single flow of control within a program, including the
initial invocation of a specific top-level function, and recursively including
every function invocation subsequently executed by the fiber. The execution
steps of a fiber are performed by a thread.

\tsnote{``Flow of control'' here refers to state necessary to program
execution, for example the contents of a processor's registers including its
instruction pointer, and the invocation sequence \xref{stacktrace.general} of
functions that have been entered but have not yet returned.}

2 A new fiber may be created using \fiber ([fiber.context.class]).
\emph{explicit fiber} denotes a fiber created by program evaluation;
conversely, \emph{implicit fiber} denotes the default fiber on any
thread.
\tsnote{An explicit fiber is useful for distributing computation across
distinct invocation sequences.}

3 Each thread is the \emph{owning thread} of its default fiber. A newly
created explicit fiber has no owning thread until some thread first
calls \anyresume on its \fiber object, at which point the calling thread
becomes that fiber's owning thread. No thread may resume a fiber owned by
another thread. \tsnote{The thread that first resumes a newly created explicit
fiber need not be the same as the thread that created it.}

4 Let fcurrent designate the currently running fiber, and ftarget be a
suspended fiber represented by an object \cpp{fc} of type \fiber. When
fcurrent calls \anyresume[fc], the current thread stops running fcurrent and
instead runs ftarget. If ftarget has not yet been entered, this action enters
its top-level function, else it \emph{resumes} ftarget. fcurrent is left in a
suspended state.

5 The sequence of steps performed by the current thread is the sequence
specified by the current fiber, until one of \fiber's \anyresume member
functions is called. A fiber is itself an execution agent with weakly parallel
forward progress guarantees \xref{intro.progress}, though its steps are
performed by a thread.

\zs{Append to §14.4 \stdclause{except.handle} paragraph 8 as indicated:}

\add{It is implementation-defined whether the currently handled exception
designates the exception with the most recently}\\
\add{activated handler that is still active within:}
\begin{itemize}
    \item \add{the current fiber ([intro.fibers]), or}
    \item \add{the current thread.}
\end{itemize}

\zs{Append to §14.6.3 \stdclause{except.uncaught} paragraph 1 as indicated:}

\add{It is implementation-defined whether \cpp{uncaught\_exceptions()} returns
the number of uncaught exceptions in:}
\begin{itemize}
    \item \add{the current fiber ([intro.fibers]), or}
    \item \add{the current thread.}
\end{itemize}

\zs{Insert new final subclause in clause 33 \stdclause{thread} as indicated:}

\setcounter{section}{33}
\setcounter{subsection}{11}
\setcounter{secnumdepth}{4}

\rSec2[fiber.context]{fiber\_context}

\rSec3[fiber.context.pre]{Overview}

1 A \fiber object is either \emph{empty} or \emph{non-empty}. A
default-constructed \fiber is empty. A moved-from \fiber is empty. A \fiber
representing a suspended fiber is non-empty.

2 When the running fiber returns a \fiber from its \entryfn, thus resuming the
designated fiber, the synthesized \fiber passed into the resumed fiber is
empty.

3 An explicit fiber is instantiated by passing an \emph{\entryfn} to \fiber's
constructor. This function is not entered until the first call to one of
the \anyresume member functions.

4 When a fiber is first entered, a synthesized non-empty \fiber object
representing the newly-suspended previous fiber is passed as a parameter to
its \entryfn. Once entered, a fiber may suspend by calling one of the \anyresume
member functions on any accessible non-empty \fiber object. When the
suspended fiber is resumed, that member function returns a synthesized \fiber object
representing the newly-suspended previous fiber.

5 The synthesized \fiber object received in either of those ways might
represent either an explicit fiber or an implicit fiber.

6 An explicit fiber terminates by returning a non-empty \fiber object from
its \entryfn. \Thefiber{that \fiber object} is resumed.

%% \rSec3[fiber.context.toplevel]{Implicit Top-Level Function}

%% On every explicit fiber, the behaviour is equivalent to calling the \entryfn
%% passed to \fiber's constructor from an implicit top-level function.
%% If the fiber is later
%% unwound, this conceptual top-level stack frame serves as delimiter: this point
%% is where unwinding stops.

7 If the fiber's \entryfn returns an empty \fiber object, \cpp{std::terminate} is called.
If the fiber's \entryfn exits via an exception, \cpp{std::terminate} is called.

8 Regardless of the number of \fiber declarations, exactly one \fiber object
represents each suspended fiber.

%% Returning a \fiber object from the explicit fiber's \entryfn is equivalent
%% to returning control to the implicit top-level function.
%% Similarly,
%% when \unwindfib unwinds a fiber stack, it conceptually returns the \fiber
%% object it was passed to the implicit top-level function. Either way, the
%% The
%% conceptual implicit top-level function is responsible for deallocating the
%% explicit fiber's stack memory on return from the \entryfn.
%% 
%% Similarly, on every implicit fiber, the behaviour is equivalent to passing control through an
%% implicit top-level function above \justmain and above the \entryfn for
%% each \thread.
%% The conceptual stack frame for this implicit top-level function delimits
%% stack unwinding for each of these stacks. If the fiber stack is unwound,
%% control is conceptually returned to this implicit top-level function.
%% The conceptual top-level
%% function for an implicit fiber does not deallocate the fiber's stack memory,
%% since the host environment will do that.

%% \begin{itemize}
%%     \item
%%     \item If an empty \fiber object is returned to the conceptual top-level
%%     function for an explicit fiber, the calling thread is terminated.
%%     \item If an empty \fiber object is returned to the conceptual top-level
%%     function for the default fiber of an explicit thread, that thread is
%%     terminated.
%%     \item If an empty \fiber object is returned to the conceptual top-level
%%     function above \justmain, the process is terminated.
%% \end{itemize}

%--------------------------------- class def ---------------------------------
\rSec3[fiber.context.class]{Class fiber\_context}

\cppf{synopsis}
\cppf{fiber}

\newcommand{\state}{\cpp{state}}

\rSec4[fiber.context.cons]{Constructors, move and assignment}

%---------------------------- implicit stack ctor ----------------------------
\mbrhdr{template<class F> explicit fiber\_context(F\&\& entry)}

1 \mandates
\begin{description}
    \item[---] \cpp{is\_constructible\_v<decay\_t<F>, F>} is \true.
    \item[---] \cpp{is\_invocable\_r\_v<fiber\_context, decay\_t<F>,
               fiber\_context&&>} is \true.
\end{description}

2 \constraints
\begin{description}
    \item[---] \cpp{remove\_cvref\_t<F>} is not the same type as \fiber.
    \item[---] \cpp{F} meets the \emph{Cpp17CopyConstructible} requirements.
\end{description}

3 \effects
\begin{description}
    \item[---] Let \cpp{entry\_copy} be a copy of \cpp{entry}.
               \tsnote{\cpp{entry\_copy} is not a member of \fiber because it
               is destroyed on fiber termination, not when a \fiber object is
               destroyed.}
    \item[---] Initializes \cpp{state} to represent a fiber suspended before
              entry to \cpp{entry\_copy}.
              \tsnote{\cpp{entry\_copy} is entered only when \anyresume is called.}
    \item[---] Any necessary resources are created. \tsnote{This includes the
               new fiber's function call stack.}
\end{description}

4 \postcond
\emptyfn is \false.

5 \except
\begin{description}
    \item[---] \cpp{bad\_alloc} if unable to acquire a new function call stack.
    \item[---] \cpp{system\_error} if unable to start the new fiber.
    \item[---] Any exception thrown by the selected constructor of \cpp{entry}.
\end{description}

6 \errors
\cpp{resource\_unavailable\_try\_again} -- the system lacked the necessary resources to create another fiber.

%---------------------------- explicit stack ctor ----------------------------
\mbrhdr{template<class F, class D> fiber\_context(F\&\& entry, span<byte> stack, D\&\& deleter)}

1 \mandates
\begin{description}
    \item[---] \cpp{is\_constructible\_v<decay\_t<F>, F>} is \true.
    \item[---] \cpp{is\_constructible\_v<decay\_t<D>, D>} is \true.
    \item[---] \cpp{is\_invocable\_r\_v<fiber\_context, decay\_t<F>,
               fiber\_context&&>} is \true.
    \item[---] \cpp{is\_invocable\_v<decay\_t<D>, span<byte>>} is \true.
\end{description}

2 \constraints
\begin{description}
    \item[---] \cpp{F} meets the \emph{Cpp17CopyConstructible} requirements.
    \item[---] \cpp{D} meets the \emph{Cpp17CopyConstructible} requirements.
\end{description}

3 \effects
\begin{description}
    \item[---] Let \cpp{entry\_copy} be a copy of \cpp{entry}.
    \item[---] Let \cpp{stack\_copy} be a copy of \cpp{stack}.
    \item[---] Let \cpp{deleter\_copy} be a copy of \cpp{deleter}.
               \tsnote{\cpp{entry\_copy}, \cpp{stack\_copy} and
                \cpp{deleter\_copy} are not members of \fiber because they 
               are destroyed on fiber termination, not when a \fiber object is
               destroyed.}
    \item[---] Initializes \cpp{state} to represent a fiber suspended before
              entry to \cpp{entry\_copy}.
              \tsnote{\cpp{entry\_copy} is entered only when \anyresume is called.}
    \item[---] Any necessary resources are created.
              \tsnote{It is the caller's responsibility to provide a span of
              sufficient size for the most deeply nested function calls that
              will be performed by the new fiber. It may be advantageous to
              obtain from the operating system a memory block with a read-only
              guard page to trap stack overflow.}
\end{description}

4 \postcond
\emptyfn is \false.

5 \except
\begin{description}
    \item[---] \cpp{invalid\_argument} if \cpp{stack} fails to meet
               implementation-defined alignment requirements.
    \item[---] \cpp{length\_error} if \cpp{stack} is less than the
               implementation-defined minimum length.
    \item[---] \cpp{system\_error} if unable to start the new fiber.
    \item[---] Any exception thrown by the selected constructor of \cpp{entry}.
    \item[---] Any exception thrown by the selected constructor of \cpp{deleter}.
\end{description}

6 \errors
\cpp{resource\_unavailable\_try\_again} -- the system lacked the necessary resources to create another fiber.

7 \remarks
If at any time during the life of the newly created fiber the
function call stack depth exceeds the size of \cpp{stack}, the behaviour is
undefined.

%----------------------------------- dtor ------------------------------------
\mbrhdr{\cpp{\~fiber\_context()}}

1 \effects
If \emptyfn is \false, \cpp{terminate} is invoked \xref{except.terminate}.

\tsnote{If a \fiber object to be destroyed is not yet empty, an application
must convey to the suspended fiber the need to terminate voluntarily.}

%------------------------------ move assignment ------------------------------
\mbrhdr{fiber\_context\& operator=(fiber\_context\&\& other) noexcept}

1 \effects
\begin{description}
    \item[---] If \emptyfn is \false, \cpp{terminate} is invoked \xref{except.terminate}.
    \item[---] Equivalent to: \cpp{this->state = exchange(other.state, nullptr)}.
\end{description}

2 \returns
\this

3 \postcond
\emptyfn[other] is \true

\rSec4[fiber.context.mem]{Members}
%-------------------------------- resume_with --------------------------------
\mbrhdr{template<class Fn> fiber\_context resume\_with(Fn\&\& fn) \&\&}

1 \mandates
\cpp{is\_invocable\_r\_v<fiber\_context, decay\_t<Fn>, fiber\_context&&>}

2 \precond
\canresume is \true

\newcommand{\continuation}{\cpp{continuation}}

3 Let \cpp{target} be \thefiber{\state}.

4 Let \cpp{caller} be a synthesized \fiber object representing
the suspended calling fiber.

5 \effects
\begin{description}
    \item[---] Resets \state so that \emptyfn is \true.
    \item[---] Switches the current fiber to \cpp{target}.
    \item[---] Before this point, no exceptions are thrown.
    \item[---] Evaluates \cpp{invoke(std::forward<Fn>(fn), std::move(caller))}.
               Let \cpp{returned} be the \fiber object returned by \cpp{fn}.
               \tsnote{\cpp{returned} may or may not be the same as \cpp{caller}.}
               \tsnote{\cpp{returned} may be empty.}
    \item[---] If \cpp{target} previously
               suspended itself by calling one of \anyresume,
               returns \cpp{returned} from that resume function.
    \item[---] Otherwise, \cpp{target} has not yet been entered.
               Passes \cpp{returned} to its \entryfn.
               Let \continuation be the result of executing
               \cpp{invoke\_r<fiber\_context>(entry\_copy, std::move(returned))}. On return:
        \begin{itemize}
            \item destroys \cpp{entry\_copy}
            \item if \cpp{target} has an associated \cpp{stack\_copy} and \cpp{deleter\_copy}:
                \begin{itemize}
                    \item executes \cpp{invoke(deleter\_copy, stack\_copy)}
                    \item destroys \cpp{deleter\_copy}
                \end{itemize}
            \item otherwise reclaims the implementation-provided stack
            \item executes \resume[continuation].
        \end{itemize}
\end{description}

6 \remarks
A newly constructed but not yet resumed fiber may be resumed by
any thread.

7 \returns
\begin{description}
    \item[---] If the previous fiber resumed this one by returning a \fiber,
               an empty \fiber.
    \item[---] If the previous fiber resumed this one by passing some \cpp{fn}
               to \anyresumewith, the \fiber returned by that \cpp{fn}.
\end{description}

8 \except
\begin{description}
%   \item[---] \anyresume throws
%             \unwindex when, while suspended, the \fiber object representing
%             the suspended fiber is destroyed
    \item[---] On being resumed:
    \begin{itemize}
        \item If the previous fiber resumed this one by returning a \fiber:
            \begin{itemize}
                \item Any exception thrown as a result of destroying the
                      previous fiber's associated \cpp{entry\_copy}.
                \item Any exception thrown by the previous fiber's
                      associated \cpp{deleter\_copy}.
                \item Any exception thrown as a result of destroying the
                      previous fiber's associated \cpp{deleter\_copy}.
            \end{itemize}
        \item If the previous fiber resumed this one by calling \anyresumewith:
        \begin{itemize}
            \item Any exception thrown by the \cpp{fn} passed by the previous
                  fiber to \anyresumewith.
        \end{itemize}
    \end{itemize}
\end{description}

9 \postcond
\emptyfn is \true.

\tsnote{The returned \fiber indicates via \emptyfn whether the previous active
fiber has terminated (returned from \entryfn).}

\tsnote{Because \anyresume empties the object on which it is called, these
member functions are rvalue-reference qualified. No \fiber object ever
represents the currently-running fiber.}

%---------------------------------- resume -----------------------------------
\mbrhdr{fiber\_context resume() \&\&}

1 \effects
Equivalent to:\\
\cpp{return resume\_with([](fiber\_context&& caller)\{ return std::move(caller); \});}

%-------------------------------- can_resume ---------------------------------
\mbrhdr{bool can\_resume() noexcept}

1 \returns
\begin{description}
    \item[---] \false if \emptyfn is \true
    \item[---] \true if \thisfiber has no owning thread
    \item[---] \true if \currthread is \ownthread
    \item[---] \false otherwise.
\end{description}

\tsnote{When \canresume is \true, the \fiber object may be resumed
by \anyresume.}

\red{\EnterBlock{Editorial note} \canresume is intentionally not
marked \cpp{const}. \ExitBlock{editorial note}}

%----------------------------------- empty -----------------------------------
\mbrhdr{bool empty() const noexcept}

1 \effects
Equivalent to: \cpp{return (\! state);}.

%------------------------------- operator bool -------------------------------
\mbrhdr{explicit operator bool() const noexcept}

1 \effects
Equivalent to: \cpp{return (\! empty());}.

%---------------------- current_exception_within_fiber -----------------------

\mbrhdr{static constexpr bool current\_exception\_within\_fiber() noexcept}

1 \returns
\true if the implementation of \exfns reports the current exception(s) within
the current fiber, \false if they consider all exceptions within the owning thread.

%----------------------------------- swap ------------------------------------
\mbrhdr{void swap(fiber\_context\& other) noexcept}

1 \effects
Equivalent to: \cpp{swap(this->state, other.state)}.

\rSec4[fiber.context.special]{Specialized algorithms}
\mbrhdr{friend void swap(fiber\_context\& lhs, fiber\_context\& rhs) noexcept}

1 \effects
Equivalent to: \cpp{lhs.swap(rhs)}.


%% \rSec3[fiber.context.unwinding]{Function unwind\_fiber()}
%% 
%% \mbrhdr{[[ noreturn ]] void unwind\_fiber(fiber\_context\&\& other)}
%% 
%% 1 \effects
%% terminate the current running fiber.
%% 
%% 2 \remarks
%% \begin{description}
%%     \item[---] The underlying Unwinding facility (for instance the unwind facility
%%                described in \emph{System V ABI for AMD64}) unwinds the stack
%%                to the implicit top-level stack frame and terminates the
%%                current fiber as described above.
%%     \item[---] Unwinding the fiber's stack causes its stack variables to be
%%                destroyed.
%%     \item[---] During this specific stack unwinding, 
%% %% only \catchall clauses are executed. No other
%%                no \cpp{catch} clauses are executed, not even \catchall.
%%     \item[---] Once the running fiber has been fully unwound, \cpp{other} is
%%                returned to the fiber's conceptual top-level function as
%%                described in \nameref{fiber-context.toplevel}.
%% %%  \item[---] Unwinding the fiber's stack causes relevant \catchall
%% %%             clauses to be executed.
%% %%  \item[---] During this specific stack unwinding, a \catchall
%% %%             clause that does not execute a \cpp{throw;} statement behaves
%% %%             as if it ended with a \cpp{throw;} statement.
%% %%  \item[---] During this specific stack unwinding, if a \catchall
%% %%             clause attempts to throw any C++ exception, the
%% %%             behaviour is undefined.
%% \end{description}
%% 
%% 3 \returns
%% \begin{description}
%%     \item[---] None: \unwindfib does not return
%% \end{description}
%% 
%% 4 \except
%% \begin{description}
%%     \item[---] None catchable by C++
%% \end{description}

\zs{Modify §19.6.1 \stdclause{stacktrace.general} as indicated:}

1 Subclause \stdclause{stacktrace} describes components that C++ programs may use to
store the stacktrace of the current \delete{thread of}\\
\replace{execution}{fiber ([intro.fibers])}
and query information about the stored stacktrace at runtime.

2 The \emph{invocation sequence} of the current evaluation $x_0$
in the current \replace{thread of execution}{fiber} is a sequence
($x_0$,...,$x_n$) of evaluations such that, for $i \geq 0$,
$x_i$ is within the function invocation $x_{i+1}$ \xref{intro.execution}.

\abschnitt{Feature-test Macro}
\zs{Add a new feature-test macro to §17.3.2 \stdclause{version.syn} as indicated:}

\cpp{#define \__cpp\_lib\_fiber\_context 202XXXL // also in <fiber\_context>}
