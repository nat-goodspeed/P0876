\newpage
\abschnitt{Wording}\label{api}

This wording is relative to N4981.\cite{Standard}

\zs{Append to \stdsection{3.6}{defns.block} as indicated:}

\add{\tsnoten{1 to entry}{Unless stated otherwise, blocking blocks the current
thread.}}

\zs{Modify \stdsection{4.1.2}{intro.abstract} paragraph 7.3 as indicated:}

\begin{description}
    \item[---] The input and output dynamics of interactive devices shall take
               place in such a fashion that prompting output is actually
               delivered before \replace{a program}{an input operation} waits
               for input. What constitutes an interactive device is
               implementation-defined.
\end{description}

%*****************************************************************************
%   TODO:
% * [defns.deadlock] threads are unable to continue execution because each is
%   blocked waiting for one or more of the others to satisfy some condition
% * [defns.unblock] blocked threads of execution

% * [intro.execution] the block is suspended (by call of a function,
%   suspension of a coroutine or receipt of a signal); each evaluation that
%   does not occur within F but is evaluated on the same thread
% * [intro.multithread.general] single flow of control
% * [intro.progress] standard library function that blocks intransitively;
%   thread that is not blocked in a std library function; blocking with
%   forward progress guarantee delegation; blocking synchronization
% * [basic.start.main] control flows off the end of main
% * [basic.start.term] flow of control passes through the definition
% * [expr.await] returning control flow; control flow returns
% * [expr.const] a control flow that passes through a decl of a block var
% * [stmt.dcl] control passes through its declaration; control enters the
%   declaration concurrently; the concurrent execution shall wait; upon each
%   transfer of control (including sequential execution of statements)
% * [dcl.fct.def.coroutine] control flows off the end of the coroutine;
%   flowing off the end of a coroutine
% * [except.pre] transferring control and information; transfers control to a
%   handler
% * [except.throw] nearest means the handler for which the try-block was most
%   recently entered by the thread of control and not yet exited
% * [except.handle] control reaches the end of a handler; flowing off the end
%   of the compount-statement
% * [diff.stat] If some flow paths execute a return...
% * [support.start.term] control is returned to the host environment
% * [support.signal] control entering a try-block
% * [stmt.switch] control is passed; control passes
% * [stmt.return] flowing off the end of a function
% * [stmt.return.coroutine] same ^
% * [stmt.jump.general] transfer control
% * [algorithms.parallel.exec] blocking synchronization; thread of execution;
%   block with forward progress guarantee delegation
% * [coroutine.handle.resumption] refs ex agent, thread, jthread, main;
%   coroutine ... "holding a mutex object"
% * [coro.generator.promise] control flow returns
% * [thread.req.timing] discusses "waiting function" without reference to agent
% * [thread.req.lockable.general] defines ex agent and calling agent (but
%   never referenced); ex agent owns a lock (code font; probably a misuse?)
% * [thread.req.lockable.basic] refs current ex agent; intransitive "blocks";
%   "without blocking"; ex agent holds a lock
% * [thread.req.lockable.shared] same ^
% * [thread.mutex.requirements.general] an ex agent owns a mutex
% * [thread.mutex.requirements.mutex.general] m.lock() blocks the calling
%   thread until ownership of the mutex can be obtained for the calling
%   thread; try_lock() without blocking
% * [thread.mutex.class] If one thread owns a mutex object, attempts by
%   another thread to acquire ownership of that object ... will block (for
%   lock()) until the owning thread has released ownership.
% * [thread.mutex.recursive] same ^
% * [thread.sharedmutex.requirements.general] refs ex agents and calling
%   thread; intransitive "block"; ex agents hold a shared lock; blocks the
%   calling thread
% * [thread.timedmutex.requirements.general] try_lock_for(), try_lock_until():
%   the function attempts to obtain ownership of the mutex ... without
%   blocking
% * [thread.sharedtimedmutex.requirements.general] same ^
% * [thread.timedmutex.class] thread owns object, attempts will block
% * [thread.timedmutex.recursive] same ^
% * [thread.lock] refs ex agent and calling thread; lock owns a lockable
%   object
% * [thread.lock.guard] calling thread holds a non-shared lock
% * [thread.lock.scoped] same ^
% * [thread.lock.unique.cons] same ^
% * [thread.lock.shared.cons] same ^
% * [stopcallback.cons] destructor does not block waiting...
% * [thread.thread.member] join() intransitive blocks; detach() ... calling
%   thread blocking
% * [thread.jthread.mem] same ^
% * [thread.thread.class.general] wait for a thread to complete
% * [thread.thread.this] sleep_until(), sleep_for() blocks the calling thread
% * [atomics.lockfree] atomic operations ... potentially block intransitive
% * [atomics.wait] (call to) operation may block intransitive
% * [atomics.ref.ops] wait() blocks intransitive, xref [atomics.wait];
%   notify_one(); notify_all()
% * [atomics.types.operations] same ^
% * [util.smartptr.atomic.shared] same ^
% * [util.smartptr.atomic.weak] same ^
% * [atomics.flag] same ^
% * [futures.state] the provider "unblocks any execution agents waiting";
%   waiting function potentially blocks; actions will not block; the waiting
%   thread
% * [futures.unique.future] wait(), wait_for(), wait_until() blocks
% * [futures.shared.future] same ^
% * [futures.async] a call shall block intransitively; destructor can block;
%   waiting function
% * [futures.task.members] threads blocked
% * [saferecl.rcu.domain.func] rcu_synchronize(), rcu_barrier() blocks intransitively
% * [syncstream.syncbuf.members] emit() may ... hold a lock
% * [thread.condition.general] sync primitives used to block a thread; wait on
%   unique_lock<mutex>
% * [thread.condition.condvar] thread blocked on *this; threads block on the
%   lock specified in the wait; threads blocked waiting for *this; wait(),
%   wait_until() blocks intransitively on *this; concurrently waiting threads
% * [thread.condition.nonmember] waiting threads ... holding the lock; thread
%   waiting on cond
% * [thread.sema.cnt] thread will block; acquire() blocks intransitively on
%   *this; threads waiting
% * [thread.latch.general] threads to block; threads can block
% * [thread.barrier.general] same ^
% * [except.terminate] must mention exception leaving a fiber function, or
%   destroying or assigning to a non-empty fiber_context
%*****************************************************************************

\zs{Modify \stdsection{6.9.2.1}{intro.multithread.general} paragraph 1 as indicated:}

A \emph{thread of execution} (also known as a \emph{thread}) is
\replace{a single flow of control}{the primary execution agent}\\
\add{\xref{thread.req.lockable.general}}
within a program\delete{, including the initial invocation of a specific top-level
function, and}\\
\replace{recursively including every function invocation subsequently
executed by the thread}{. When the host environment first}\\
\add{enters a program, it provides a default thread to perform the program's
execution steps}.

\add{When a thread is created, it runs a default fiber ([intro.fibers]).}

\zs{Insert before \stdsection{6.9.3}{basic.start} and renumber existing 6.9.3 to 6.9.4:}

\setcounter{section}{6}
\setcounter{subsection}{9}
\setcounter{subsubsection}{2}
\setcounter{secnumdepth}{4}
\cbstart

\rSec3[intro.fibers]{Fibers and Threads}

1 A \emph{fiber} is a single flow of control within a program, including the
initial invocation of a specific top-level function, and recursively including
every function invocation subsequently executed by the fiber. The execution
steps of a fiber are performed by a thread.

\tsnote{``Flow of control'' here refers to state necessary to program
execution, for example the contents of a processor's registers including its
instruction pointer, and the invocation sequence \xref{stacktrace.general} of
functions that have been entered but have not yet returned.}

2 An \emph{implicit fiber} is the default fiber on any thread. All other
fibers are \emph{explicit fibers.}
\tsnote{A new fiber can be created using \fiber.}

3 A thread is always running exactly one fiber. Member functions of \fiber
([fiber.context.class]) can direct the calling thread to \emph{suspend} the
current fiber and \emph{resume} a designated other fiber.

4 When a thread first enters a fiber, that thread becomes the
fiber's \emph{owning thread.} The owning thread never changes.
\tsnote{A thread is the owning thread of its default fiber.}
\tsnote{If a thread resumes a fiber owned by another thread, the behaviour is
undefined.}
\cbend

\zs{Modify \stdsection{14.2}{except.throw} paragraph 2 as indicated:}

When an exception is thrown, control is transferred to the nearest handler
with a matching type \xref{except.handle}; ``nearest'' means the handler for
which the \nt{stmt.block}{compound-statement} or
\nt{class.base.init}{ctor-initializer} following the \cpp{try} keyword was
most recently entered by the \replace{thread of control}{current fiber} and
not yet exited.

\zs{Modify \stdsection{14.2}{except.throw} paragraph 4 Note 3 as indicated:}

\tsnoten{3}{A thrown exception does not propagate to other
\replace{threads}{fibers} unless caught, stored, and rethrown using
appropriate library functions; see \stdclause{propagation} and \stdclause{futures}.}

\zs{Modify \stdsection{14.4}{except.handle} paragraph 6 as indicated:}

If no match is found among the handlers for a try block, the search for a
matching handler continues in a dynamically surrounding try block of the same
\replace{thread}{fiber}.

\zs{Append to \stdsection{14.4}{except.handle} paragraph 8 as indicated:}

\cbstart
It is implementation-defined whether the currently handled exception
designates the exception with the most recently\\
activated handler that is still active within:
\begin{itemize}
    \item the current fiber ([intro.fibers]), or
    \item the current thread.
\end{itemize}
\cbend

\zs{Modify \stdsection{14.6.3}{except.uncaught} paragraph 1 as indicated:}

The function \stdterm{\cpp{std::uncaught\_exceptions}}{uncaught.exceptions}
returns the number of uncaught exceptions \delete{in the current thread}.

\cbstart
It is implementation-defined whether \cpp{uncaught\_exceptions()} returns
the number of uncaught exceptions in:
\begin{itemize}
    \item the current fiber ([intro.fibers]), or
    \item the current thread.
\end{itemize}
\cbend

\zs{Insert new final subclause in clause 33 \stdclause{thread} as indicated:}

\setcounter{section}{33}
\setcounter{subsection}{11}
\setcounter{secnumdepth}{4}

\cbstart

\rSec2[fiber.context]{fiber\_context}

\rSec3[fiber.context.overview]{Overview}

1 A \fiber object is either \emph{empty} or \emph{non-empty}. A
default-constructed or moved-from \fiber is empty. A \fiber
representing a suspended fiber is non-empty.

2 An explicit fiber is prepared by passing an \emph{\entryfn} to \fiber's
constructor. The fiber conceptually comes into existence at the first call to
one of the \anyresume member functions, which is when that \entryfn is
entered.

3 When a fiber is first entered, a synthesized non-empty \fiber object
representing the newly-suspended previous fiber is passed as a parameter to
its \entryfn. Once entered, a fiber can suspend by calling one of the \anyresume
functions on any accessible non-empty \fiber object. When the
suspended fiber is resumed, a synthesized \fiber object
representing the newly-suspended previous fiber is returned.
\tsnote{The synthesized \fiber object received in either of those ways might
represent either an explicit fiber or an implicit fiber.}

4 An explicit fiber terminates by returning a non-empty \fiber object from
its \entryfn. \Thefiber{that \fiber object} is resumed.

%% \rSec3[fiber.context.toplevel]{Implicit Top-Level Function}

%% On every explicit fiber, the behaviour is equivalent to calling the \entryfn
%% passed to \fiber's constructor from an implicit top-level function.
%% If the fiber is later
%% unwound, this conceptual top-level stack frame serves as delimiter: this point
%% is where unwinding stops.

5 When a running fiber returns a \fiber from its \entryfn, thus resuming the
designated fiber, the synthesized \fiber passed into the resumed fiber is
empty.

6 If the fiber's \entryfn returns an empty \fiber object, \cpp{std::terminate} is called.
If the fiber's \entryfn exits via an exception, \cpp{std::terminate} is called.

7 Regardless of the number of \fiber objects in the program, exactly one of them
represents each suspended fiber. No \fiber object represents a running fiber.

8 A \fiber object can optionally be constructed by passing an explicit
\cpp{span<byte>} in which to track the fiber's
\stdterm{invocation sequence}{stacktrace.general}. If at any time during the
life of a fiber the data storage required to track its invocation sequence
exceeds the \cpp{size()} of that \cpp{span<byte>}, the behaviour is undefined.

%% Returning a \fiber object from the explicit fiber's \entryfn is equivalent
%% to returning control to the implicit top-level function.
%% Similarly,
%% when \unwindfib unwinds a fiber stack, it conceptually returns the \fiber
%% object it was passed to the implicit top-level function. Either way, the
%% The
%% conceptual implicit top-level function is responsible for deallocating the
%% explicit fiber's stack memory on return from the \entryfn.
%% 
%% Similarly, on every implicit fiber, the behaviour is equivalent to passing control through an
%% implicit top-level function above \justmain and above the \entryfn for
%% each \thread.
%% The conceptual stack frame for this implicit top-level function delimits
%% stack unwinding for each of these stacks. If the fiber stack is unwound,
%% control is conceptually returned to this implicit top-level function.
%% The conceptual top-level
%% function for an implicit fiber does not deallocate the fiber's stack memory,
%% since the host environment will do that.

%% \begin{itemize}
%%     \item
%%     \item If an empty \fiber object is returned to the conceptual top-level
%%     function for an explicit fiber, the calling thread is terminated.
%%     \item If an empty \fiber object is returned to the conceptual top-level
%%     function for the default fiber of an explicit thread, that thread is
%%     terminated.
%%     \item If an empty \fiber object is returned to the conceptual top-level
%%     function above \justmain, the process is terminated.
%% \end{itemize}

%--------------------------------- synopsis ----------------------------------
\rSec3[fiber.context.syn]{Header <fiber\_context> synopsis}

\cppf{synopsis}

%--------------------------------- class def ---------------------------------
\rSec3[fiber.context.class]{Class fiber\_context}

\cppf{fiber}

\newcommand{\state}{\cpp{state}}

\rSec4[fiber.context.cons]{Constructors, move and assignment}

%---------------------------- implicit stack ctor ----------------------------
\mbrhdr{template<class F> explicit fiber\_context(F\&\& entry)}

1 \constraints
\begin{description}
    \item[---] \cpp{remove\_cvref\_t<F>} is not the same type as \fiber.
\end{description}

2 \mandates
\begin{description}
    \item[---] \cpp{is\_constructible\_v<decay\_t<F>, F>} is \true.
    \item[---] \cpp{is\_invocable\_r\_v<fiber\_context, decay\_t<F>,
               fiber\_context&&>} is \true.
\end{description}

3 \effects
\begin{description}
    \item[---] Let \cpp{entry\_copy} be an object of
               type \cpp{decay\_t<F>} direct-non-list-initialized
               with \cpp{std::forward<F>(entry)}. 
               \tsnote{\cpp{entry\_copy} is not a member of \fiber because it
               is destroyed on fiber termination, not when a \fiber object is
               destroyed.}
    \item[---] Initializes \cpp{state} to prepare a fiber that will, when
               first resumed, enter \cpp{entry\_copy}.
    \item[---] Any necessary resources are created. \tsnote{This includes the
               new fiber's function call stack.}
\end{description}

4 \postcond
\emptyfn is \false.

5 \except
\begin{description}
    \item[---] \cpp{bad\_alloc} or \cpp{system\_error} if unable to prepare the new fiber.
    \item[---] Any exception from initialization of \cpp{entry\_copy}.
\end{description}

6 \errors
\cpp{resource\_unavailable\_try\_again} -- the system lacked the necessary resources to prepare another fiber.

%---------------------------- explicit stack ctor ----------------------------
\mbrhdr{template<class F, class D> fiber\_context(F\&\& entry, span<byte> stack, D\&\& deleter)}

1 \mandates
\begin{description}
    \item[---] \cpp{is\_constructible\_v<decay\_t<F>, F>} is \true.
    \item[---] \cpp{is\_constructible\_v<decay\_t<D>, D>} is \true.
    \item[---] \cpp{is\_invocable\_r\_v<fiber\_context, decay\_t<F>,
               fiber\_context&&>} is \true.
    \item[---] \cpp{is\_invocable\_v<decay\_t<D>, span<byte>>} is \true.
\end{description}

2 \precond
\begin{description}
    \item[---] \cpp{decay\_t<D>} meets the \emph{Cpp17MoveConstructible} requirements.
    \item[---] \cpp{invoke(deleter, stack)} does not throw an exception.
\end{description}

3 \effects
\begin{description}
    \item[---] Let \cpp{entry\_copy} be an object of
               type \cpp{decay\_t<F>} direct-non-list-initialized
               with \cpp{std::forward<F>(entry)}. 
    \item[---] Let \cpp{stack\_copy} be a copy of \cpp{stack}.
    \item[---] Let \cpp{deleter\_copy} be an object of
               type \cpp{decay\_t<D>} direct-non-list-initialized
               with \cpp{std::forward<F>(deleter)}. 
               \tsnote{\cpp{entry\_copy}, \cpp{stack\_copy} and
                \cpp{deleter\_copy} are not members of \fiber because they 
               are destroyed on fiber termination, not when a \fiber object is
               destroyed.}
    \item[---] Initializes \cpp{state} to prepare a fiber that will, when
               first resumed, enter \cpp{entry\_copy}.
    \item[---] Any necessary resources are created.
               \tsnote{It might be advantageous to obtain from the host
               environment a memory block with a read-only guard page to trap
               stack overflow.}
\end{description}

4 \postcond
\emptyfn is \false.

5 \except
\begin{description}
    \item[---] \cpp{invalid\_argument} if \cpp{stack.data()} fails to meet
               implementation-defined alignment requirements.
    \item[---] \cpp{length\_error} if \cpp{stack.size()} is less than the
               implementation-defined minimum length.
    \item[---] \cpp{system\_error} if unable to prepare the new fiber.
    \item[---] Any exception from initialization of \cpp{entry\_copy}.
    \item[---] Any exception from initialization of \cpp{deleter\_copy}.
\end{description}

6 \errors
\cpp{resource\_unavailable\_try\_again} -- the system lacked the necessary resources to prepare another fiber.

%--------------------------------- move ctor ---------------------------------
\mbrhdr{fiber\_context(fiber\_context\&\& other) noexcept}

1 \effects
Initializes \cpp{state} with \cpp{exchange(other.state, nullptr)}.

%----------------------------------- dtor ------------------------------------
\mbrhdr{\cpp{\~fiber\_context()}}

1 \effects
If \emptyfn is \false, \cpp{terminate} is invoked \xref{except.terminate}.

%------------------------------ move assignment ------------------------------
\mbrhdr{fiber\_context\& operator=(fiber\_context\&\& other) noexcept}

1 \effects
\begin{description}
    \item[---] If \emptyfn is \false, \cpp{terminate} is invoked \xref{except.terminate}.
    \item[---] Equivalent to: \cpp{this->state = exchange(other.state, nullptr)}.
\end{description}

2 \returns
\this

\rSec4[fiber.context.mem]{Members}
%-------------------------------- resume_with --------------------------------
\mbrhdr{template<class Fn> fiber\_context resume\_with(Fn\&\& fn) \&\&}

The operation of \resumewith involves at least two and possibly three fibers:

\begin{itemize}
    \item The ``calling fiber'' is the fiber calling \resumewith.
    \item The ``target fiber'' is \thefiber{\state}.
    \item \resumewith suspends the calling fiber and synthesizes
          a \fiber object representing the suspended calling fiber. Let
          \cpp{caller} be the synthesized \fiber object representing the
          calling fiber.
    \item \resumewith does not return until a running fiber, the
          ``previous fiber,'' resumes \cpp{caller}.
          \tsnote{The previous fiber can be other than the target fiber.}
          Let \cpp{previous} be the synthesized \fiber object representing the
          previous fiber.
\end{itemize}

Given a non-empty \fiber object \cpp{after}, the ``destination fiber''
(\thefiber{\cpp{after}}) can be in either of two states, as follows.

\begin{description}
    \item[---]
          Let \emph{internal-resume(\cpp{after}, \cpp{before})},
          where \cpp{after} is a non-empty \fiber object and \cpp{before} is
          another \fiber object, denote the following sequence of steps if the
          destination fiber has already been entered. That implies that at
          some earlier time, the destination fiber suspended itself by
          calling \resumewith[other], where \cpp{other} was some
          non-empty \fiber object.
        \begin{itemize}
            \item Return \cpp{before} from \resumewith[other].
        \end{itemize}   
    \item[---] Otherwise (the destination fiber's \entryfn has not
          yet been entered), let \emph{internal-resume(\cpp{after}, \cpp{before})}
          denote the following sequence of steps:
        \begin{itemize}
            \item Let \cpp{after\_entry\_copy} be the \cpp{entry\_copy} associated with \cpp{after}.
            \item Let \cpp{after\_stack\_copy} be the \cpp{stack\_copy}, if any, associated with \cpp{after}.
            \item Let \cpp{after\_deleter\_copy} be the \cpp{deleter\_copy}, if any, associated with \cpp{after}.
            \item Let \cpp{successor} be the result of executing
                  \cpp{invoke\_r<fiber\_context>(after\_entry\_copy, std::move(before))}.
            \item On return from \cpp{after\_entry\_copy}:
            \begin{itemize}
                \item destroy \cpp{after\_entry\_copy}
                \item if \cpp{after\_stack\_copy} and \cpp{after\_deleter\_copy} exist:
                    \begin{itemize}
                        \item execute \cpp{invoke(after\_deleter\_copy, after\_stack\_copy)}
                        \item destroy \cpp{after\_deleter\_copy}
                    \end{itemize}
                \item reclaim implementation-provided resources
                \item exit \thefiber{\cpp{after}}
                \item direct the current thread to resume \thefiber{\cpp{successor}}
                \item execute internal-resume(\cpp{successor}, \cpp{fiber\_context()}).
            \end{itemize}
        \end{itemize}
\end{description}

1 \mandates
\cpp{is\_invocable\_r\_v<fiber\_context, decay\_t<Fn>, fiber\_context&&>} is \true.

2 \precond
\canresume is \true.

3 \effects
\begin{description}
    \item[---] Resets \state so that \emptyfn is \true.
    \item[---] Directs the current thread to suspend the calling fiber and resume
               the target fiber.
    \item[---] Evaluates \cpp{invoke\_r(std::forward<Fn>(fn), std::move(caller))}.
               Let \cpp{returned} be the \fiber object returned by \cpp{fn}.
               \tsnote{\cpp{returned} can be other than \cpp{caller}.
               \cpp{returned} can be empty.}
    \item[---] Executes internal-resume(\cpp{target}, \cpp{returned}).
\end{description}

4 \returns

\begin{description}
    \item[---] If the previous fiber resumed the calling fiber by returning \cpp{caller},
          an empty \fiber.
    \item[---] If the previous fiber resumed the calling fiber by
          calling \cpp{caller.resume\_with(somefn)}, the \fiber object returned
          by \cpp{invoke\_r<fiber\_context>(somefn, std::move(previous))}.
\end{description}

5 \except

If the previous fiber resumed the calling fiber by calling \cpp{caller.resume\_with(somefn)}:
\begin{itemize}
    \item Any exception thrown by \cpp{invoke\_r<fiber\_context>(somefn,
          std::move(previous))}.
\end{itemize}

\tsnote{\resumewith throws nothing before suspending the calling fiber and
ensuring \emptyfn is \true.} 

6 \postcond
\emptyfn is \true.

\tsnote{Because \anyresume empties the object on which it is called, these
member functions are rvalue-reference qualified.}

%---------------------------------- resume -----------------------------------
\mbrhdr{fiber\_context resume() \&\&}

1 \effects
Equivalent to:\\
\cpp{return resume\_with(identity());}

%-------------------------------- can_resume ---------------------------------
\mbrhdr{bool can\_resume() noexcept}

1 \returns
\begin{description}
    \item[---] \false if \emptyfn is \true
    \item[---] \true if \thisfiber has no owning thread
    \item[---] \true if \currthread is \ownthread
    \item[---] \false otherwise.
\end{description}

\red{\EnterBlock{Editorial note} \canresume is intentionally not
marked \cpp{const}. \ExitBlock{editorial note}}

%----------------------------------- empty -----------------------------------
\mbrhdr{bool empty() const noexcept}

1 \effects
Equivalent to: \cpp{return (\! state);}

%------------------------------- operator bool -------------------------------
\mbrhdr{explicit operator bool() const noexcept}

1 \effects
Equivalent to: \cpp{return (\! empty());}

%---------------------- current_exception_within_fiber -----------------------

\mbrhdr{static bool current\_exception\_within\_fiber() noexcept}

1 \returns
\true if the implementation of \exfns reports the current exception(s) only within
the current fiber, \false if they consider all exceptions within the owning thread.

%----------------------------------- swap ------------------------------------
\mbrhdr{void swap(fiber\_context\& other) noexcept}

1 \effects
Equivalent to: \cpp{swap(this->state, other.state)}.

\rSec4[fiber.context.special]{Specialized algorithms}
\mbrhdr{friend void swap(fiber\_context\& lhs, fiber\_context\& rhs) noexcept}

1 \effects
Equivalent to: \cpp{lhs.swap(rhs)}.


%% \rSec3[fiber.context.unwinding]{Function unwind\_fiber()}
%% 
%% \mbrhdr{[[ noreturn ]] void unwind\_fiber(fiber\_context\&\& other)}
%% 
%% 1 \effects
%% terminate the current running fiber.
%% 
%% 2 \remarks
%% \begin{description}
%%     \item[---] The underlying Unwinding facility (for instance the unwind facility
%%                described in \emph{System V ABI for AMD64}) unwinds the stack
%%                to the implicit top-level stack frame and terminates the
%%                current fiber as described above.
%%     \item[---] Unwinding the fiber's stack causes its stack variables to be
%%                destroyed.
%%     \item[---] During this specific stack unwinding, 
%% %% only \catchall clauses are executed. No other
%%                no \cpp{catch} clauses are executed, not even \catchall.
%%     \item[---] Once the running fiber has been fully unwound, \cpp{other} is
%%                returned to the fiber's conceptual top-level function as
%%                described in \nameref{fiber-context.toplevel}.
%% %%  \item[---] Unwinding the fiber's stack causes relevant \catchall
%% %%             clauses to be executed.
%% %%  \item[---] During this specific stack unwinding, a \catchall
%% %%             clause that does not execute a \cpp{throw;} statement behaves
%% %%             as if it ended with a \cpp{throw;} statement.
%% %%  \item[---] During this specific stack unwinding, if a \catchall
%% %%             clause attempts to throw any C++ exception, the
%% %%             behaviour is undefined.
%% \end{description}
%% 
%% 3 \returns
%% \begin{description}
%%     \item[---] None: \unwindfib does not return
%% \end{description}
%% 
%% 4 \except
%% \begin{description}
%%     \item[---] None catchable by C++
%% \end{description}

\cbend

\zs{Modify \stdsection{19.6.1}{stacktrace.general} as indicated:}

1 Subclause \stdclause{stacktrace} describes components that C++ programs may use to
store the stacktrace of the current \delete{thread of}\\
\replace{execution}{fiber ([intro.fibers])}
and query information about the stored stacktrace at runtime.

2 The \emph{invocation sequence} of the current evaluation $x_0$
in the current \replace{thread of execution}{fiber} is a sequence
($x_0$,...,$x_n$) of evaluations such that, for $i \geq 0$,
$x_i$ is within the function invocation $x_{i+1}$ \xref{intro.execution}.

\abschnitt{Header File}

\zs{Add a new header file to Table 24 in \stdsection{16.4.2.3}{headers}:}

\add{\cpp{<fiber\_context>}}

\abschnitt{Feature-test Macro}
\zs{Add a new feature-test macro to \stdsection{17.3.2}{version.syn} as indicated:}

\add{\cpp{#define \__cpp\_lib\_fiber\_context 202XXXL // also in <fiber\_context>}}
