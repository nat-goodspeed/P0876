\abschnitt{appendix: API}\label{api}

\uabschnitt{class fiber}

\cppf{fiber}

\paragraph*{member functions}

\subparagraph*{(constructor)}
constructs new fiber\\

\begin{tabular}{ l l }
    \midrule

    \cpp{fiber() noexcept} & (1)\\

    \midrule

    \cpp{template<typename Fn>}\\
    \cpp{fiber(Fn&& fn)} & (2)\\

    \midrule

    \cpp{template<typename StackAlloc, typename Fn>}\\
    \cpp{fiber(std::allocator\_arg\_t, StackAlloc&& salloc, Fn&& fn)} & (3)\\

    \midrule

    \cpp{fiber(fiber&& other) noexcept} & (4)\\

    \midrule

    \cpp{fiber(const fiber& other)=delete} & (5)\\

    \midrule
\end{tabular}

\begin{description}
    \item[1)] this constructor instantiates an invalid \fiber. Its \opbool
              returns \cpp{false}; its \cpp{operator\!()} returns \cpp{true}.
    \item[2)] takes a callable (function, lambda, object with \op) as
              argument. The callable must have signature as described
              in \nameref{solution_gpub}. The stack is constructed using
              either \cpp{fixedsize} or \cpp{segmented} (see \nameref{subsec:stackalloc}).
              An implementation may infer which of these best suits the code
              in \cpp{fn}. If it cannot infer, \cpp{fixedsize} will be used.
    \item[3)] takes a callable as argument, requirements as for (2). The stack
              is constructed using \emph{salloc}
              (see \nameref{subsec:stackalloc}).\footnote{This constructor,
              along with the \nameref{subsec:stackalloc} section, is an
              optional part of the proposal. It might be that implementations
              can reliably infer the optimal stack representation.}
    \item[4)] moves underlying state to new \fiber
    \item[5)] copy constructor deleted
\end{description}

{\bfseries Notes}
\begin{description}
    \item The entry-function \cpp{fn} passed to \fiber is passed a synthesized \fiber
          instance representing the suspended caller of \resume.
    \item The function \cpp{fn} passed to \resumewith is passed a
          synthesized \fiber instance representing the suspended caller of \resumewith.
\end{description}

\subparagraph*{(destructor)}\label{subpara:destructor}
destroys a fiber\\

\begin{tabular}{ l l }
    \midrule

    \dtor & (1)\\

    \midrule
\end{tabular}

\begin{description}
    \item[1)] destroys a \fiber instance. If this instance represents a fiber
              of execution (\opbool returns \cpp{true}), then the fiber of
              execution is destroyed too. Specifically, the stack is unwound
              by throwing \unwindex.\footnote{ In a program in which exceptions
              are thrown, it is prudent to code a fiber's \entryfn with a
              last-ditch \cpp{catch (...)} clause: in general, exceptions must
              \emph{not} leak out of the \entryfn. However, since stack
              unwinding is implemented by throwing an exception, a correct
              \entryfn\ \cpp{try} statement must also
              \cpp{catch (std::unwind\_exception const&)} and rethrow it.}
\end{description}


\subparagraph*{operator=}
moves the fiber object\\

\begin{tabular}{ l l }
    \midrule

    \cpp{fiber& operator=(fiber&& other) noexcept} & (1)\\

    \midrule

    \cpp{fiber& operator=(const fiber& other)=delete} & (2)\\

    \midrule
\end{tabular}

\begin{description}
    \item[1)] assigns the state of \cpp{other} to \cpp{*this} using move semantics
    \item[2)] copy assignment operator deleted
\end{description}

{\bfseries Parameters}
\begin{description}
    \item[other]   another fiber to assign to this object\\
\end{description}

{\bfseries Return value}
\begin{description}
    \item[*this]
\end{description}


\subparagraph*{resume()}
resumes a fiber\\

\begin{tabular}{ l l }
    \midrule

    \cpp{fiber resume()} & (1)\\

    \midrule

    \cpp{template<typename Fn>}\\
    \cpp{fiber resume\_with(Fn&& fn)} & (2)\\

    \midrule
\end{tabular}

\begin{description}
    \item[1)] suspends the active fiber, resumes fiber \cpp{*this}
    \item[2)] suspends the active fiber, resumes fiber \cpp{*this}
              but calls \cpp{fn()} in the resumed fiber (as if called by the
              suspended function)
\end{description}

{\bfseries Parameters}
\begin{description}
    \item[fn] function injected into resumed fiber\\
\end{description}

{\bfseries Return value}
\begin{description}
    \item[fiber] the returned instance represents the fiber that has been
                 suspended in order to resume the current fiber
\end{description}

{\bfseries Exceptions}
\begin{description}
    \item[1)] \resume or \resumewith might throw \unwindex if, while suspended,
              the calling fiber is destroyed
    \item[2)] \resume or \resumewith might throw \emph{any} exception if,
              while suspended:
              \begin{itemize}
                  \item some other fiber calls \resumewith to resume this
                        suspended fiber
                  \item the function \cpp{fn} passed to \resumewith -- or some
                        function called by \cpp{fn} -- throws an exception
              \end{itemize}
    \item[3)] Any exception thrown by the function \cpp{fn} passed to
              \resumewith, or any function called by \cpp{fn}, is thrown in the
              fiber referenced by \cpp{*this} rather than in the fiber of
              the caller of \resumewith.
\end{description}

{\bfseries Preconditions}
\begin{description}
    \item[1)] \cpp{*this} represents a valid fiber (\opbool returns \cpp{true})
    \item[2)] the current \cpp{std::thread} is the same as the thread on which
              \cpp{*this} was originally launched
\end{description}

{\bfseries Postcondition}
\begin{description}
    \item[1)] \cpp{*this} is invalidated (\opbool returns \cpp{false})
\end{description}

{\bfseries Notes}
\newline
\resume preserves the execution context of the calling fiber. Those data are
restored if the calling fiber is resumed.\\
A suspended \cpp{fiber} can be destroyed. Its resources will be cleaned
up at that time.\\
The returned \cpp{fiber} indicates whether the suspended fiber has terminated
(returned from \entryfn) via \opbool.\\
Because \resume invalidates the instance on which it is called, \emph{no valid
\fiber instance ever represents the currently-running fiber.}\\
When calling \resume, it is conventional to replace the newly-invalidated
instance -- the instance on which \resume was called -- with the new instance
returned by that \resume call. This helps to avoid inadvertent calls to \resume
on the old, invalidated instance.
\newline
An injected function \cpp{fn()} must accept \cpp{std::fiber&&} and return
\fiber. The returned \fiber instance is, in turn, used as the return value for
the suspended function: \resume or \resumewith.


\subparagraph*{operator bool}
test whether fiber is valid\\

\begin{tabular}{ l l }
    \midrule

    \cpp{explicit operator bool() const noexcept} & (1)\\

    \midrule
\end{tabular}

\begin{description}
    \item[1)] returns \cpp{true} if \cpp{*this} represents a fiber of
              execution, \cpp{false} otherwise.
\end{description}

{\bfseries Notes}
\newline
A \fiber instance might not represent a valid fiber for any of a number of reasons.
\begin{itemize}
    \item It might have been default-constructed.
    \item It might have been assigned to another instance, or passed into a
          function.\\
          \fiber instances are move-only.
    \item It might already have been resumed -- calling \resume invalidates the
          instance.
    \item The \entryfn might have voluntarily terminated the fiber by
          returning.
\end{itemize}
The essential points:
\begin{itemize}
    \item Regardless of the number of \fiber declarations, exactly one\\
          \fiber instance represents each suspended fiber.
    \item No \fiber instance represents the currently-running fiber.
\end{itemize}


\subparagraph*{operator!}
test whether fiber is invalid\\

\begin{tabular}{ l l }
    \midrule

    \cpp{bool operator\!() const noexcept} & (1)\\

    \midrule
\end{tabular}

\begin{description}
    \item[1)] returns \cpp{false} if \cpp{*this} represents a valid fiber,
              \cpp{true} otherwise.
\end{description}

{\bfseries Notes}
\newline
See {\bfseries Notes} for \opbool.

\subparagraph*{(comparisons)}
establish an arbitrary total ordering for \fiber instances\\

\begin{tabular}{ l l }
    \midrule

    \cpp{bool operator==(const fiber& other) const noexcept} & (1)\\

    \midrule

    \cpp{bool operator\!=(const fiber& other) const noexcept} & (1)\\

    \midrule

    \cpp{bool operator<(const fiber& other) const noexcept} & (2)\\

    \midrule

    \cpp{bool operator>(const fiber& other) const noexcept} & (2)\\

    \midrule

    \cpp{bool operator<=(const fiber& other) const noexcept} & (2)\\

    \midrule

    \cpp{bool operator>=(const fiber& other) const noexcept} & (2)\\

    \midrule
\end{tabular}

\begin{description}
    \item[1)] Every invalid \fiber instance compares equal to every other invalid
              instance. But because the running fiber is never represented by
              a valid \fiber instance, and because every suspended fiber is
              represented by exactly one valid instance, \emph{no valid instance
              can ever compare equal to any other valid instance.}
    \item[2)] These comparisons establish an arbitrary total ordering of \fiber
              instances, for example to store in ordered containers. (However,
              key lookup is meaningless, since you cannot construct a search key
              that would compare equal to any entry.) There is no significance
              to the relative order of two instances.
\end{description}


\subparagraph*{swap}
swaps two \fiber instances\\

\begin{tabular}{ l l }
    \midrule

    \cpp{void swap(fiber& other) noexcept} & (1)\\

    \midrule
\end{tabular}

\begin{description}
    \item[1)] Exchanges the state of \cpp{*this} with \cpp{other}.\\
\end{description}


\uabschnitt{std::unwind\_fiber()}

terminate the current running fiber, switching to the fiber represented by
the passed \fiber. This is like returning that \fiber from the \entryfn, but may
be called from any function on that fiber.

\begin{tabular}{ l l }
    \midrule

    \cpp{void unwind\_fiber(fiber&& other)} & (1)\\

    \midrule
\end{tabular}

\begin{description}
    \item[1)] throws \unwindex, binding the passed \fiber. The running fiber's
              first stack entry catches \unwindex, extracts the bound \fiber
              and terminates the current fiber by returning that \fiber.
\end{description}

\bfs{Parameters}
\begin{description}
    \item[other] the \fiber to which to switch once the current fiber has terminated
\end{description}

\bfs{Preconditions}
\begin{description}
    \item[1)] \cpp{cont} must be valid (\cpp{operator bool()} returns \cpp{true})
\end{description}

\bfs{Return value}
\begin{description}
    \item[1)] None: \unwindcont does not return
\end{description}

\bfs{Exceptions}
\begin{description}
    \item[1)] throws \unwindex
\end{description}


\uabschnitt{std::unwind\_exception}

is the exception used to unwind the stack referenced by a \fiber being destroyed.
t is thrown by \unwindcont. \unwindex binds a \fiber referencing the fiber to
which control should be passed once the current fiber is unwound and destroyed.


\uabschnitt{Stack allocators}
\label{subsec:stackalloc}

are the means by which stacks with non-default properties may be requested by
the caller of \fiber constructor. The stack allocator concept is
implementation-dependent; the means by which an implementation's
stack allocators communicate with \fiber constructor is unspecified.\\

An implementation may provide zero or more stack allocators. However, a stack
allocator with semantics matching any of the following must use the
corresponding name.
\begin{description}
  \item[protected\_fixedsize] The constructor accepts a \cpp{size\_t} parameter.
        This stack allocator constructs a contiguous stack of specified size,
        appending a guard page at the end to protect against overflow. If the
        guard page is accessed (read or write operation), a segmentation
        fault/access violation is generated by the operating system.
  \item[fixedsize] The constructor accepts a \cpp{size\_t} parameter.
        This stack allocator constructs a contiguous stack of specified size.
        In contrast to \cpp{protected\_fixedsize}, it does not append a guard
        page. The memory is simply managed by \cpp{std::malloc()}
        and \cpp{std::free()}, avoiding kernel involvement.
  \item[segmented] The constructor accepts a \cpp{size\_t} parameter.
        This stack allocator creates a segmented stack\cite{gccsplit} with the
        specified initial size, which grows on demand.
\end{description}
