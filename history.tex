\abschnitt{Revision History}\label{history}
This document supersedes P0876R5.\\
\newline
Changes since P0876R5:

\begin{itemize}
    \item \cpp{std::unwind\_exception} removed: stack unwinding must be
      performed by platform facilities.
    \item \cpp{fiber\_context::can\_resume\_from\_any\_thread()} renamed to
      \canxtresume.
    \item \cpp{fiber\_context::valid()} renamed to \cpp{empty()} with inverted
      sense.
    \item Material has been added concerning the top-level wrapper
      logic governing each fiber.
\end{itemize}

The change to unwinding fiber stacks using an anonymous \foreignex not
catchable by C++ \cpp{try} / \cpp{catch} blocks is in response to deep
discussions in Kona 2019 of the surprisingly numerous problems surfaced by
using an ordinary C++ exception for that purpose.

Further information about the specific mechanism can be found in
\nameref{unwinding} et ff.

Problems resolved by discarding \unwindex in favor of a non-C++ \foreignex:
\begin{itemize}
    \item When unwinding a fiber stack, it is essential to know the subsequent
          fiber to resume. \unwindex therefore bound a \fiber. \fiber is
          move-only. But exceptions must be copyable.
    \item It was possible to catch and discard \unwindex, with problematic
          consequences for its bound \fiber. The new mechanism does not permit
          that.
    \item Similarly, it used to be possible to catch \unwindex but not rethrow it.
    \item If we attempted to address the problem above by introducing a
          \unwindex operation to extract the bound \fiber, it became possible
          to rethrow the exception with an empty (moved-from) \fiber instance.
    \item Throwing a C++ exception during C++ exception unwinding terminates
          the program. But destroying a \fiber no longer causes a C++
          exception to be thrown.
    \item It is no longer possible to capture \unwindex with
          \cpp{std::exception\_ptr} and migrate it to a different fiber -- or
          a different thread.
\end{itemize}

We have also addressed what happens when a \fiber instance representing \main,
or the default fiber on a \thread, is destroyed.

Not yet addressed:

\begin{itemize}
    \item During stack unwinding, is an object's destructor allowed to resume
      some other fiber?
%%  \item During stack unwinding, is a \catchall block allowed to
%%    resume some other fiber?
\end{itemize}
