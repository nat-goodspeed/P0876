\abschnitt{Revision History}\label{history}
This document supersedes P0876R12.

\uabschnitt{Changes since P0876R12}

\begin{itemize}
    \item Specified that constructors \emph{decay-copy} the \entryfn.
    \item Changed \cpp{span<byte, N>} constructor param to
          simply \cpp{span<byte>}; also accepted deleter function, which it
          must \emph{decay-copy}.
    \item Specified constructor exceptions.
    \item Specified that destroying a non-empty \fiber
          calls \cpp{terminate()}.
    \item Clarified that when \resumewith is called, \emptyfn becomes
          true immediately.
    \item Introduced exposition-only \cpp{fiber\_context::state} member to
          streamline wording.
    \item Removed \cpp{concurrency\_v2} namespace.
    \item Changed ``Equivalent to'' to ``As-if''.
    \item Clarified Preconditions vs. Mandates.
\end{itemize}

\uabschnitt{Changes since P0876R11}

\begin{itemize}
    \item Removed \getsource, \gettoken, \reqstop and
          exposition-only \cpp{ssource} members.
    \item Added a \fiber constructor accepting a caller-provided
          uninitialized memory area for the new fiber's function call stack.
\end{itemize}

Bundling a \cpp{stop\_source} into \fiber presented implementability concerns.
Although each fiber (specifically, its function call stack) is itself a
persistent entity, the \fiber representing that fiber is not: a
new \fiber object is synthesized on every suspension. This
presents a problem: how does the code that suspends a fiber find its
associated \cpp{stop\_source} shared state?

A consumer wishing to pass a \cpp{std::stop\_token} to a new fiber can itself
instantiate \cpp{std::stop\_source}, obtain from it a \cpp{stop\_token} and
bind that \cpp{stop\_token} in a lambda passed to the \fiber
constructor. Accordingly, the \fiber API need not explicitly support that.

\uabschnitt{Changes since P0876R10}

\begin{itemize}
    \item Removed \cpp{cancel()} method and the \cancelfn constructor
          argument. Replaced with the \cpp{std::jthread} stop token handling
          API: \getsource, \gettoken and \reqstop. This simplifies examples by
          eliminating \cpp{launch()} and \cpp{assert\_on\_cancel}.
    \item Added a section exploring the relationship of \fiber to the larger
          C++ ecosystem.
    \item Reordered some sections to make the paper more accessible for new readers.
\end{itemize}

\uabschnitt{Changes since P0876R9}

\begin{itemize}
    \item Removed \xtresume, \xtresumewith, \xtcancel and \canxtresume, along
          with stated support for resuming a suspended fiber on some thread
          other than the one on which it was launched.
\end{itemize}

In Belfast, EWG came down strongly against cross-thread fiber resumption. The
most emphatic objection was that for a function referencing TLS, multiple
compilers cache TLS pointers on the function's stack frame. Resuming a fiber
containing that stack frame on some other thread would cause problems. In the
best case, the resumed function would merely reference TLS belonging to the
wrong thread -- but at some point the original thread will terminate, its TLS
will be destroyed, and the cached pointers will be left dangling.

With \fiber, any opaque function call might possibly suspend -- but
invalidating cached TLS pointers across every opaque function call is deemed
unacceptable overhead.

\uabschnitt{Changes since P0876R8}

\begin{itemize}
    \item Reinstated cancellation function constructor argument.
    \item Added \cpp{cancel()} and \cpp{cancel\_from\_any\_thread()} member
          functions.
    \item Re-removed \unwindfib.
\end{itemize}

SG1 directed P0876R9 to conform to the Cologne 2019 recommendations, with any
other changes proposed in a separate paper.

\uabschnitt{Changes since D0876R7}

\begin{itemize}
    \item Cancellation function removed from \fiber constructor.
    \item \unwindfib re-added, with implementation-defined behaviour.
    \item Added elaboration of \cpp{filament} example to bind cancellation
          function.
\end{itemize}

P0876R8 diverged from the recommendations of the second SG1 round in Cologne
2019. It did not introduce \cpp{cancel()} or \cpp{cancel\_from\_any\_thread()}
member functions. In fact it removed the \cancelfn constructor argument.

\fiber is intended as the lowest-level stackful context-switching API. Binding
a \cancelfn on the fiber stack is a flourish rather than a necessity. It adds
overhead in both space (on the fiber stack) and time (to traverse the stack to
retrieve the \cancelfn). For this API, it should suffice to pass the desired
\cancelfn to \anyresumewith. If it is important to associate a \cancelfn with
a particular fiber earlier in the lifespan of the fiber, a struct serves.

A more compelling reason to avoid constructing an explicit fiber with
a \cancelfn is that no implicit fiber has any such \cancelfn\xspace -- and the
consuming application cannot tell, a priori, whether a given \fiber instance
represents an explicit or an implicit fiber. If \this represents an
implicit fiber, what should the proposed \cpp{cancel()} member function do?

Passing a specific \cancelfn to \anyresumewith avoids that problem.

P0876R8 follows SG1 recommendation in making it Undefined Behaviour to destroy
(or assign to) a non-empty \fiber instance.

\unwindfib was reintroduced with implementation-defined behaviour to allow fiber
cleanup leveraging implementation internals. Its use was entirely optional (and
auditable). 

\uabschnitt{Changes since P0876R6}

\begin{itemize}
    \item Implicit stack unwinding (by non-C++ exception) removed.
    \item \unwindfib removed.
    \item Cancellation function added to \fiber constructor.
\end{itemize}

In Cologne 2019, SG1 took the position that:

\begin{itemize}
    \item The \fiber facility is not the only C++ feature that
          requires ``special'' unwinding (special function exit path).
    \item Such functionality should be decoupled from \fiber. It requires its
          own proposal that follows its own course through WG21 process.
    \item Depending on this (yet to be written) proposal would unduly delay
          the \fiber facility.
    \item For now, the \fiber facility should adopt a ``less is
          more'' approach, removing promises about implicit unwinding, placing
          the burden on the consumer of the facility instead.
    \item This leaves the way open for \fiber to integrate with
          a new, improved unwind facility when such becomes available.
\end{itemize}

The idea of making \fiber's constructor accept a cancellation function was
suggested to permit consumer opt-in to P0876R5 functionality where
permissible, or convey to the fiber in question by any suitable means the need
to clean up and terminate.

Requiring the cancellation function is partly because it remains unclear what
the default should be. This could be one of the questions to be answered by a
TS. Moreover, the absence of a default permits specifying later that the
default engages the new, improved unwind facility.

\uabschnitt{Changes since P0876R5}

\begin{itemize}
    \item \cpp{std::unwind\_exception} removed.
    \item \cpp{fiber\_context::can\_resume\_from\_any\_thread()} renamed to
      \canxtresume.
    \item \cpp{fiber\_context::valid()} renamed to \emptyfn with inverted
      sense.
    \item Material has been added concerning the top-level wrapper
      logic governing each fiber.
\end{itemize}

\unwindex was removed in response to deep
discussions in Kona 2019 of the surprisingly numerous problems surfaced by
using an ordinary C++ exception for that purpose.

Problems resolved by discarding \unwindex:
\begin{itemize}
    \item When unwinding a fiber stack, it is essential to know the subsequent
          fiber to resume. \unwindex therefore bound a \fiber. \fiber is
          move-only. But C++ exceptions must be copyable.
    \item It was possible to catch and discard \unwindex, with problematic
          consequences for its bound \fiber.
    \item Similarly, it was possible to catch \unwindex but not rethrow it.
    \item If we attempted to address the problem above by introducing a
          \unwindex operation to extract the bound \fiber, it became possible
          to rethrow the exception with an empty (moved-from) \fiber instance.
    \item Throwing a C++ exception during C++ exception unwinding terminates
          the program. It was possible for an exception implementation based
          on \cpp{thread\_local} to become confused by exceptions on different
          fibers on the same thread.
    \item It was possible to capture \unwindex with \cpp{std::exception\_ptr}
          and migrate it to a different fiber -- or a different thread.
\end{itemize}
