\abschnitt{Revision History}\label{history}
This document supersedes P0876R6.

Changes since P0876R6:

\begin{itemize}
    \item Implicit stack unwinding (by non-C++ exception) removed.
    \item \unwindfib removed.
    \item Cancellation function added to \fiber constructor.
\end{itemize}

In Cologne 2019, SG1 took the position that:

\begin{itemize}
    \item The \cpp{fiber\_context} facility is not the only C++ feature that
          requires ``special'' unwinding (special function exit path).
    \item Such functionality should be decoupled from \fiber. It requires its
          own proposal that follows its own course through WG21 process.
    \item Depending on this (yet to be written) proposal would unduly delay
          the \cpp{fiber\_context} facility.
    \item For now, the \cpp{fiber\_context} facility should adopt a ``less is
          more'' approach, removing promises about implicit unwinding, placing
          the burden on the consumer of the facility instead.
    \item This leaves the way open for \cpp{fiber\_context} to integrate with
          a new, improved unwind facility when such becomes available.
\end{itemize}

The idea of making \fiber's constructor accept a cancellation function was
suggested to permit consumer opt-in to P0876R5 functionality where
permissible, or convey to the fiber in question by any suitable means the need
to clean up and terminate.

Requiring the cancellation function is partly because it remains unclear what
the default should be. This could be one of the questions to be answered by a
TS. Moreover, the absence of a default permits specifying later that the
default engages the new, improved unwind facility.

Changes since P0876R5:

\begin{itemize}
    \item \cpp{std::unwind\_exception} removed.
    \item \cpp{fiber\_context::can\_resume\_from\_any\_thread()} renamed to
      \canxtresume.
    \item \cpp{fiber\_context::valid()} renamed to \cpp{empty()} with inverted
      sense.
    \item Material has been added concerning the top-level wrapper
      logic governing each fiber.
\end{itemize}

\unwindex was removed in response to deep
discussions in Kona 2019 of the surprisingly numerous problems surfaced by
using an ordinary C++ exception for that purpose.

Problems resolved by discarding \unwindex:
\begin{itemize}
    \item When unwinding a fiber stack, it is essential to know the subsequent
          fiber to resume. \unwindex therefore bound a \fiber. \fiber is
          move-only. But C++ exceptions must be copyable.
    \item It was possible to catch and discard \unwindex, with problematic
          consequences for its bound \fiber.
    \item Similarly, it was possible to catch \unwindex but not rethrow it.
    \item If we attempted to address the problem above by introducing a
          \unwindex operation to extract the bound \fiber, it became possible
          to rethrow the exception with an empty (moved-from) \fiber instance.
    \item Throwing a C++ exception during C++ exception unwinding terminates
          the program. It was possible for an exception implementation based
          on \cpp{thread\_local} to become confused by exceptions on different
          fibers on the same thread.
    \item It was possible to capture \unwindex with \cpp{std::exception\_ptr}
          and migrate it to a different fiber -- or a different thread.
\end{itemize}
