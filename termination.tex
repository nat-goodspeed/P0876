\abschnitt{termination}\label{termination}

%% There are a few different ways to terminate a given fiber without
%% terminating the whole process, or engaging undefined behaviour.
%% 
%% When a \fiber instance is constructed with an \entryfn, its new stack is
%% initialized with the frame of an implicit top-level function that marks the
%% end of the stack. \unwindfib unwinds the stack back to
%% that top-level function, which resumes the \fiber passed to \unwindfib.
%% 
%% Therefore, any of the following will gracefully terminate a fiber:
%% 
%% \begin{itemize}
%%     \item Cause its \entryfn to return a non-empty \fiber.
%%     \item From within the fiber you wish to terminate, call \unwindfib with a
%%           non-empty \fiber. That fiber will be resumed
%%           when the active fiber terminates.
%%     \item Call \cpp{fiber\_context::resume\_with(unwind\_fiber)}. This is what \dtor
%%           does. Since\\\unwindfib accepts a \fiber, and since \resumewith
%%           synthesizes a\\\fiber representing its caller and passes it to the
%%           subject function, this terminates the fiber referenced by the
%%           original \fiber instance and then resumes the caller.
%%     \item Engage \dtor: switch to some other fiber, which will
%%           receive a \fiber instance representing the current fiber. Make that
%%           other fiber destroy the received \fiber instance.
%% \end{itemize}
%% 
%% The above are all equivalent: stack variables are properly destroyed, since
%% the stack is unwound. (See \nameref{unwinding}.)
%% 
%% In an environment that forbids exceptions, 
Every \fiber you launch must
terminate gracefully by returning from its \entryfn.
%% You may not
%% call \unwindfib. You may not call \dtor, explicitly or implicitly, on a
%% non-empty \fiber instance. With these restrictions, it is possible to use
%% the \fiber facility without exception support.

When an explicitly-launched fiber's \entryfn returns a non-empty \fiber
instance, the running fiber is terminated. Control switches to the fiber
indicated by the returned \fiber instance. The \entryfn may return (switch to)
any reachable non-empty \fiber instance -- it need not be the instance originally
passed in, or an instance returned from the \resume family of methods.

\emph{Calling} \resume means: ``Please switch to the indicated fiber; I
am suspending; please resume me later.''

\emph{Returning} a particular \fiber means: ``Please switch to the indicated
fiber; and by the way, I am done.''

Cancellation of another fiber is not explicitly supported
by \cpp{fiber\_context}. If it is important for consuming code to communicate
to a suspended fiber the desire that it should terminate, lambda binding may
be used to pass some relevant object, e.g. a \cpp{stop\_token}.

It is up to the code running on the fiber in question to observe and respond
to any such termination request. The fiber must be resumed \emph{after} the
request before it could possibly observe the change. Even then, the \entryfn
might not immediately return.

One tactic would be to request termination, then loop over \anyresume calls until
the returned \fiber is \cpp{empty()}. However, that information is ambiguous.

Suppose we have a \fiber instance \cpp{f1} representing suspended fiber F,
with an application-specific termination request mechanism. The running fiber
M requests F to terminate, then calls \cpp{f1.resume()}, which in due course
returns another \fiber instance \cpp{f2}.

\cpp{f2} has various possible values.

\begin{itemize}
    \item \cpp{f2} might be empty. This might mean that fiber F did in fact
          terminate.
    \item Alternatively, it might mean that fiber F, instead of terminating,
          resumed fiber G, which terminated by resuming fiber M.
    \item Or fiber F might have terminated by resuming fiber G, which might
          have terminated by resuming fiber M.
    \item In other words, if \cpp{f2} is empty, fiber M cannot know the
          present state of fiber F.
    \item \cpp{f2} might not be empty. That might mean that fiber F did not
          terminate before resuming fiber M. \cpp{f2} would represent fiber F.
    \item Or it might mean that fiber F terminated by resuming fiber G, which
          might have resumed fiber M. \cpp{f2} would represent fiber G.
    \item Or it might mean that fiber F, instead of terminating, resumed fiber
          G, which resumed fiber M. \cpp{f2} would (again) represent fiber G.
    \item In other words, if \cpp{f2} is not empty, fiber M cannot know the
          present state of fiber F.
\end{itemize}

The \cpp{autocancel} class introduced in \nameref{launch} illustrates a
possible cancellation implementation, subject to the limitations described
above.

%% The \emph{last} fiber on a particular thread has no non-empty \fiber to
%% return. For this reason, returning an empty \fiber instance (\opbool
%% returns \cpp{false}) terminates the calling thread. This is true whether or
%% not the thread's default fiber (see \nameref{fiber-context.implicit}) has
%% terminated.

%% \uabschnitt{stack unwinding}\label{unwinding}
%% 
%% Stack unwinding caused by an exception, thread termination or fiber
%% destruction exits functions on that stack without executing a \cpp{return} statement. Local variables
%% that go out of scope may have destructors that must be called.
%% The implementation must walk the stack and call the destructor for each object
%% in every such stack frame.
%% 
%% The C++ standard does not define how exception handling is implemented. Stack unwinding differs
%% among different systems. The process of stack unwinding is described in the
%% system ABI, for instance:
%% \begin{itemize}
%%     \item \emph{.eh\_frame}/\emph{personality routine} on SYS V AMD64 ABI\cite{SYSVAMD64} (de facto standard among Unix-like operating systems)
%%     \item \emph{RUNTIME\_FUNCTION}/\emph{UNWIND\_INFO} on x64 Windows\cite{WinX64}
%%     \item \emph{.pdata}/\emph{.xdata} on ARM64 Windows\cite{WinARM64}
%% \end{itemize}
%% 
%% \paragraph{SYS V AMD64 unwind library}
%% is based on DWARF CFI (call frame information) that are stored in the \emph{.eh\_frame} section.
%% Unwinding happens under following circumstances:
%% \begin{itemize}
%%     \item A C++ exception has been thrown
%%     \item unwinding is forced by an external agent (longjmp for instance)
%% \end{itemize}
%% \uwforced takes a \foreignex (non-C++ exception; for instance Java or GO) and walks the stack frame by frame
%% inspecting the \emph{unwind tables} for cleanup functions (for instance destructors of
%% local variables) and catch blocks.
%% 
%% \uwforced calls a \emph{personality routine} (\cpp{__gxx_personality_v0()} for GCC).\footnote{The
%% personality routine passed by a specific runtime serves as interface between system unwinding library
%% and language specific exception handling (not only C++; GO and Java are also supported). It is always invoked via pointer (saved
%% as a function pointer in \ehframe\xspace for each stack frame).}
%% \uwforced takes a stop function that controls the termination of the unwinding
%% (reaching end of stack for fibers).
%% The stop function intercepts calls to the personality routine, letting the external
%% agent override the defaults of the stack frame's personality routine.\footnote{As
%% a consequence the C++ personality routine deals only with C++ exceptions;
%% it does not need to know anything specific about unwinding done by an external
%% agent such as fiber or pthreads cancellation.}
%% When the destination frame (last frame on fiber
%% stack) is reached, control jumps back to the caller without further popping
%% the stack.
%% 
%% The code snippet below is a proof of concept available at \href{https://github.com/boostorg/context/tree/p0876r6}{Boost.Context branch p0876r6}.
%% \cppf{unwind}
%% \cpp{fiber_unwind()} is called by \unwindfib or \dtor and starts the stack unwinding.
%% The foreign exception \cpp{foreign_unwind_ex}\footnote{setting member variable makes \cpp{foreign_unwind_ex} a foreign exception}
%% is allocated and passed as parameter to the unwinding library. Function \cpp{fiber_unwind_stop()} transfers execution control
%% to the calling fiber once the last stack frame has been unwound.
%% 
%% \subparagraph{non-catchable \foreignex}
%% \unwindfib uses a non-C++ \foreignex to force stack unwinding.
%% As stated in the \emph{SYS V AMD64 ABI}\cite{SYSVAMD64} standard:
%% "A runtime is not allowed to catch an exception if the \cpp{_UA_FORCE_UNWIND} flag was passed to the personality routine."
%% and "... since it is not possible to determine if a given catch clause will re-throw or not without executing it ...", the
%% \foreignex must not be catchable by C++ \cpp{try-catch} blocks.
%% 
%% As a consequence, \curex can not return a \cpp{std::exception\_ptr} pointing
%% to a \foreignex.
%% 
%% In order to detect if stack unwinding is currently in progress \uncex returns \cpp{true} and\\
%% \uncexs counts the \foreignex.
%% 
%% The rationale for moving to an uncatchable exception is further explained in
%% the \nameref{history}.
%% 
%% The specific characteristics of a \foreignex:
%% 
%% \begin{itemize}
%%     \item Throwing the \foreignex can only be effected by the \fiber
%%     facility. The proposed \unwindfib function is the only way to cause that
%%     explicitly.
%%     \item The ultimate "catch" -- the point at which stack unwinding stops --
%%     is likewise determined by the \fiber facility. There is no explicit syntax
%%     for this.
%%     \item Along the way, as with a normal C++ exception, every object in every
%%     stack frame is destroyed.
%%  \item \catchall clauses along the way are executed, but:
%%  \begin{itemize}
%%      \item \cpp{throw;} resumes stack unwinding, as usual
%%      \item a \catchall clause that does not execute a \cpp{throw;}
%%      statement behaves as if it ends with a \cpp{throw;} statement
%%      \item a \catchall clause that attempts to throw a normal C++
%%      exception engages Undefined Behaviour
%%  \end{itemize}
%%  \item \cpp{catch (}\emph{anything else}\cpp{)}
%%     \item \cpp{catch} clauses along the way are ignored.
%% \end{itemize}
%% 
%% \zs{The system's exception handling, i.e. its unwinding framework, is used to clean up the stack
%% of a fiber by using a foreign exception that is not catchable by C++ \cpp{try-catch} blocks.}
%% 
%% Since unwinding a fiber's stack requires destroying objects declared in stack
%% frames,
%% %% and may involve executing \catchall clauses,
%% it is worth pointing out that destroying a non-empty \fiber on a thread other
%% than the thread on which it was last resumed will run those object destructors
%% %% and \catchall clauses
%% on the thread destroying the \fiber instance.
%% 
%% As a consequence, destroying a \fiber instance representing a thread's default
%% fiber (see \nameref{fiber-context.implicit})
%% from any other thread engages Undefined Behaviour.\footnote{One unobvious case
%% would be if a fiber running on non-\main thread \cpp{T} stores a \fiber
%% representing \cpp{T}'s default fiber in a static variable, whether
%% module-scope or function-scope. That variable will be destroyed at program
%% termination, probably on a thread other than \cpp{T}.}
%% 
%% \subparagraph{marker frame}
%% 
%% The \fiber facility behaves as if there is an implicit top-level function above
%% each explicit fiber's \entryfn. (See \nameref{fiber-context.implicit}) This
%% top-level function serves to delimit stack unwinding. Once the stack has been
%% unwound to that point, it is as if control returns to the implicit top-level function. The
%% implicit top-level function is conceptually responsible for freeing the explicit fiber's
%% stack memory and for resuming the \fiber designated as the next fiber.
%% 
%% \subparagraph{destroying a \fiber representing a thread's default fiber}
%% 
%% Similarly, the C++ runtime behaves as if there is a stack marker at or above \main (and
%% each explicitly-launched thread's \entryfn) that serves to delimit stack
%% unwinding due to the \foreignex. Unlike an explicit fiber's top-level
%% function, though, the conceptual top-level function on a thread's default fiber
%% does \emph{not} deallocate that fiber's stack: the OS, which provided the
%% stack in the first place, will do that.
%% 
%% Unwinding the stack belonging to a thread's default fiber leaves the stack
%% allocated but unreachable. That thread may continue to execute explict fibers
%% as long as desired.
%% 
%% Ultimately, however, it must be possible to exit a fiber in such as way as to
%% terminate the calling thread. Returning an empty \fiber instance from
%% a fiber's \entryfn terminates the running thread. Consequently, passing an
%% empty \fiber instance to \unwindfib also terminates the calling thread.
%% 
%% The \fiber facility does not defend against the case in which a thread's
%% default fiber suspends (rather than terminating), but the explicit fiber it
%% resumes ultimately causes thread termination in either of the ways described
%% above. A higher-level library built on \fiber can provide a scheduler.
%% The \fiber facility intentionally does not.
%% 
%% The conceptual top-level function above \main, given an empty \fiber instance to resume,
%% terminates the whole process instead of that one thread.
