\abschnitt{termination}

There are a few different ways to terminate a given fiber without
terminating the whole process, or engaging undefined behavior.\\

When a \fiber instance is constructed with an \entryfn, its new stack is
initialized with the frame of an implicit top-level function that marks the
end of the stack. \unwindfib unwinds the stack back to
that top-level function, which returns to the \fiber passed to \unwindfib.\\

Therefore, any of the following will gracefully terminate a fiber:

\begin{itemize}
    \item Cause its \entryfn to return a non-empty \fiber.
    \item From within the fiber you wish to terminate, call \unwindfib with a
          non-empty \fiber. That fiber will be resumed
          when the active fiber terminates.
    \item Call \cpp{fiber\_context::resume\_with(unwind\_fiber)}. This is what \dtor
          does. Since\\\unwindfib accepts a \fiber, and since \resumewith
          synthesizes a\\\fiber representing its caller and passes it to the
          subject function, this terminates the fiber referenced by the
          original \fiber instance and switches back to the caller.
    \item Engage \dtor: switch to some other fiber, which will
          receive a \fiber instance representing the current fiber. Make that
          other fiber destroy the received \fiber instance.
\end{itemize}

The above are all equivalent: stack variables are properly destroyed, since
the stack is unwound. (See \nameref{destruction}.)\\

In an environment that forbids exceptions, every \fiber you launch must
terminate gracefully, by returning from its top-level function. You may not
call \unwindfib. You may not call \dtor, explicitly or implicitly, on a
non-empty \fiber instance.\\

When an explicitly-launched fiber's \entryfn returns a non-empty \fiber
instance, the running fiber is terminated. Control switches to the fiber
indicated by the returned \fiber instance. The \entryfn may return (switch to)
any reachable non-empty \fiber instance -- it need not be the instance originally
passed in, or an instance returned from any of the \resume family of
methods.\\

Returning an empty \fiber instance (\opbool returns \cpp{false}) invokes
undefined behavior.\\

\emph{Calling} \resume means: ``Please switch to the indicated fiber; I
am suspending; please resume me later.''\\

\emph{Returning} a particular \fiber means: ``Please switch to the indicated
fiber; and by the way, I am done.''


\uabschnitt{stack unwinding}\label{unwinding}

Stack unwinding caused by an exception, thread termination or fiber
destruction exits functions on that stack without executing a \cpp{return} statement. Local variables
that go out of scope may have destructors that must be called.
The implementation must walk the stack and call the destructor for each object
in every such stack frame.\\

The C++ standard does not define how exception handling is implemented. Stack unwinding differs
among different systems. The process of stack unwinding is described in the
system ABI, for instance:
\begin{itemize}
    \item \emph{.eh\_frame}/\emph{personality routine} on SYS V AMD64 ABI\cite{SYSVAMD64} (de facto standard among Unix-like operating systems)
    \item \emph{RUNTIME\_FUNCTION}/\emph{UNWIND\_INFO} on x64 Windows\cite{WinX64}
    \item \emph{.pdata}/\emph{.xdata} on ARM64 Windows\cite{WinARM64}
\end{itemize}

\paragraph{SYS V AMD64 unwind library}
is based on DWARF CFI (call frame information) that are stored in the \emph{.eh\_frame} section.
Unwinding happens under following circumstances:
\begin{itemize}
    \item A C++ exception has been thrown
    \item unwinding is forced by an external agent (longjmp for instance)
\end{itemize}
\uwforced takes a \foreignex (non-C++ exception; for instance Java or GO) and walks the stack frame by frame
inspecting the \emph{unwind tables} for cleanup functions (for instance destructors of
local variables) and catch blocks.\\

\uwforced calls a \emph{personality routine} (\cpp{__gxx_personality_v0()} forGCC)\footnote{The
personality routine passed by a specific runtime serves as interface between system unwinding library
and language specific exception handling (not only C++; GO and Java are also supported). It is always invoked via pointer (saved
as a function pointer in \ehframe for each stack frame).}
\uwforced takes a stop function that controls the termination of the unwinding
(reaching end of stack for fibers).
The stop function intercepts calls to the personality routine, letting the external
agent override the defaults of the stack frame's personality routine.\footnote{As
a consequence the C++ personality routine deals only with C++ exceptions;
it does not need to know anything specific about unwinding done by an external
agent such as fiber or pthreads cancellation.}
When the destination frame (last frame on fiber
stack) is reached, control jumps back to the caller without literally returning.\\

The code snippet below is a proof of concept available at \href{https://github.com/boostorg/context/tree/p0876r6}{Boost.Context branch p0876r6}.
\cppf{unwind}
\cpp{fiber_unwind()} is called by \unwindfib or \dtor and starts the stack unwinding.
The foreign exception \cpp{foreign_unwind_ex}\footnote{setting member variable makes \cpp{foreign_unwind_ex} a foreign exception}
is allocated and passed as parameter to the unwinding library. Function \cpp{fiber_unwind_stop()} transfers execution control
to the calling fiber once the last stack frame has been unwound.

\subparagraph{non-catchable \foreignex}
\unwindfib uses a non-C++ \foreignex to force stack unwinding.
As stated in the \emph{SYS V AMD64 ABI}\cite{SYSVAMD64} standard:
"A runtime is not allowed to catch an exception if the \cpp{_UA_FORCE_UNWIND} flag was passed to the personality routine."
and "... since it is not possible to determine if a given catch clause will re-throw or not without executing it ...", the
\foreignex must not be catchable by C++ \cpp{try-catch} blocks.\\
As a consequence, \curex can not return a \cpp{std::exception\_ptr} pointing to a \foreignex.\\
In order to detect if stack unwinding is currently in progress \uncex returns \cpp{true} and\\
\uncexs counts the \foreignex.\\

The rationale for moving to an uncatchable exception is further explained in
the \nameref{history}.\\

The specific characteristics of a \foreignex:

\begin{itemize}
    \item Throwing the \foreignex can only be effected by the \fiber
    facility. The proposed \unwindfib is the only way to cause that
    explicitly.
    \item The ultimate "catch" -- the point at which stack unwinding stops --
    is likewise determined by the \fiber facility. There is no explicit syntax
    for this.
    \item Along the way, as with a normal C++ exception, every object in every
    stack frame is destroyed.
%%  \item \catchall clauses along the way are executed, but:
%%  \begin{itemize}
%%      \item \cpp{throw;} resumes stack unwinding, as usual
%%      \item a \catchall clause that does not execute a \cpp{throw;}
%%      statement behaves as if it ends with a \cpp{throw;} statement
%%      \item a \catchall clause that attempts to throw a normal C++
%%      exception engages Undefined Behavior
%%  \end{itemize}
%%  \item \cpp{catch (}\emph{anything else}\cpp{)}
    \item \cpp{catch} clauses along the way are ignored. This is
    what is meant by the shorthand "uncatchable."
\end{itemize}

Since unwinding a fiber's stack requires destroying objects declared in stack
frames,
%% and may involve executing \catchall clauses,
it is worth pointing out that destroying a non-empty \fiber on a thread other
than the thread on which it was last resumed will run those object destructors
%% and \catchall clauses
on the thread destroying the \fiber instance.\\

As a consequence, destroying a \fiber instance representing an implicit fiber
from any other thread engages Undefined Behavior.\footnote{One unobvious case
would be if a fiber running on non-\main thread \cpp{T} stores a \fiber
representing \cpp{T}'s default fiber in a static variable, whether
module-scope or function-scope. That variable will be destroyed at program
termination, probably on a thread other than \cpp{T}.}\\

Otherwise:

\subparagraph{destroying a \fiber representing \main}\label{destroymain} is equivalent to terminating the application.

The C++ runtime must inject a stack frame above \main that serves to stop
stack unwinding due to the \foreignex. This is similar to, but different than,
the stack frame above the \entryfn for an explicitly launched fiber.

\unwindfib must always be called with a non-empty \fiber instance. That \fiber
value is cached while the terminated fiber's stack is unwound.

As its last act, the top-level stack frame on an explicitly launched fiber
resumes the cached \fiber.

In contrast, the stack frame injected above \main must finally \emph{destroy}
the cached \fiber.

It is not reasonable to unwind (e.g.) the stack belonging to \main to a
certain point and then simply abandon it, leaving it unreachable.\footnote{A
higher-level library built on this facility can introduce a scheduler. A
scheduler could permit abandoning the \main fiber while continuing to run
other fibers on the same thread; once the last of those fibers terminates, the
scheduler can finally return from the \main fiber. But \fiber explicitly
avoids introducing a scheduler.} Destroying the \fiber representing \main must
end by returning to the runtime.

Consider fiber F0: the default thread's default fiber, the fiber on
which \main is called. F0 launches fiber F1, storing F1's \fiber in a
stack variable, then launches F2. F2 destroys the \fiber instance representing
F0.

F0's stack is unwound with F2's \fiber instance cached.

At some point along the way, the stack frame containing F1's \fiber is
destroyed, thus F1's \fiber is destroyed. F0's stack unwinding is temporarily
suspended and F1's stack is unwound. During this nested unwinding, F0's \fiber
is cached.

Once F1's stack is unwound -- because F1 was explicitly launched -- its
implicit top-level function resumes F0's \fiber. This frees F1's stack memory
and resumes unwinding F0's stack.

Once F0's stack is unwound, as noted above, its top-level function must return
to the runtime to terminate the process. However, F2 is still suspended at the
point of destroying F0's (original) \fiber instance! F2's stack still exists!

Therefore, F0's top-level function must destroy F2's cached \fiber instance
before returning to the runtime.

\subparagraph{destroying a \fiber representing \cpp{std::thread}} is
equivalent to exiting the thread.

The reasoning is the same as for the \fiber representing \main.

\zs{The system's exception handling, i.e. its unwinding framework, is used to cleanup the stack
of a fiber by using a foreign exception that is not catchable by C++ \cpp{try-catch} blocks.}

