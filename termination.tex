\abschnitt{termination}

There are a few different ways to terminate a given fiber without
terminating the whole process, or engaging undefined behavior.

\begin{itemize}
    \item Return a valid fiber from the \entryfn.
    \item Call \unwindfib with a valid fiber. This throws a \unwindex
          instance that binds that fiber.
    \item Then another supported way would be to construct and throw \unwindex
          ``by hand,'' which is what \unwindfib does internally.
    \item Call \cpp{fiber::resume\_with(unwind\_fiber)}.
          This is what \dtor does. Since \unwindfib accepts a \fiber, and
          since \resumewith synthesizes a \fiber and passes it to the subject
          function, this terminates the fiber referenced by the original
          \fiber instance and switches back to the caller.
    \item Engage \dtor: switch to some other fiber, which will
          receive a \fiber instance representing the current fiber. Make that
          other fiber destroy the received \fiber instance.
\end{itemize}

When the \entryfn invoked by \resume returns a valid \fiber instance, the
running fiber is terminated. Control switches to the fiber indicated by the
returned \fiber instance.\\

Returning an invalid \fiber instance (\opbool returns \cpp{false}) invokes
undefined behavior.\\

If the \entryfn returns the same \fiber instance it was originally passed (or
rather, the most recently updated \fiber returned from the previous instance's
\resume), control returns to the fiber that most recently resumed the running
fiber. However, the \entryfn may return (switch to) any reachable valid \fiber
instance.\\

\emph{Calling} \resume means: ``Please switch to the indicated fiber; I
am suspending; please resume me later.''\\

\emph{Returning} a particular \fiber means: ``Please switch to the indicated
fiber; and by the way, I am done.''
