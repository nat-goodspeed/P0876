\newpage
\addcontentsline{toc}{subsection}{references}
\begin{thebibliography}{99}

    \bibitem{SYSVAMD64}
        \href{https://software.intel.com/sites/default/files/article/402129/mpx-linux64-abi.pdf}{SYS V AMD64 unwinding}

    \bibitem{WinX64}
        \href{https://docs.microsoft.com/en-us/cpp/build/exception-handling-x64?view=vs-2019}{x64 Windows unwinding}

    \bibitem{WinARM64}
        \href{https://docs.microsoft.com/en-us/cpp/build/arm64-exception-handling?view=vs-2019}{ARM64 Windows unwinding}

    \bibitem{OpenAcc}
        {Chandrasekaran, Sunita and Juckeland, Guido (2018). "OpenACC for Programmers: Concepts and Strategies", (1st ed.).
         Pearson Education, Inc}

    \bibitem{CUDA}
        {Wilt, Nicolas (2013). "The CUDA Handbook: A Comprehensive Guide to GPU Programming", (1st ed.).
         Addison Wesley}

    \bibitem{Tanenbaum2009}
        {Tannenbaum, Andrew S. (2009). "Operating Systems. Design and Implementation", (3rd ed.).
         Pearson Education, Inc}

    \bibitem{Moura2009}
        \href{http://www.inf.puc-rio.br/~roberto/docs/MCC15-04.pdf}
        {Moura, Ana L\'{u}cia De and Ierusalimschy, Roberto. "Revisiting coroutines".
         ACM Trans. Program. Lang. Syst., Volume 31 Issue 2, February 2009, Article No. 6}

    \bibitem{N3985}
        \href{http://isocpp.org/files/papers/n3985.pdf}
        {N3985: A proposal to add coroutines to the C++ standard library}

    \bibitem{Standard}
        \href{https://www.open-std.org/jtc1/sc22/wg21/docs/papers/2022/n4917.pdf}
        {N4917: Working Draft, Standard for Programming Language C++}

    \bibitem{P0099R0}
        \href{http://www.open-std.org/jtc1/sc22/wg21/docs/papers/2015/p0099r0.pdf}
        {P0099R0: A low-level API for stackful context switching}

    \bibitem{P0099R1}
        \href{http://www.open-std.org/jtc1/sc22/wg21/docs/papers/2016/p0099r1.pdf}
        {P0099R1: A low-level API for stackful context switching}

    \bibitem{P0534R3}
        \href{http://www.open-std.org/jtc1/sc22/wg21/docs/papers/2017/p0534r3.pdf}
        {P0534R3: call/cc (call-with-current-continuation): A low-level API for stackful
        context switching}

    \bibitem{P0660R10}
        \href{https://www.open-std.org/jtc1/sc22/wg21/docs/papers/2019/p0660r10.pdf}
        {P0660R10: Stop Tokens and a Joining Thread}

    \bibitem{P0709R4}
        \href{https://www.open-std.org/jtc1/sc22/wg21/docs/papers/2019/p0709r4.pdf}
        {P0709R4: Zero-overhead deterministic exceptions: Throwing values}

    \bibitem{P0772R1}
        \href{https://www.open-std.org/jtc1/sc22/wg21/docs/papers/2018/p0772r1.pdf}
        {P0772R1: Execution Agent Local Storage}

    \bibitem{P0876R0}
        \href{http://www.open-std.org/jtc1/sc22/wg21/docs/papers/2018/p0876r0.pdf}
        {P0876R0: fibers without scheduler}

    \bibitem{P0876R2}
        \href{http://www.open-std.org/jtc1/sc22/wg21/docs/papers/2018/p0876r2.pdf}
        {P0876R2: fibers without scheduler}

    \bibitem{P0876R3}
        \href{http://www.open-std.org/jtc1/sc22/wg21/docs/papers/2018/p0876r3.pdf}
        {P0876R3: fibers without scheduler}

    \bibitem{P0876R5}
        \href{http://www.open-std.org/jtc1/sc22/wg21/docs/papers/2019/p0876r5.pdf}
        {P0876R5: fibers without scheduler}

    \bibitem{P0876R6}
        \href{http://www.open-std.org/jtc1/sc22/wg21/docs/papers/2019/p0876r6.pdf}
        {P0876R6: fibers without scheduler}

    \bibitem{D0876R7}
        \href{http://wiki.edg.com/pub/Wg21cologne2019/SG1/D0876R7.pdf}
        {D0876R7: fibers without scheduler}

    \bibitem{P0876R8}
        \href{http://www.open-std.org/jtc1/sc22/wg21/docs/papers/2019/p0876r8.pdf}
        {P0876R8: fibers without scheduler}

    \bibitem{P0876R9}
        \href{http://www.open-std.org/jtc1/sc22/wg21/docs/papers/2019/p0876r9.pdf}
        {P0876R9: fibers without scheduler}

    \bibitem{P0876R10}
        \href{https://www.open-std.org/jtc1/sc22/wg21/docs/papers/2020/p0876r10.pdf}
        {P0876R10: fibers without scheduler}

    \bibitem{P1677R2}
        \href{https://www.open-std.org/jtc1/sc22/wg21/docs/papers/2019/p1677r2.pdf}
        {P1677R2: Cancellation is serendipitous-success}

    \bibitem{P1820R0}
        \href{https://www.open-std.org/jtc1/sc22/wg21/docs/papers/2019/p1820r0.html}
        {P1820R0: Recommendations for a compromise on handling errors and cancellations in executors}

    \bibitem{P2175R0}
        \href{https://www.open-std.org/jtc1/sc22/wg21/docs/papers/2020/p2175r0.html}
        {P2175R0: Composable cancellation for sender-based async operations}

    \bibitem{coreguidlines}
        \href{http://isocpp.github.io/CppCoreGuidelines/CppCoreGuidelines#Ri-global}
        {C++ Core Guidelines}

    \bibitem{SYSVABI}
        \href{http://software.intel.com/sites/default/files/article/402129/mpx-linux64-abi.pdf}
        {System V Application Binary Interface AMD64 Architecture Processor
        Supplement}

    \bibitem{bcontext}
        \href{http://www.boost.org/doc/libs/release/libs/context/doc/html/index.html}
        {Library \emph{Boost.Context}}

    \bibitem{bcoroutine2}
        \href{http://www.boost.org/doc/libs/release/libs/coroutine2/doc/html/index.html}
        {Library \emph{Boost.Coroutine2}}

    \bibitem{bfiber}
        \href{http://www.boost.org/doc/libs/release/libs/fiber/doc/html/index.html}
        {Library \emph{Boost.Fiber}}

    \bibitem{fbmcrouter}
        \href{https://code.facebook.com/posts/296442737213493/introducing-mcrouter-a-memcached-protocol-router-for-scaling-memcached-deployments}
        {Facebook's \emph{mcrouter}}

    \bibitem{fbthrift}
        \href{https://github.com/facebook/fbthrift}
        {Facebook's \emph{Thrift}}

    \bibitem{fbfiber}
        \href{https://github.com/facebook/folly/tree/master/folly/fibers}
        {Facebook's \emph{folly::fibers}}

    \bibitem{bbquantum}
        \href{https://github.com/bloomberg/quantum}
        {Bloomberg's \emph{quantum}}

    \bibitem{habanero}
    \href{https://wiki.rice.edu/confluence/display/HABANERO/Habanero+Extreme+Scale+Software+Research+Project}
        {Habanero Extreme Scale Software Research Project}

    \bibitem{hclib}
    \href{https://github.com/habanero-rice/hclib}
        {Habanero HClib}

    \bibitem{synca}
        \href{https://github.com/gridem/Synca}
        {Library \emph{Synca}}

    \bibitem{tbb}
        \href{https://github.com/intel/tbb}
        {Intels's \emph{TBB}}

\end{thebibliography}
