\newpage
\abschnitt{Appendix B: throw-expression with no operand}\label{throw}

Both \stdclause{expr.throw} paragraph 3 and \cpp{current\_exception()}
(\stdclause{propagation} paragraph 8) reference the ``currently handled
exception'' (\stdclause{except.handle} paragraph 8). Thus, the
construct \cpp{throw;} is by definition equivalent to\\
\cpp{std::rethrow\_exception(std::current\_exception());}
(\stdclause{propagation} paragraph 9).

The existing definition of currently handled exception:

``The exception with the most recently activated handler that is still active
is called the \emph{currently handled exception.}''

does not clearly constrain the scope to the current thread. This constraint
must be inferred from \stdclause{except.throw} paragraph 2:

``When an exception is thrown, control is transferred to the nearest handler
with a matching type \xref{except.handle}; ``nearest'' means the handler for
which the \nt{stmt.block}{compound-statement} or
\nt{class.base.init}{ctor-initializer} following the \cpp{try} keyword was
most recently entered by the thread of control and not yet exited.''

This is the reason for the proposed changes to \stdclause{except}.
If ``currently handled exception'' means the exception with the
most recently activated handler within any fiber on the current thread, we can get
\href{https://github.com/secondlife/3p-boost/blob/nat/exstate/tests/nullary_throw.cpp}{the following result}.

\cppfl{nullary_throw}

Worse still, the exceptions in question aren't necessarily related to each
other, and line 36 is more likely to read \cpp{catch (const Bad& caught)} --
in which case the \cpp{throw;} on line 34 would \emph{not} be caught.
