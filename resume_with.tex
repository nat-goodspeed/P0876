\abschnitt{inject function into suspended fiber}\label{resumewith}
Sometimes it is useful to inject a new function (for instance, to throw an
exception or assigning the synthesized fiber to the caller as described in
\nameref{fiberreturn}) into a suspended fiber. For this purpose
\cpp{resume\_with(Fn&& fn)} may be called, passing the function \cpp{fn()} to
execute.\\
If the fiber represented by the \fiber instance \cpp{f} has called a function
\cpp{suspender()}, which has called \resume and is now suspended, function
\cpp{fn()} is injected into fiber \cpp{f} as if \cpp{suspender()}'s \resume call
had directly called \cpp{fn()}.\\

Like an \entryfn\xspace passed to \fiber, \cpp{fn()} must accept
\cpp{std::fiber&&} and return \fiber. The \fiber instance returned by \cpp{fn()}
will, in turn, be returned to \cpp{suspender()} by \resume.\\

Suppose that code running on the program's main fiber calls \resume, thereby
entering the first lambda shown below. This is the point at which \cpp{m} is
synthesized and passed into the lambda at (2).\\
Suppose further that after doing some work (4), the lambda calls
\cpp{m.resume()}, thereby switching back to the main fiber. The lambda remains
suspended in the call to \cpp{m.resume()} at (8).\\
At (18) the main fiber calls \cpp{f.resume\_with()} where the passed lambda
accepts \cpp{fiber &&}. That new lambda is entered in the fiber of the suspended
lambda. It is as if the \cpp{m.resume()} call at (8) directly called the second
lambda.\\

The function passed to \resumewith has almost the same range of possibilities as
any function called on the fiber represented by \cpp{f}. Its special invocation
matters when control leaves it in either of two ways:

\begin{enumerate}
  \item If it throws an exception, that exception unwinds all previous stack
        entries in that fiber (such as the first lambda's) as well, back to
        a matching \cpp{catch} clause.\footnote{As stated
        in \nameref{exceptions}, if there is no matching \cpp{catch}
        clause in that fiber, \cpp{std::terminate()} is called.}
  \item If the function returns, the returned \fiber instance is returned by
        the suspended \cpp{m.resume()} (or \entryfn, or \resumewith) call.
\end{enumerate}

\cppfl{ontop}

The \cpp{f.resume\_with(<lambda>)} call at (18) passes control to the second
lambda on the fiber of the first lambda.\\

As usual, \resumewith synthesizes a \fiber instance representing the calling
fiber, passed into the lambda as \cpp{m}. This particular lambda returns \cpp{m}
unchanged at (21); thus that \cpp{m} instance is returned by the \resume call
and assigned at (8).\\

Finally, the first lambda returns at (10) the \cpp{m} variable updated at (8),
switching back to the main fiber.\\

\zs{Member function \resumewith allows to inject a function into suspended
fiber.}
