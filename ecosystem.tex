\abschnitt{\fiber and the larger C++ ecosystem}

\uabschnitt{higher-level libraries}

\nameref{low_level} enumerates a number of higher-level abstraction libraries
built upon the \bcontext\xspace implementation of the API proposed in this paper.
This is not an exhaustive list, but it suffices to illustrate that there is
widespread interest in this functionality.

The most significant point about this proposal is that, given \fiber, all
those libraries can be written in standard C++. They need not themselves be
integrated into the Standard.

Because it creates and switches between different function call stacks,
though, the \fiber facility cannot be written in portable C++. There is real
value to integrating this library into the Standard.

\emph{Boost.Context} is maintained by one individual to support the specific
set of processors and operating systems to which he has access. The \fiber
facility will ensure support in every implementation of the C++ runtime,
extending into the future.

Given the lively ecosystem of open-source libraries, it's possible that
standardizing \fiber could suffice. It is not essential that
WG21 must standardize additional higher-level libraries before the facility
would become useful. The uptake of \emph{Boost.Context} illustrates that the
community can make good use of \fiber.

However, the evolution of this proposal and the WG21 discussions thereof have
surfaced a number of interesting adjacencies.

\uabschnitt{cancellation}

Given C++ support for concurrency, in various forms, within a program,
cancellation of an asynchronous task remains a topic of widespread interest.
It has been much discussed, e.g. in P1677R2\cite{P1677R2},
P1820R0\cite{P1820R0} and P2175R0\cite{P2175R0}.

Previous revisions of this paper have proposed canceling a suspended fiber by
injecting an exception, e.g. using \fiber\cpp{::}\resumewith. A comparable
approach was rejected for \cpp{std::jthread}, although it's worth noting that
cooperative fibers differ in a very significant respect: every fiber suspends
at a well-defined point, namely a call to \resumewith.\footnote{Although
exception-based cancellation is not implicitly supported, a consumer of \fiber
may still explicitly pass to \resumewith an invocable that raises an exception
in the suspended fiber.}

Evolution of the exception mechanism itself\cite{P0709R4} may affect the
viability of using exceptions for cancellation.

This paper simply notes that an invoker can use lambda binding to pass (e.g.)
a \cpp{std::stop\_token} from the Standard\cite{Standard}, section 33.3, to a
fiber at launch time.

\uabschnitt{modules and optimizations}

Before modules, the only information the compiler could know about a function
in an external translation unit was what a human coder stated in the relevant
header file. But since the information in a module is prepared by the compiler
itself, a subsequent compile of a translation unit that imports that module
can know as much about each module function as it would if the function's
source code was found within the current translation unit.

This permits the compiler to infer and propagate attributes. If a function
neither contains a throw statement nor calls other functions, the compiler can
conclude that it doesn't throw. It can encode this information in the module
produced for that translation unit, so that subsequent compiles can make use
of the knowledge. If another function contains no throw statement and calls
only functions known not to throw, it too can be implicitly marked nothrow.

Similarly, when compiling a function that can never return, the compiler can
so indicate in the output module. Any caller whose code path leads
unconditionally to any such function can also be known never to return.

In much the same way, the module describing the
library's \fiber\cpp{::}\resumewith method can mark it as \emph{can-suspend}.
Then any caller of \resumewith will also be marked \emph{can-suspend}, and so
forth. The compiler can use this to improve its optimization tactics around
any call to a \emph{can-suspend} function.

(The \emph{can-suspend} characteristic of a \cpp{co\_await} coroutine function
is just as pervasive, but in that case the coder must manually propagate it.)

\uabschnitt{synchronization primitives}

The Standard\cite{Standard} provides an assortment of primitives for
synchronizing work between threads, e.g. sections 33.6, 33.7, 33.8, 33.9,
33.10. An essential behaviour of many such synchronization primitives is to
pause, or suspend, execution of the current thread until some external
condition is satisfied.

Such suspension is very different from fiber suspension as proposed in this
paper. This proposal neither requires nor implies a scheduler. A fiber
suspends by explicitly designating the next fiber to resume, either by passing
its \fiber to \resumewith or by returning that \fiber from its \entryfn.

C++ threads, in contrast, assume a thread scheduler, usually provided by the
operating system. Suspending a thread means passing control to the scheduler,
which reallocates CPU resources to other pending threads. At some future time,
the scheduler is responsible for directing some CPU core to resume the suspended
thread.

Fiber suspension as implemented by \fiber is independent of thread suspension.
Suspending the running fiber simply means directing the thread to run a
different fiber; the thread continues running. Conversely, suspending the host
thread (e.g. by invoking a synchronization primitive) means that \emph{no}
fiber is running on that thread.

A higher-level fiber-based library that emulates the \cpp{std::thread} API,
such as \bfiber\cite{bfiber}, necessarily implements a fiber scheduler,
permitting implicit fiber suspension. Standardizing such a library would raise
the interesting question of how to present fiber-aware synchronization
primitives.

A straightforward approach is to present a suite of fiber-aware
synchronization primitives distinct from, but analogous to, the thread-based
synchronization primitives.\footnote{This is the approach taken
by \emph{Boost.Fiber}.} A program running multiple fibers within a thread
would use fiber-aware synchronization primitives rather than thread-based
synchronization primitives. Evaluating a thread-based synchronization
primitive would suspend the entire thread, as usual, halting all fibers within
that thread.

It is tempting to contemplate modifying the semantics of the present suite of
synchronization primitives to make them fiber-aware. Naturally this is a
matter of some concern.

For purposes of this \fiber proposal, though, it is entirely moot.

\uabschnitt{Execution Agent Local Storage}

A similar question arises concerning variable storage duration. Should the
Standard introduce a fiber-specific storage duration, e.g. \cpp{fiber\_local},
analogous to \cpp{thread\_local}\cite{Standard}? (section 6.7.5.3 \bfs{Thread
storage duration})

The Standard defines the general term \emph{execution agent} (section
33.2.5.1) to allow for multiple kinds of parallelism. It seems reasonable to
assume that over time, new types of execution agents will be defined. Will we
want the Standard to present a new \cpp{xyz\_local} storage duration for each
new ``xyz'' execution agent type?

P0772R1\cite{P0772R1} notes that library code should not have to care what
kind of execution agent is running it. Already it's important to ensure that
library code avoids \cpp{static} variables because any such variable prohibits
calling that library from more than one thread. P0772R1 suggests a generalized
variable storage duration dynamically local to the innermost current execution
agent.

(The same consideration about library code impacts the above question about
presenting fiber-aware synchronization primitives.)

It's true that if:

\begin{itemize}
    \item on fiber X, function F relies on a \cpp{thread\_local} variable V
    \item function F calls function G that resumes fiber Y
    \item fiber Y calls function F, or another function that modifies variable V
    \item fiber Y resumes fiber X
    \item on fiber X, function G returns to function F
\end{itemize}

then function F on fiber X will observe fiber Y's value for variable V.

This is analogous to use of a \cpp{static} variable by multiple threads in the
same program -- though not as bad, since it doesn't produce race-related
Undefined Behaviour on top of correctness problems.

\cpp{std::thread} was introduced despite this problem because it's \emph{useful.}

Multiple C++ implementations cache a pointer to thread-local storage in the
stack frame of a function referencing TLS. If a suspended fiber were resumed
by a thread other than the one on which it previously ran, such cached TLS
pointers would point to TLS for the wrong thread. This is why such
cross-thread resumption is forbidden.

(This is the only optimization that has yet been surfaced by implementers as a
potentially problematic interaction with fibers.)

P3346R0\cite{P3346R0} proposed to modify \tlocal to mean fiber-specific. This
was rejected by SG1 in Wrocław in 2024\cite{wroclawp3346}.

That said, in an environment in which \tlocal referenced fiber-specific
storage, TLS pointers cached in function stack frames would remain valid even
if the original fiber were later resumed on some other thread, thus removing
the restriction against cross-thread resumption.

\uabschnitt{tooling} One particularly valuable consequence of adding \fiber to
the Standard will be to add fiber awareness to debuggers, performance
analyzers and other tools that inspect a running C++ program.

Such tools need only be aware of \fiber. They would \emph{not} need to be
further adapted to support higher-level libraries built on
the \fiber facility.

\newpage
