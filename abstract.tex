\abschnitt{abstract}

This paper addresses concerns, questions and suggestions from the past meetings.
The proposed API supersedes the former proposals N3985\cite{N3985},
P0099R1\cite{P0099R1}, P0534R3\cite{P0534R3}, P0876R0\cite{P0876R0}
and P0876R2\cite{P0876R2}.\\
Because of name clashes with \emph{coroutine} from coroutine TS,
\emph{execution context} from executor proposals and \emph{continuation} used
in the context of \cpp{future::then()}, the committee has indicated
that \emph{fiber} is preferable. However, given the foundational, low-level
nature of this proposal, we choose \emph{fiber\_context}, leaving the
term \emph{fiber} for a higher-level facility built on top of this one.\\

A previous revision of this proposal suggested \emph{fiber\_context}, but SG1
felt that use of the term ``context'' posed potential confusion. The name 
\emph{basic\_fiber} has also been suggested. Consider that the name is not yet
final.\\

The minimal API enables stackful context switching \bfs{without} the need for a
\bfs{scheduler}. The API is suitable to act as building-block for high-level
constructs such as stackful coroutines as well as cooperative multitasking
(aka user-land/green threads that incorporate a \bfs{scheduling facility}).\\

Informally within this proposal, the term \emph{fiber} is used to denote the
lightweight thread of execution launched and represented by the first-class
object \fiber.
