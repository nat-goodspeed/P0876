\abschnitt{abstract}\label{abstract}

This paper proposes a
minimal API that enables stackful context switching \bfs{without} the need for a
\bfs{scheduler}. The API is suitable to act as building-block for high-level
constructs such as stackful coroutines as well as cooperative multitasking
(aka user-land/green threads that incorporate a \bfs{scheduling facility}).

This revision addresses concerns, questions and suggestions from the past meetings.
The proposed API supersedes the former proposals N3985\cite{N3985},
P0099R1\cite{P0099R1}, P0534R3\cite{P0534R3} and P0876R20\cite{P0876R20}.

Because of name clashes with \emph{coroutine} from C++20,
\emph{execution context} from executor proposals and \emph{continuation} used
in the context of \cpp{future::then()}, the committee has indicated
that \emph{fiber} is preferable. However, given the foundational, low-level
nature of this proposal, we choose \emph{fiber\_context}, leaving the
term \emph{fiber} for a higher-level facility built on top of this one.

Informally within this proposal, the term \emph{fiber} is used to denote the
flow of control launched and represented by the first-class object \fiber.

It's telling that when Hana Dusikova was working on implementations of P3367R3
constexpr coroutines\cite{P3367R3}, the ``easiest way to model a coroutine,''
the ``obvious first choice,'' was to use fibers in the constexpr evaluator.
